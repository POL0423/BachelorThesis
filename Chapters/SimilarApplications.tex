\chapter{Podobné aplikace}
Prostým položením dotazu do Google vyhledávače jsem byl schopen najít
informace k již existujícím podobným aplikacím. Jedná se zejména o články
porovnávající jednotlivé aplikace mezi sebou.

\section{iPumpuj a Pumpdroid}
Tato aplikace poskytuje informace o cenách PHM na čerpacích stanicích.
Kromě toho poskytuje i detailní informace o dané ČS, včetně otevírací
doby a informací o případných pokutách od ČOI. Verze pro Android
se nazývá \emph{Pumpdroid}, zatímco uživatelé iOS aplikaci najdou
pod názvem \emph{iPumpuj}.
\cite{Vrablova2022}
\cite{Sarikova2021}

\section{mBenzin.cz}
Nejedná se o mobilní aplikaci, ale o web, na kterém lze vyhledávat
nejbližší a nejlevnější ČS s možností notifikací na motoristické
novinky. Web ukazuje také průměrné ceny PHM v Česku i v Evropě.
Magazín také poskytuje spojnicové grafy ukazující změnu cen
v průběhu jednotlivých dnů a měsíců.
\cite{Vrablova2022}

\section{Waze}
Tato aplikace je primárně určena k navigaci, ale uživatelům poskytuje
také přehled cen PHM. Ve vyhledávacím poli stačí vybrat symbol ČS.
Aplikace se tím transformuje na srovnávač cen PHM a automaticky zobrazí
ČS v okolí s preferovaným typem paliva společně s aktuální cenou.
\cite{Vrablova2022}

\section{Mapy.cz}
Známý český vyhledávač Seznam.cz je autorem aplikace Mapy.cz, která
primárně slouží pro navigaci a orientaci. Tato aplikace je však také
schopna srovnat ceny PHM, podobně jako Waze.
\cite{Vrablova2023}

\section{Evropské aplikace}
Podobné aplikace jsou dostupné také pro ostatní evropské země. Jednou
takovou může být aplikace \emph{Mehr-tanken}, která je dostupná
pro uživatele iOS i Androidu. Tato aplikace získává data od uživatelů
a také od autorizované jednotky hlídající transparentnost trhu.
Aplikace podporuje i notifikace oznamující, kdy některá z oblíbených
ČS nebo ČS v okolí významně sníží ceny PHM. Aplikace také nabízí
srovnávač nabíjecích stanic pro elektromobily a vodíkových ČS.
\cite{r6fadX3YRnFIir68}

Aplikaci k porovnávání PHM vyvinul a zdarma publikoval také automobilový
svaz ADAC pod názvem \emph{ADAC Spritpreise}. Aplikace je vhodná pro rychlý
přehled cen PHM.
\cite{r6fadX3YRnFIir68}

Další evropskou aplikací je také např. \emph{PACE Drive}, která
porovnává řadu typů PHM a aplikace je bez reklam dostupná zdarma
pro uživatele iOS a Androidu. Porovnávat lze ceny PHM v Německu,
Španělsku, Francii, Itálii a Portugalsku.
\cite{r6fadX3YRnFIir68}

\section{Aplikace tankovacích karet}
Držitelé tankovacích karet mohou také využít aplikace vydavatelů těchto
karet, které mají v sobě zabudovaný srovnávač cen PHM na ČS podporující
tyto tankovací karty. Jedním takovým vydavatelem je například CCS, který
nabízí aplikaci s názvem \emph{Tank Navigator}. Aplikaci lze zdarma
používat na iOS i Androidu, a nabízí vyhledávání ČS v okolí dle vzdálenosti,
zobrazení ČS na mapě, navigaci k vybrané ČS, filtraci ČS dle značky řetězce
a dostupných služeb, a další.
\cite{Khcm5FZT2rH5pABQ}

\endinput
