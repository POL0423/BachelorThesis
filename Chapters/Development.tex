\chapter{Vývoj aplikace}

Samotná implementace celé aplikace je rozdělena na několik částí.

\begin{itemize}
    \item \textbf{Crawler:} Sbírá data a zapisuje je do relační databáze.
        Využit je software \emph{Crawlee} s knihovnou \emph{Playwright Chromium}.
        K geolokaci jednotlivých stanic je použit API mapového poskytovatele OSM.
    \item \textbf{Frontend:} Čte data z databáze, filtruje na základě
        uživatelského vstupu geolokace a zvoleného okruhu a filtrovaná
        data následně porovnává a řadí ve výchozím nastavení od nejlevnějších
        po nejdražší. Využit je framework \emph{Next.js} s použitím API
        mapového poskytovatele OSM.
    \item \textbf{Databáze:} Uchovává nasbíraná data z crawleru a poskytuje je
        frontendu k uživatelskému zpracování. Využit je software \emph{Oracle MySQL}.
    \item \textbf{Reverse proxy:} Zprostředkovává přístup k frontendu z webového
        prohlížeče uživatele a zajišťuje kryptografické šifrování komunikace
        mezi prohlížečem a serverem. Využit je software \emph{Nginx}.
\end{itemize}

\section{Crawler}

Implementace crawleru je řešena pomocí \emph{Node.js}, který poskytuje funkční
rozhraní pro spouštění aplikace, s využitím knihoven \emph{Crawlee},
\emph{Playwright} a \emph{MySQL}, které poskytují rozhraní nutné k procházení
webů a zapisování do databáze. Nejdůležitější součástí crawleru jsou dílčí
moduly, které implementují procházení webů. Technický postup automatizace
tohoto procházení a čtení dat je popsán v následujících podsekcích.

\subsection{Procházení Globusu}

\begin{enumerate}
    \item Po načtení webové stránky se crawler snaží identifikovat tlačítko
        souhlasu s využitím cookies. Tlačítko souhlasu je popsáno CSS selektorem\\
        \texttt{button\#CybotCookiebotDialogBodyLevelButtonLevelOptinAllowAll}.
        Pokud není toto tlačítko nalezeno, procedura se přeskočí.
    \item Nyní je třeba zjistit seznam jednotlivých poboček kliknutím na tlačítko
        \emph{Vybrat pobočku}. Toto tlačítko se nachází v hlavičce stránky,
        popsáno je CSS selektorem \texttt{div\#\_\_nuxt header\#header button.btn-lg}.
    \item [...]
\end{enumerate}

\subsection{Procházení ČS ONO}

[...]

\section{Frontend}

[...]

\section{Ostatní součásti a jejich propojení}

[...]

\endinput
