\chapter{Vlastní vývoj aplikace}
\label{ch:development}

Tato kapitola se~zabývá již samotným vývojem a~implementací aplikace,
kde~jsou popsány technické postupy a~algoritmy crawleru, součásti
webového frontendu aplikace a~jejich funkce, a~následně spojení jednotlivých
částí do~jednoho celku pomocí soustavy Docker kontejnerů a~finální spuštění
na~serveru. Během vývoje jsem použil umělou inteligenci k~řešení chyb
a~potíží, jedná~se o~GitHub Copilot \cite{shCgjU2g2YA0ff8T}. %Ref: GHCopilot

Samotná implementace celé aplikace je~rozdělena na~několik částí.

\begin{itemize}
    \item \textbf{Crawler:} Sbírá data a~zapisuje~je do~relační databáze.
        Využit je~software \emph{Crawlee} s~knihovnou \emph{Playwright
        Chromium}. K~geolokaci jednotlivých stanic je~použit API mapového
        poskytovatele OSM.
    \item \textbf{Frontend:} Čte data z~databáze, filtruje na~základě
        uživatelského vstupu geolokace a~zvoleného okruhu a~filtrovaná
        data následně porovnává a~řadí ve~výchozím nastavení od~nejlevnějších
        po~nejdražší. Využit je~framework \emph{Next.js} s~použitím API
        mapového poskytovatele OSM.
    \item \textbf{Databáze:} Uchovává nasbíraná data z~crawleru a~poskytuje~je
        frontendu k~uživatelskému zpracování. Využit je~software
        \emph{Oracle MySQL}.
    \item \textbf{Reverse proxy:} Zprostředkovává přístup k~frontendu
        z~webového prohlížeče uživatele a~zajišťuje kryptografické
        šifrování komunikace mezi prohlížečem a~serverem. Využit je~software
        \emph{Nginx}.
\end{itemize}

\section{Crawler}
\label{sec:development-crawler}

Implementace crawleru je~řešena pomocí \emph{Node.js}, který poskytuje funkční
rozhraní pro~spouštění aplikace, s~využitím knihoven \emph{Crawlee},
\emph{Playwright} a \emph{MySQL}, které poskytují rozhraní nutné k~procházení
webů a~zapisování do~databáze. Nejdůležitější součástí crawleru jsou dílčí
moduly, které implementují procházení webů. Technický postup automatizace
tohoto procházení a~čtení dat je~popsán v~následujících podsekcích.

Nad~rámec zadání bakalářské práce jsem se~rozhodl mezi PHM zařadit také
doplňkové produkty nabízené v~tankovacích stojanech, které fundamentálně
neslouží jako palivo, ale~řidiče mohou tyto produkty zajímat. Jedná~se
zejména o~provozní kapaliny sloužící k~bezpečnému a~ekologickému provozu
automobilu, jako např. AdBlue (syntetická močovina) nebo kapalina
do~ostřikovačů.

Zdrojový kód této části se~nachází v~datové příloze a~aktuální verze
je~k~dispozici v~GitHub repozitáři
\texttt{https://githubcom/POL0423/petrolscan-crawler}.

\subsection{Procházení Globusu}
\label{sec:crawling-globus}

Během vývoje tohoto crawleru jsem řešil zapeklitou situaci, kterou bych
čtením podrobné dokumentace řešil velmi zdlouhavě. Proto jsem základní
kostru tohoto crawleru nechal vygenerovat umělou inteligencí na~základě
zadaných parametrů (popis činností, které má~crawler vykonat, a~k~tomu
přidružené CSS selektory, které má~crawler zohlednit.
\cite{shCgjU2g2YA0ff8T} %Ref: GHCopilot

\begin{enumerate}
    \item Po~načtení webové stránky se~crawler snaží identifikovat tlačítko
        souhlasu s~využitím cookies. Tlačítko souhlasu je~popsáno CSS
        selektorem\\
        \texttt{button\#CybotCookiebotDialogBodyLevelButtonLevelOptinAllowAll}.
        Pokud není toto tlačítko nalezeno, procedura se~přeskočí.
    \item Nyní je~třeba zjistit seznam jednotlivých poboček kliknutím
        na~tlačítko \emph{Vybrat pobočku}. Toto tlačítko se~nachází
        v~hlavičce stránky, popsáno je~CSS selektorem
        \texttt{div\#\_\_nuxt header\#header button.btn-lg}.
    \item Po~kliknutí na~tlačítko by~se~měl otevřít panel se~seznamem
        prodejen. Pokud se~panel neotevřel, zkouší crawler kliknout
        na~tlačítko znovu. Nepovede-li se to ani tehdy, proces ukončuje.
        Panel je~popsán CSS selektorem \texttt{\#input\_9}.
    \item Po~otevření panelu by se~měl zobrazit seznam prodejen, který
        je~popsán CSS selektorem \texttt{\#input\_9 > ul}. Veškeré lokace
        (popsané CSS selektorem \texttt{\#input\_9 > ul > li}) se~propíšou
        do~pole strukturovaného pomocí objektu v~každé položce, kde~atribut
        \texttt{value} má~hodnotu \texttt{data-option-value} samotné položky
        HTML seznamu (\texttt{\#input\_9 > ul > li}) a~atribut \texttt{name}
        má~hodnotu popisku (\texttt{\#input\_9 > ul > li label}). Atribut
        \texttt{name} je~složen ze~dvou částí:

        \begin{enumerate}
            \item Hlavní popisek
                (\texttt{label span.items-baseline > span.text-base})
            \item Vedlejší popisek
                (\texttt{label span.items-baseline > span.text-xs})
        \end{enumerate}
        
        Vedlejší popisek není vždy přítomen. Pokud je~přítomen, vloží~se
        jeho obsah s~pomlčkou před vedlejším popiskem za~ten hlavní.

        Následující sekvence popisuje proceduru pro~jednu lokalitu (tato
        procedura se~opakuje do~poslední lokality, opakovaná sekvence
        je~uzavřena mezi horizontální čáry).
        \medskip  \hrule
    \item Pro~každou prodejnu se~nyní vytváří nový prohlížeč. V~tomto
        prohlížeči se~znovu načítá webová stránka Globusu. Opět
        se~odsouhlasí cookies. Pokud není dialog s~cookies k~dispozici,
        procedura se~přeskočí. Pokud se~nepovede vytvořit pro~lokaci
        prohlížeč, lokace se~přeskakuje.
    \item Opět se~kliká na~tlačítko \emph{Vybrat pobočku}. Pokud se~to
        nepovede, cyklus končí chybou a~lokace se~přeskočí.
    \item Crawler čeká na~otevření panelu. Neotevře-li se do~20 sekund,
        lokace se~přeskočí. Posléze čeká na~načtení seznamu prodejen.
    \item Zde je~důležitý atribut \texttt{value}, který si~crawler ukládal.
        Jeho hodnota je~totiž shodná s~hodnotou \texttt{data-option-value}
        u~položky lokace v~seznamu. Crawler nyní zkouší kliknout na~popisek
        této položky (\texttt{\#input\_9 > ul >
        li[data-option-value="\$\{location.value\}"] > label},
        kde \texttt{\$\{location.value\}} je~hodnota uloženého atributu
        \texttt{value}). K~tomu má k~dispozici 3~metody:
        
        \begin{enumerate}
            \item \textbf{Kliknutí pomocí JS.} V~prohližeči crawler zkouší
                kliknout na~dané tlačítko. Zda-li byla prodejna vybrána,
                verifikuje crawler kontrolou viditelnosti nadpisu detailu.
                V~případě, že~nadpis detailu je~viditelný, pokus byl~úspěšný.
                V~opačném případě crawler zkouší další metodu.
            \item \textbf{Kliknutí tzv. „na~sílu“.} Crawler zkouší na~položku
                kliknout mimo konzoli prohlížeče, opět probíhá verifikace
                prostřednictvím kontroly viditelnosti nadpisu detailu.
                Neúspěch znamená přesun na~poslední metodu.
            \item \textbf{Vyhledání pomocí textu.} Jako poslední možnost
                využije crawler vyhledání textu. Prohledá všechny položky
                a~snaží se~najít odpovídající text z~popisku položky.
                V~případě, že~lokaci najde, zkouší na~ní kliknout, úspěch
                verifikuje kontrolou viditelnosti nadpisu detailu. Pokud
                je~nadpis viditelný, pokus byl~úspěšný a~vyhledávání končí.
                Pokud nebyl úspěšný, crawler nechá vyhledávání doběhnout
                a~posléze, pokud do~té doby nebyl úspěšný, končí pokus
                o~otevření lokace a její zpracování se~přeskakuje.
        \end{enumerate}
    \item Nyní crawler zkouší načíst stránku s~detaily (popsána CSS
        selektorem \texttt{\#teleport-target div.flex.items-center.gap-x-4}).
        Z~této stránky si~vytáhne název stanice (\texttt{\#teleport-target
        span.text-sm.lg\textbackslash:text-base}) a~následně si~z~posledního
        sloupce z~první sekce z~každého řádku tabulky
        (\texttt{\#teleport-target
        div.lg\textbackslash:pl-6 section:nth-child(1)
        > table.w-full > tbody > tr}) vytáhne údaje o~palivu
        (\texttt{th.text-left}) a~jeho ceně (\texttt{td.text-right}). Cenu
        naformátuje do~číselného datového typu (odstraní identifikátor měny
        a~nahradí čárku tečkou, posléze údaj předá číselnému parseru)
        a~údaje o~palivu naformátuje do~objektu s~těmito atributy:

        \begin{itemize}
            \item \texttt{name}: Název paliva nebo doplňkového zboží z~tankovacího
                stojanu. Doplňkové zboží může být kapalina do~ostřikovačů nebo AdBlue.
            \item \texttt{price}: Cena paliva v~Kč.
        \end{itemize}
        
        Pokud nebyly nalezeny žádné informace o~palivech, prodejna
        je~přeskočena. Objekty s~údaji o~palivech crawler vkládá do~pole,
        které je~vkládáno do~objektu s~údaji o~prodejně (název a~lokalita).
        Tento objekt je~vkládán do~pole obsahující všechny prodejny. Pokud
        při~zpracování lokality dojde k~chybě, crawler pořídí ladící screenshot
        a~lokalitu přeskakuje.
        \medskip    \hrule  \medskip
        Zde je~konec opakované procedury. Nasbíraná data si~crawler uloží
        do~svého datasetu a~pokračuje dále.
        
        Následující sekvence popisuje nový cyklus, ohraničený novými
        horizontálními čarami.
        \medskip    \hrule
    \item Pro každou lokaci s~nasbíranými daty následně crawler vyhledá GPS
        souřadnice ČS u~dané prodejny. K~tomu jsem si~vybral API \emph{OSM
        Nominatim}, které~je zdarma a~open-source. API má~rate limit, tudíž
        každé volání musí být s~prodlevou alespoň 1~sekundy. URL má~tento
        formát:
        
        \begin{tt}
            \centering
            https://nominatim.openstreetmap.org/search?q=Globus+\$\{searchTerm\}\&format=json
        \end{tt}
        
        Výraz \texttt{\$\{searchTerm\}} označuje název lokace, který
        se~pro~vyhledávání použije. Pokud má~lokace upřesnění místa
        (za~pomlčkou, jedná~se zejména o~pražské pobočky a~o~tu~plzeňskou),
        použije~se pouze část za~pomlčkou. Je~to z~toho důvodu, že~během
        testování jsem zjistil, že~OSM nebere název hlavního města společně
        s~upřesněním čtvrti, ale~pouze název městské čtvrti, v~souvislosti
        s~Globusem. Odpověď, která je~ve~formátu JSON, crawler interpretuje
        jako pole objektů. Z~těchto dat nás zajímají jen GPS souřadnice
        odpovídající položky. Pokud OSM server nevrátí data ve~formátu pole,
        nastavujeme nulové souřadnice a~pokračujeme na~zápis dat do~databáze.
        V~opačném případě postupujeme dalším krokem.
    \item GPS souřadnice jsou uvedené v~textovém datovém typu, nikoliv
        v~číselném. Z~toho důvodu je~třeba souřadnice na~číselný datový typ
        převést. Hledáme souřadnice pro~položku typu ČS, která je definována
        atributem \texttt{type} s~hodnotou \texttt{fuel}. Pokud nelze čísla
        nalézt nebo jejich formát není strojově čitelný, použijí se
        souřadnice prodejny (\texttt{type: "supermarket"}). Pokud ani tehdy
        nelze souřadnice najít, použijou se~nulové. Z~těchto dat se~následně
        vytvoří objekt lokace, který obsahuje název lokace a~její souřadnice.
    \item V tomto bodě probíhá iterace jednotlivých paliv dané prodejny.
        V~každé iteraci jsou údaje o~názvu a~lokace ČS společně
        s~jednotlivými palivy zapisovány do~databáze. K~tomuto účelu jsem~si
        vytvořil vlastní knihovnu, která data nejen zapisuje, ale~také
        kontroluje jejich přítomnost a~zda nedošlo k~jejich aktualizaci
        od~posledního zápisu. Tato knihovna používá speciálně navržený formát
        dat, který je~popsán dále.
        \medskip    \hrule
    \item Zde je~konec procedury. Posledním krokem je~ukončení procesu.
\end{enumerate}

Jedna z~věcí, kterou musím zmínit, je~fakt, že Globus, soudě dle názvů CSS
tříd, používá na~své webové stránce Tailwind CSS. Tailwind je~CSS knihovna
vyvinutá pro~React a~podobné knihovny a~frameworky. Z~jednoho HTML elementu
vyplývá, že~by~mohlo jít o~Nuxt.js, což~je~framework postavený na~*Vue.js.

\subsection{Procházení ČS ONO}
\label{sec:crawling-ono}

Síť ČS~ONO používá oproti Globusu jednodušší a~starší formát webové aplikace.
Většina moderních webů dnes využívá React. ONO zůstává u~osvědčeného modelu
PHP aplikace, který se~stále používá i~dnes (např. WordPress nebo knihovna
Laravel běží na~PHP backendu). Výsledkem je~více člověkem čitelnější HTML
kód, který je~rovněž jednodušší ke~strojovému čtení a~analýze pro~automatický
crawler.

\begin{enumerate}
    \item Po~načtení hlavní webové stránky crawler hledá mapu odkazů
        na~dílčí stránky jednotlivých ČS, která se~nachází na~obrázku
        obsahující mapu ČR (znázorněna na~obrázku \ref{fig:tank-ono-mapa}
        na~straně \pageref{fig:tank-ono-mapa}). Jediný důvod, proč
        je~používaná mapa odkazů na~obrázku, je~především skutečnost,
        že~jsou odkazy v~kódu seřazené tak, jak jdou číselně za~sebou
        dle tabulky níže. Tabulka sice udává jejich pořadí semanticky,
        nicméně odkazy jsou seřazené do~sloupců, kód však postupuje
        v~posloupnosti podle řádků. Reálně se~jedná spíše o~mou preferenci,
        crawler by~stejně tak mohl použít i~odkazy v~tabulce. Jednotlivé
        dílčí odkazy jsou popsány CSS selektorem
        \texttt{map\#mapname > area}.
    \item Z~každého odkazu jsou brané atributy \texttt{href} a~\texttt{alt}.
        Atribut \texttt{href} označuje URL komponentu, která vede na~stránku
        dané ČS, zatímco atribut \texttt{alt} obsahuje lokalitu dané ČS.
        Tyto údaje jsou formátovány do~objektu, který se~stává položkou
        v~připraveném seznamu odkazů a~jejich popisků. Tento bod se~opakuje
        dokud nejsou vyčerpány všechny odkazy.

        Následující sekvence (ohraničená horizontálními čarami) obsahuje
        proceduru extrakce dat z~dílčích stránek jednotlivých ČS. Tato
        sekvence se~opakuje do~vyčerpání všech lokalit.
        \medskip    \hrule
    \item Pro~každou stránku jednotlivých ČS vytváříme nový prohlížeč.
        Pokud z~jakéhokoliv důvodu vytvoření prohlížeče selže, lokalita
        je~přeskočena a~do~konzole je~vypsána chyba.
    \item Prohlížeč načítá stránku detailu ČS z~vytaženého odkazu.
        Po~načtení extrahuje obsah nadpisu (\texttt{div\#nadpis}), který
        představuje plný název lokality. Vytažený nadpis je~posléze upraven
        odstraněním přebytečných mezer. Pokud tento či~jakýkoliv následující
        bod v~této sekvenci selže, crawler pořídí ladící screenshot
        a~lokalitu přeskakuje.
    \item Uložený název lokality z~mapy na~hlavní stránce crawler
        nyní upravuje pro~pozdější vyhledávací dotaz pro~OSM Nominatim.
        Během vývoje jsem zjistil, že~některé ČS ONO nelze vyhledat jednoduše
        podle vytaženého názvu lokality, ale~obměnou nebo vypuštěním některých
        částí názvu lokality. Název je~upraven dle následujících pravidel:

        \begin{itemize}
            \item Obsahuje-li název lokality slovo \emph{exit}
                (implikující umístění v~nespecifikované blízkosti sjezdu
                z~dálnice mimo její trasu), specifikace tohoto dálničního
                sjezdu je~z~názvu vypuštěna. V~současnosti se to týká jen ČS
                v~obci Studénka. Regulární výraz této specifikace je~popsán
                následovně:
                
                \begin{tt}
                    \centering
                    / *[,-] +D[0-9]+ +exit +[0-9]+/g
                \end{tt}
            \item Končí-li název lokality na~sekvenci znaků \texttt{ONO I}
                nebo \texttt{ONO II}, specifikace řetězce je~z~názvu
                vypuštěna a~římská číslice je~nahrazena jejím arabským
                protějškem.
            \item Během testování vyhledávacího dotazu v~API OSM Nominatim
                jsem zjistil konkrétně u~lokality ČS~Břest u~Kroměříže,
                že~je~třeba zadat pouze název \emph{Břest}, jelikož
                vyhledávací dotaz obsahující plný název lokality nefunguje.
                Přebytečnou specifikaci tak crawler v~tomto bodě vypouští.
        \end{itemize}
    \item Crawler nyní vytahuje jednotlivá data o~palivech (jejich název
        a~cenu). Síť ČS~ONO řeší tuto tabulku poněkud netradičně. Názvy
        paliv jsou ve~formě obrázku s~textem, zatímco ceny jsou poskládané
        z~jednotlivých číslic, která připomínají sedmisegmentový displej.
        Patrně se~tak snaží imitovat fyzické stojany s~ceníkem paliv
        a~doplňkových produktů na~daných lokalitách. Jelikož tabulka je~celá
        tvořena obrázky, je~potřeba identifikovat pouze ty~buňky, které
        obsahují údaje, které nás zajímají. To částečně řeší následující
        CSS selektor:

        \begin{tt}
            \centering
            div\#stojan > table tr:has(> td:nth-child(1) > img):has(> td:nth-child(2) > img)
        \end{tt}

        Částečně, protože tento selektor vybere i~ty~řádky, které obsahují
        pouze piktogramy informující o~nabízených službách (například
        automyčka). Ty však nejsou pro~účely tohoto projektu relevantní,
        navíc neobsahují žádné informace o~ceně. Tabulka má~také třetí
        sloupec. Ten však obsahuje ceník v~eurech, který pro~nás není
        relevantní. Spokojíme~se tedy jen s~prvním a~druhým sloupcem.

        Pro~každý řádek, který odpovídá našemu zadání, vybereme obrázek
        v~buňce obsahující název produktu (\texttt{td:nth-child(1) > img})
        a~buňku obsahující cenu produktu v~Kč (\texttt{td:nth-child(2)}).
        Tyto řádky strukturujeme do~objektu, kde každý objekt představuje
        položku v~poli.

        Tyto položky poté filtrujeme podle platných názvů palivových
        a~doplňkových produktů z~tankovacích stojanů ve~formátu URL jejich
        obrázků, a~podle platných číslic (prázdné ceny jsou přeskočeny).
    \item Vyfiltrované položky znovu projdeme a~provedeme konverzi
        z~URL obrázků na~platné hodnoty jmen produktů a~číslic. Algoritmus
        kontroluje, kde~se~nachází první desetinná číslice, a~v~tomto bodě
        před číslicí interpretuje desetinnou tečku. Prázdnou cifru
        si~lze~představit jako pomyslnou úvodní nulu před desetinnou čárkou
        a~pomyslnou závěrečnou nulu za~desetinnou čárkou. Tyto číslice
        konverzní program extrahuje ze~sekvence spojených URL údajů
        do~textového formátu, který následně převádí na~číselný datový typ.
        Zkonvertované datové údaje tak lze vložit do~objektu uvádějící název
        a~cenu daného paliva, a~tyto objekty nyní tvoří položky v~seznamu
        paliv. Pokud žádná paliva nebyla nalezena, lokalita je~přeskočena.
        To se~může stát v~případě, kdy je~ČS uzavřena. Zpravidla to~však
        značí chybový stav.
    \item Jako poslední bod této sekvence je poskládání informací o~palivech
        společně s~údaji o~lokalitě do~společného objektu, který se~následně
        vkládá do~pole obsahující všechny ČS této sítě.
        \medskip    \hrule
    \item Toto je~konec předchozí cyklické sekvence. Následně data uložíme
        do~datasetu crawleru a~pokračujeme dále na~zápis do~databáze.
        Postup zápisu do~databáze je~fundamentálně shodný s~postupem
        v~případě Globusu, liší se~jen způsob tvorby vyhledávacího dotazu
        pro~OSM Nominatim. Jelikož v~názvu se~nyní vyskytují mezery,
        ty je~třeba nahradit znakem \texttt{+}, který v~URL query sekvenci
        znamená právě mezeru. Teoreticky by~se~měla také provést kodifikace
        Unicode znaků (znaky mimo běžnou ASCII hodnotu), ale~\emph{Node.js
        Fetch} tuto konverzi provede sám.
\end{enumerate}

\subsection{Zápis do~databáze}
\label{sec:writing-to-db}

Jelikož databáze je~strukturovaná pomocí tabulky, údaje o~ČS se~opakují
pro~každé nabízené palivo a~dodatkové produkty z~tankovacích stojanů.
JSON formát těchto dat, jak je~přijímá logovací knihovna pro~databázi,
je~k~dispozici v~listingu \ref{lst:dbdata-json}. Formát je~zkrácen na~příklad
a~komentáře, plný popis tohoto formátu je~ve~zdrojovém kódu v~datové příloze
a~v~GitHub repozitáři. Datová struktura MySQL tabulky je vidět
v~tabulce~\ref{tab:petrolscan_data-structure}
na~straně~\pageref{tab:petrolscan_data-structure}.
\bigskip

\begin{lstlisting}[caption={JSON formát dat pro logovací knihovnu},label=lst:dbdata-json]
{
    "StationName": "Globus",        // Název řetězce
    "StationLocation": {
        "name": "Ostrava",          // Název lokace
        "lat": 0,                   // Zeměpisná šířka (latitude)
        "lon": 0                    // Zeměpisná délka (longitude)
    },
    "FuelName": "Natural 95",       // Název paliva
    "FuelType": "PETROL",           // Typ paliva
    "FuelQuality": "STANDARD",      // Kvalita paliva
    "FuelPrice": 21.6               // Cena paliva v Kč
}
\end{lstlisting}

\section{Frontend}
\label{sec:development-frontend}

Implementace frontendu srovnávací aplikace je~řešena pomocí \emph{Node.js}
s~využitím frameworku \emph{Next.js}, který je~založen na~Reactu. Next.js
poskytuje vývojářské nástroje pro~rychlou tvorbu webového designu, a~poskytuje
nástroje, které umožňují kombinovat klientské prvky s~těmi serverovými,
kde~část stránky zpracovává data na~pozadí na~straně serveru, a~ta~poté již
zpracovaná pošle uživateli. Výhodou této formy rozdělení úkonů aplikace je,
že~není potřeba zvlášť vytvářet API pro~komunikaci s~databází, část aplikace
prostě kontaktuje databázi přímo, aniž by~došlo k~úniku autentizačních
údajů na~stranu uživatelskou, a~uživatel i~přesto dostane data zpracovaná
dle~jeho požadavků.

Zdrojový kód této části je~v~datové příloze a~aktuální verze kódu
je~k~dispozici v~GitHub repozitáři
\texttt{http://github.com/POL0423/petrolscan-web-app}.

\subsection{Vytvoření projektu}

Prvním krokem je~samozřejmě vytvoření projektu. Předpokladem je~samozřejmě
nainstalovaný Node.js na~vývojovém systému. Spustíme příkaz
\texttt{\$ npx create-next-app@latest} a~nastavíme základní informace
o~projektu (viz~listing~\ref{lst:npx-create-next}
na~straně~\pageref{lst:npx-create-next}).

\subsection{Tvorba hlavní stránky}

Hlavní stránka je~vstupní branou do~aplikace. Ukáže~se, když zadáme
do~prohlížeče doménu aplikace. Má~aplikace je~celkem jednoduchá. Cílem
je~poskytnout jednoduché rozhraní, ze~kterého může uživatel jednoduše
zadat lokaci a~radius vyhledávání, a~výsledky vyhledávání řadit dle~ceny
vzestupně či~sestupně (od~nejlevnějšího či~nejdražšího) a~také filtrovat
dle~typu či~kvality produktu z~tankovacího stojanu. Na~některých ČS totiž
nejsou v~tankovacích stojanech k~dispozici jen~paliva, ale~také produkty,
které neslouží jako palivo, ale~slouží jako vedlejší spotřební zboží nutné
k~bezpečnému a~ekologickému provozu automobilu (např. AdBlue nebo kapalina
do~ostřikovačů).

V~datové příloze je k~dispozici zdrojový kód této aplikace. Soubor
\texttt{src/app/page.tsx} obsahuje TS kód definující rozvržení a~obsah
hlavní stránky. Design byl~vypůjčen z~ukázkového projektu z~kurzu Next.js
vyvinutého společností Vercel, který jsem absolvoval před zahájením práce
na~frontendu mé~aplikace. Stránka se~skládá zejména z~hlavičky,
postranního panelu s~vyhledávacím formulářem a~ilustračního obrázku vpravo,
který ukazuje příklad s~výsledky vyhledávání ve~formě prohlížeče v~PC
i~v~mobilním zobrazení (viz~obrázek~\ref{fig:web-home-desktop}
na~straně~\pageref{fig:web-home-desktop}). Mobilní rozvržení této stránky
ukazuje alternativní obrázek jen s~mobilním zobrazením výsledků vyhledávání
a~postranní panel se~posouvá nad~ilustrační obrázek
(viz~obrázek~\ref{fig:web-home-mobile}
na~straně~\pageref{fig:web-home-mobile}).

V~postranním panelu se~nachází vyhledávací formulář. Jeho součástí je~pole
pro~zadání vztažné lokality, a~zároveň je~zobrazeno pole pro~zadání okruhu
vyhledávání. Pole pro~zadání vztažné lokality je~interaktivní, uživatel
při~zadávání dostává našeptávač s~upřesněním lokality. Aplikace pro~tento účel
využívá API OSM Nominatim, které poskytuje geografické vyhledávací rozhraní.
Při~psaní se~v~prohlížeči uživatele spouští kód, který kontaktuje API OSM
s~vyhledávacím dotazem vepsaným do~vyhledávacího pole, a~pokud jsou vráceny
výsledky vyhledávání, zobrazí~se našeptávač s~jednotlivými nalezenými
položkami, jejichž počet je~omezen na~10. Uživatel může vybrat jednu
z~položek, čímž našeptávač zavře a~svůj vyhledávací dotaz nahradí vybranou
položkou z~našeptávače. Odesláním formuláře se~dostane uživatel na~stránku
s~výsledky vyhledávání.

\subsection{Výsledky vyhledávání}
\label{sec:search-results}

Tato stránka má~design shodný s~hlavní stránkou, ovšem místo ilustračního
obrázku se~na~stránce nachází výsledky vyhledávání seřazené ve~výchozím
nastavení od~nejlevnějších po~nejdražší
(viz obrázky~\ref{fig:web-search-desktop}
na~straně~\pageref{fig:web-search-desktop}
a~\ref{fig:web-search-mobile} na~straně~\pageref{fig:web-search-mobile}).
Pokud nebyly nalezeny žádné výsledky, stránka skutečnost uživateli
oznámí. Ve~zdrojovém kódu v~datové příloze je~TS kód této stránky v~souboru
\texttt{src/app/search/page.tsx}.

Výsledky vyhledávání jsou doplněny o~možnosti řazení položek dle~ceny produktů
vzestupně (od~nejlevnějších) nebo sestupně (od~nejdražších) a~filtrování
položek podle typu produktu či~jeho kvality. Pro~jednoduchost lze~filtrovat
pouze dle~jednoho typu a~dle~jedné kvality. Tyto možnosti řazení a~filtrování
jsou pomocí skrytých políček ve~vyhledávacím poli zachovány při~úpravě
vyhledávacích parametrů.

Technickým řešením pro~zpracování vyhledávacího dotazu je~ještě jedno volání
API OSM Nominatim, tentokrát však výsledky vyhledávacího dotazu limituje
na~jediný. Z~vrácených dat si~vytahuje GPS souřadnice nalezeného místa
(zeměpisnou šířku a~délku). Tyto souřadnice následně využívá k~filtrování
dat v~databázi dle~vzdálenosti od~souřadnic dané ČS. K~tomu využívá speciální
SQL funkci \texttt{ST\_Distance\_Sphere()}, která počítá vzdálenost dvou bodů
na~kulové ploše. Poloměrem koule je~poloměr Země, který činí 6371 km.
Body jsou určeny pomocí GPS souřadnic vztažného místa a~umístění ČS. Jelikož
data jsou strukturovaná do~tabulky, je~potřeba sdružit data paliv a~doplňkových
produktů z~tankovacích stojanů do~společného datového objektu pro~každou ČS.
MySQL od~verze 5.7.22 nabízí JSON funkce. Tato data jsou následně seřazena
dle~ceny produktů vzestupně či~sestupně. Výchozí SQL dotaz vypadá následovně:
\bigskip

\begin{lstlisting}[label=lst:default-sql-query,caption={SQL dotaz srovnávače cen},language=sql]
SELECT
    station_name,
    station_loc_name,
    station_loc_lat,
    station_loc_lon,
    MAX(timestamp) AS last_update,
    JSON_ARAYAGG(
        JSON_OBJECT(
            'id', id,
            'timestamp', timestamp,
            'fuel_type', fuel_type,
            'fuel_quality', fuel_quality,
            'fuel_name', fuel_name,
            'fuel_price', fuel_price
        )
    ) AS fuels
FROM petrolscan_data
WHERE (
    ST_Distance_Sphere(
        POINT(station_loc_lon, station_loc_lat),
        POINT(?, ?),
        6371
    ) <= ?
)
GROUP BY
    station_name,
    station_loc_name,
    station_loc_lat,
    station_loc_lon
ORDER BY fuel_price ASC;
\end{lstlisting}

Prvním a~druhým argumentem jsou zde~GPS souřadnice vztažného místa v~pořadí
zeměpisné délky a~šířky (tedy obrácené pořadí oproti běžnému zápisu šířky
a~délky), třetím argumentem je~maximální vzdálenost v~kilometrech.
Číslo~\texttt{6371} představuje poloměr Země v~kilometrech.

Pokud chceme řazení dle ceny sestupně, místo \texttt{ORDER BY fuel\_price ASC}
se~použije \texttt{ORDER BY fuel\_price DESC}.

Pokud navíc chceme filtrovat jen určitý typ paliva či~doplňkového zboží nebo
jeho kvalitu, tak za~porovnání výsledku funkce \texttt{ST\_Distance\_Sphere()}
s~maximální vzdáleností přidáme ještě porovnání se~zadaným typem, případně
kvalitou, pomocí booleanovského operátoru \texttt{AND}.

Takto získaná data následně zpracujeme do~přehledného seznamu, který obsahuje:

\begin{itemize}
    \item název stanice včetně jejího umístění a~GPS souřadnic,
    \item seznam produktů s~jejich názvem, cenou, typem a~kvalitou,
    \item vzdálenost ČS od~vztažného místa v~kilometrech,
    \item celkový počtet produktů
    \item a~poslední aktualizaci dané lokality.
\end{itemize}

Vzdálenost ČS od~vztažného místa program počítá pomocí haversinového vzorce
\cite{Gade2010}. %Ref: Gade-2010: Haversine formula
Postup výpočtu je~následující:

\begin{equation}
    \label{eq:haversine-formula}
        d   =   2 \cdot R \cdot \arcsin \sqrt{
                    \sin^2 \frac{\Delta\varphi}{2} +
                    \cos \varphi_1 \cdot \cos \varphi_2 \cdot
                    \sin^2 \frac{\Delta\lambda}{2}
                }~[\text{km}]
\end{equation}

kde \(R = 6371~\text{km}\), \(\varphi_1\) a~\(\lambda_1\) jsou GPS souřadnice
vztažného místa (zeměpisná šířka a~délka), \(\varphi_2\) a~\(\lambda_2\)
jsou GPS souřadnice ČS, \(\Delta\varphi = \varphi_2 - \varphi_1\)
a~\(\Delta\lambda = \lambda_2 - \lambda_1\).

Zeměpisné šířky a~délky je~třeba před výpočtem převést na radiány, k~čemuž
by~měl~sloužit tento vzorec:

\begin{equation}
    \label{eq:deg-to-rad}
    \varphi_{\text{rad}} = \pi \cdot \frac{\varphi_{\text{st}}}{180}
\end{equation}

Výsledkem těchto výpočtů je~vzdálenost mezi dvěma body na~Zemi v~kilometrech.

\section{Ostatní součásti a jejich propojení}
\label{sec:other-parts-and-stitching-services-together}

Posledním krokem je~kontejnerizace těchto dílčích služeb. V~listingu
\ref{src:docker-compose.yml} na~stránce \pageref{src:docker-compose.yml}
lze vidět soubor \texttt{docker\_compose.yml}, který je~jinak dostupný
také v~datové příloze včetně dodatkových souborů a~aktuální verze
je~v~repozitáři
\texttt{https://github.com/POL0423/petrolscan-docker-compose}.

Během testování v~Docker kontejnerech jsem musel udělat některé malé
změny, neboť došlo ke~změně funkčnosti jednotlivých dílů, některé součásti
bez~úprav prostředí a~programu nefungovaly správně. Chyby se~mi~podařilo
vyřešit, a~tak zbývala poslední věc: nahrát~to vše na~server a~nastavit
reverse proxy. Zjistil jsem, že~na~serveru chyběl Docker compose,
ten tedy bylo potřeba doinstalovat, a~zřejmě nebyl správně nainstalovaný
Docker, tudíž jsem musel Docker přeinstalovat.

\begin{verbatim}
$ sudo dnf install docker
\end{verbatim}

Vyskytly se potíže s oprávněními, které se mi vyřešit nepodařilo,
takže Docker je~třeba ovládat přes~\texttt{sudo}. To~není nikterak
limitující, jen~je~třeba čas od~času zadat heslo. Větší problém byla
konfigurace sítě. Změny totiž nebyly zapsány trvale a~došlo k~odejmutí
nastavení IPv6 adresy. Řešením bylo IP adresu nastavit znovu a~změnit
konfiguraci IPv6 adresování na~ruční. Tím bylo vypnuto dotazování DHCP
serveru pro~IPv6 a~serveru tím pádem zůstává ručně přidělená IPv6.
DHCP server totiž IPv6 adresy nepřiděluje, ale~pouze IPv4.

Poté, co~jsem vše~nastavil, jsem také nainstaloval Nginx následujícím
příkazem:

\begin{verbatim}
$ sudo dnf install nginx
\end{verbatim}

Nyní zbývá nastavit SSL. V~souboru \texttt{/etc/nginx/nginx.conf}
lze~vidět v~kontextu \texttt{http} následující příkaz:

\begin{verbatim}
include /etc/nginx/conf.d/*.conf;
\end{verbatim}

To~znamená, že~konfigurace použije všechny platné moduly, které se~nachází
v~adresáři \texttt{/etc/nginx/conf.d}. Proto jsem se~rozhodl vytvořit
v~tomto adresáři soubor \texttt{pol0423-stu.vsb.cz.conf}, který obsahuje
konfiguraci pro~reverse proxy s~přesměrováním na~HTTPS
a~v~adresáři \texttt{/etc/nginx/snippets} jsem vytvořil další
soubory s~nastavením SSL. (viz~listingy~\ref{src:pol0423-stu.vsb.cz.conf}
na~straně~\pageref{src:pol0423-stu.vsb.cz.conf},
\ref{src:snippets/ssl-params.conf} na~straně
\pageref{src:snippets/ssl-params.conf} a~\ref{src:snippets/self-signed.conf}
na~straně~\pageref{src:snippets/self-signed.conf}).

K~tomu je~též~zapotřebí vytvořit SSL certifikát. Nejprve je~potřeba vytvořit
odpovídající adresáře. Adresář \texttt{/etc/ssl/certs} již~existuje, jedná~se
o~symbolický odkaz na~adresář \texttt{/etc/pki/tls/certs}.
Vytvoříme tak adresář \texttt{/etc/ssl/private} a~vygenerujeme nový
self-signed SSL certifikát:

\begin{verbatim}
$ sudo mkdir /etc/ssl/private
$ sudo openssl req -x509 -nodes -days 365 -newkey rsa:2048 -keyout \
    /etc/ssl/private/nginx-selfsigned.key -out \
    /etc/ssl/certs/nginx-selfsigned.crt
\end{verbatim}

Vyplníme údaje, necháme vygenerovat soukromý klíč a~certifikát a~pokračujeme
dále vytvořením \texttt{dhparams} souboru.

\lstinputlisting[label=lst:ssl-gen,caption={Údaje self-signed certifikátu}]{SourceCodes/ssl-gen.log}

\begin{verbatim}
$ sudo openssl dhparam -out /etc/nginx/dhparam.pem 4096
\end{verbatim}

Generování \texttt{dhparam} souboru trvá několik desítek minut, nicméně
poté je~již~konfigurace Nginx připravena. Zkontrolujeme integritu
konfigurace, a~pokud je~konfigurace v~pořádku, restartujeme Nginx.

\begin{verbatim}
$ sudo nginx -t
$ sudo systemctl restart nginx
\end{verbatim}

Nyní je~potřeba aplikaci dostat na~server. Fakt, že~je~projekt na~GitHubu,
pomohl v~tom smyslu, že~jsem mohl jednoduše stáhnout repozitáře.

\begin{verbatim}
$ git clone https://github.com/POL0423/petrolscan-crawler
$ git clone https://github.com/POL0423/petrolscan-web-app
$ git clone https://github.com/POL0423/petrolscan-docker-compose
\end{verbatim}

Přesuneme~se do~adresáře \texttt{petrolscan-docker-compose} a~spustíme
příkaz Docker compose:

\begin{verbatim}
$ cd petrolscan-docker-compose
$ sudo docker compose up -d
\end{verbatim}

Tento příkaz sestaví Docker obrázky z~dostupných repozitářů a~nastaví
jim síť, porty a~perzistentní úložiště na~základě souboru
\texttt{docker-compose.yml}.

Jakmile jsou Docker obrázky sestaveny, dojde k~vytvoření a~spuštění
odpovídajících Docker kontejnerů. Necháme proběhnout skript k~vyplnění
databáze a~aplikaci lze~nyní vyzkoušet na~adrese
\texttt{https://pol0423-stu.vsb.cz/}. Screenshot aplikace na~doméně
serveru si~lze~prohlédnout na~obrázku~\ref{fig:web-server}
na~straně~\pageref{fig:web-server}. Všimněte~si, že~web běží na~HTTPS.
Pokud se~pokusíme zadat \texttt{http://pol0423-stu.vsb.cz/}, dojde
k~přesměrování na~HTTPS. Server je~tedy připraven a~spuštěn.

Prohlížeč samozřejmě při~první návštěvě zobrazí upozornění, že~certifikát
není důvěryhodný. Je~to z~toho důvodu, že~nemá důvěryhodnou certifikační
autoritu, je~self-signed. Tím se~stává kořenovým certifikátem, který OS
samozřejmě nemá nainstalovaný v~seznamu důvěryhodných certifikátů.
Navíc je~protokol \texttt{https} zobrazen v~červené barvě a~přeškrtnutý,
a~vedle se~zobrazuje přeškrtnutá červená ikonka zámečku nebo nápis
\emph{Nezabezpečeno} (v~závislosti na~použitém prohlížeči). Ve~skutečnosti
však šifrování funguje, prohlížeči se v~tomto případě jen nelíbí skutečnost,
že~certifikát není podepsaný důvěryhodnou certifikační autoritou.
V~reálném použití se~samozřejmě self-signed certifikát nedoporučuje používat.
Jeho použití však nikterak nebrání používání aplikace, a~funkční aplikace
je~dostačující k~demonstraci. Jelikož webová stránka není dostupná mimo
školní síť, použití self-signed certifikátu v~takovém případě ničemu nevadí.

\endinput
