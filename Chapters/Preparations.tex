\chapter{Příprava}
Pro provoz webové aplikace je zapotřebí backend server.
Pro tento účel jsem zvolil ICT centrum VŠB-TUO, kde jsem si nechal
vytvořit virtuální server.

\section{Operační systém a konfigurace DNS}
Jako operační systém serveru jsem zvolil Rocky Linux,
který mi byl doporučen administrátory školní sítě. Tato linuxová
distribuce je založena na RHEL. Součástí konfigurace VM je také
nastavení sítě. Neměnná IP adresa je získána z DHCP serveru,
avšak se jedná pouze o IPv4. IPv6 je potřeba nastavit ručně.
Pro server je předkonfigurovaný také DNS záznam pro IPv4 a IPv6,
který vypadá následovně:

\begin{verbatim}
Name                       Type   TTL   Section    IPAddress
----                       ----   ---   -------    ---------
pol0423-stu.vsb.cz         AAAA   7200  Answer     2001:718:1001:207::64
pol0423-stu.vsb.cz         A      7200  Answer     158.196.109.64
\end{verbatim}

DNS záznam AAAA je vazba na IPv6 adresu, zatímco DNS záznam A
je vazba na IPv4.

\section{Instalace OS a počáteční konfigurace serveru}
Server se nachází na VMware vSphere serveru, dostupný
na adrese vcs.vsb.cz. Prvním krokem byla instalace OS Rocky Linux,
která proběhla přes virtuální konzoli serveru. Instalační program
byl ve formě GUI, přes které jsem nastavil uživatelský účet správce
\texttt{marpolda}, nastavil jsem mu heslo a zapnul mu práva správce.
Účet správce \texttt{root} je vypnutý, lze ho tedy použít jen pomocí
příkazu \texttt{sudo}. Po instalaci OS byla dalším krokem instalace
nástrojů virtuálního stroje. Po neúspěšném pokusu v instalaci balíčku
VMware Tools jsem se rozhodl využít místo toho otevřený balíček
\texttt{open-vm-tools}. Pro práci s textovými soubory jsem si
také nainstaloval balíček \texttt{nano}, který poskytuje jednoduchý
textový editor přímo v příkazovém řádku.

Dalším krokem je nastavení IPv6 sítě. Jelikož IPv6 adresa není
DHCP serverem přidělena, je potřeba adresu nastavit ručně. Pro to
využijeme výše zmíněný DNS záznam pro IPv6 vazbu. Tento proces
je také zapotřebí provést pomocí virtuální konzole (viz listing
\ref{lst:net-modify} na stránce \pageref{lst:net-modify}).

Otestujeme konektivitu IPv4 a IPv6:

\begin{verbatim}
PS C:\Users\marpo> ping -4 pol0423-stu.vsb.cz

Pinging pol0423-stu.vsb.cz [158.196.109.64] with 32 bytes of data:
Reply from 158.196.109.64: bytes=32 time<1ms TTL=61
Reply from 158.196.109.64: bytes=32 time<1ms TTL=61
Reply from 158.196.109.64: bytes=32 time<1ms TTL=61
Reply from 158.196.109.64: bytes=32 time<1ms TTL=61

Ping statistics for 158.196.109.64:
    Packets: Sent = 4, Received = 4, Lost = 0 (0% loss),
Approximate round trip times in milli-seconds:
    Minimum = 0ms, Maximum = 0ms, Average = 0ms
PS C:\Users\marpo> ping -6 pol0423-stu.vsb.cz

Pinging pol0423-stu.vsb.cz [2001:718:1001:207::64] with 32 bytes of data:
Reply from 2001:718:1001:207::64: time=2ms
Reply from 2001:718:1001:207::64: time=1ms
Reply from 2001:718:1001:207::64: time<1ms
Reply from 2001:718:1001:207::64: time=1ms

Ping statistics for 2001:718:1001:207::64:
    Packets: Sent = 4, Received = 4, Lost = 0 (0% loss),
Approximate round trip times in milli-seconds:
    Minimum = 0ms, Maximum = 2ms, Average = 1ms
\end{verbatim}

Pro přístup na server pomocí SSH jsem si také importoval ručně
přes konzoli SSH klíče obou mých počítačů, které využívají kvantově
rezistentní algoritmus \texttt{ed25519}. Z důvodu bezpečnosti jsem
také provedl vypnutí přihlašování pomocí uživatelského hesla.
Tento krok jsem provedl vytvořením nového souboru
\texttt{/etc/ssh/sshd\_config.d/01-nopasswordlogin.conf}
s následujícím obsahem:

\newpage

\begin{verbatim}
#################################################
# Disable password logins
#################################################

PasswordAuthentication no
\end{verbatim}

Soubory v adresáři \texttt{/etc/ssh/sshd\_config.d} jsou automaticky
importovány v souboru\\
\texttt{/etc/ssh/sshd\_config}, který obsahuje konfiguraci SSH Daemon
serveru.

Následně stačilo restartovat službu SSH Daemon:

\begin{verbatim}
# systemctl restart sshd.service
\end{verbatim}

Přístup na server pomocí SSH jsem následně otestoval:
\begin{verbatim}
PS C:\Users\marpo> ssh marpolda@pol0423-stu.vsb.cz
The authenticity of host 'pol0423-stu.vsb.cz (158.196.109.64)' can't be established.
ED25519 key fingerprint is SHA256:oHy0UKZisrWxLKQtp5Xpezo53FNXZudKJ6/WVHeScI4.
This host key is known by the following other names/addresses:
    ~/.ssh/known_hosts:18: 158.196.109.64
Are you sure you want to continue connecting (yes/no/[fingerprint])? yes
Warning: Permanently added 'pol0423-stu.vsb.cz' (ED25519) to the list of known hosts.
Enter passphrase for key 'C:\Users\marpo/.ssh/id_ed25519':
Last login: Sat Mar  1 14:01:55 2025 from 158.196.52.150
[marpolda@pol0423-stu ~]$
\end{verbatim}

Z mého druhého počítače jsem se také úspěšně přihlásil:
\begin{verbatim}
[marpolda@archlinuxx-laptop ~]$ ssh pol0423-stu.vsb.cz
Enter passphrase for key '/home/marpolda/.ssh/id_ed25519':
Last login: Sun Mar  2 19:21:04 2025 from 2001:718:1001:698:99e2:f1ff:4e67:7618
[marpolda@pol0423-stu ~]$
\end{verbatim}

\section{Kontejnerizace a instalace služeb}
Nyní, když máme nastavený základní přístup, můžeme přistoupit
k instalaci softwaru pro kontejnerizaci a samotných kontejnerů
potřebných služeb. Pro kontejnerizaci jsem si vybral Docker,
který jsem nainstaloval z příslušného balíčku následovně:

\begin{verbatim}
# dnf install docker
\end{verbatim}

Dalším krokem je příprava samotných součástí aplikace. K tomu
jsem si vytvořil vývojové prostředí Gitu, ve kterém bude
probíhat vývoj aplikace a příprava pro její nasazení na server.
Toto prostředí se také nachází na GitHubu, kde je vidět
aktuální podoba této aplikace. Aplikaci jsem se rozhodl pojmenovat
\emph{PetrolScan}. Všechny součásti aplikace obsahují soubor
\texttt{Dockerfile}, který vytváří Docker obrázek dané součásti.
Všechny tyto součásti jsou pospojovány souborem \texttt{docker-compose.yml},
který součásti propojuje a vytváří tak ucelenou službu.

Vyvíjené součásti aplikace jsou:

\begin{itemize}
    \item Webová aplikace:
        \texttt{\url{https://github.com/POL0423/petrolscan-web-app}}
    \item Crawler:
        \texttt{\url{https://github.com/POL0423/petrolscan-crawler}}
    \item Docker Compose:
        \texttt{\url{https://github.com/POL0423/petrolscan-docker-compose}},
        Listing \ref{src:docker-compose.yml} na stránce
        \pageref{src:docker-compose.yml}
\end{itemize}

\subsection{Výběr softwaru pro crawler}

Jako první jsem se rozhodl najít vhodné self-hosted webové crawlery
a scrapery. Na dotaz k vyhledávání mi ChatGPT našlo následující možnosti:

\begin{itemize}
    \item Crawlee
    \item Scrapy
    \item EasySpider
\end{itemize}

Rozhodl jsem se pro Crawlee.

\subsubsection{Zkouška Crawlee}

Tento crawler funguje v Node.js, a nabízí jednoduchou instalaci pomocí
jednoduchého příkazu. Crawler simuluje procházení webu člověkem pomocí
bezhlavičkových webových prohlížečů, které jsou schopné také interpretovat
Javascriptový kód.

Zkušební test ukázkového crawleru proběhl v pořádku.
První pokus o spuštění neuspěl, neboť chyběly Playwright bezhlavičkové
prohlížeče. Po instalaci těchto prohlížečů se již crawler spustil.

Velmi příjemným zjištěním bylo, že builder již automaticky vygeneruje
soubor \texttt{Dockerfile}, pro crawler tak lze rovnou vytvořit
Docker image, který se jen posléze nahraje na Docker Hub a lze s ním
dále pracovat.

\subsubsection{Tvorba crawleru}

Nejprve je nutné prozkoumat webové stránky různých čerpacích stanic.
Uvažujme tak tyto čerpací stanice:

\begin{itemize}
    \item \textbf{Globus}
    \item Orlen
    \item Shell
    \item EuroOil
    \item \textbf{ONO}
    \item MOL
    \item OMV
    \item Prim
    \item Makro
\end{itemize}

Pro všechny tyto ČS tak je nutné prozkoumat a analyzovat jejich webové
stránky, abychom mohli z daných webových stránek vytáhnout příslušná data.
Začíná mravenčí práce, a sice prozkoumávání a analýza skladby jednotlivých
webových stránek ČS, zápis struktury každého webu a sestavení jednotlivých
šablon pro náš crawler tak, aby byl crawler schopen stránky procházet
a získávat daný obsah. Crawler tato data z webových stránek čte a zaznamenává
do databáze strukturovaně podle typu jednotlivých PHM, jejich značce
a umístění ČS.

\subsubsection{Globus}

Tato ČS patří velkoobchodnímu řetězci (hypermarketu) stejného jména
a nachází se výhradně vždy v blízkosti prodejny. Globus lze v době
psaní této bakalářské práce najít jen v 16 různých městech a městských
částí v ČR:

\begin{itemize}
    \item Brno
    \item České Budějovice
    \item Chomutov
    \item Havířov
    \item Karlovy Vary ‒ Jenišov
    \item Liberec
    \item Olomouc
    \item Opava
    \item Ostrava
    \item Pardubice
    \item Plzeň ‒ Chotíkov
    \item Praha
    \begin{itemize}
        \item Čakovice
        \item Černý Most
        \item Štěrboholy
        \item Zličín
    \end{itemize}
    \item Trmice
\end{itemize}

Postup pro získání informací o cenách PHM z ČS je následující.

\begin{enumerate}
    \item Na webové stránce \url{www.globus.cz} klikneme na tlačítko s ikonou
        špendlíku. Pokud se nás web zeptá na výběr prodejny, zvolíme
        dle požadavků (například Ostravu). Pokud je zvolená prodejna
        nesprávná, lze výběr změnit kliknutím na tlačítko „Změnit“
        v prvním sloupci pop-up okénka.
    \item Požadované informace z této ČS jsou ve třetím sloupci (viz
        screenshot na obrázku \ref{fig:globus-cs} na stránce
        \pageref{fig:globus-cs}).
\end{enumerate}

Tento postup je potřeba automatizovat. K tomu nám právě poslouží již
zmíněný crawler. Každá ČS má jinou strukturu webu, a proto je potřeba
vytvořit pro každý web jeho vlastní crawler. K tomu lze využít threading
v Node.js.

\subsubsection{ONO}

ČS ONO jsou v samostatné síti ČS provozované společností Tank ONO s.r.o.
Web se nachází na adrese \url{www.tank-ono.cz}, který je velmi jednoduchý
a přehledný. Postup k získání informací z ČS je následující:

\begin{enumerate}
    \item Načteme hlavní webovou stránku \url{www.tank-ono.cz}. Zobrazí
        se nám mapa s jednotlivými pobočkami sítě ČS ONO. Na této webové
        stránce se nachází celkem 43 stanic. Jejich mapu lze vidět
        na obrázku \ref{fig:tank-ono-mapa} na stránce
        \pageref{fig:tank-ono-mapa}. V tabulce \ref{tab:tank-ono-pobocky}
        na stránce \pageref{tab:tank-ono-pobocky} lze vidět jejich výčet.
    \item Kliknutím na jednu z položek v seznamu nebo kliknutím na pozici
        na mapě si lze zobrazit detail konkrétní ČS. Na obrázku
        \ref{fig:tank-ono-stranka} na stránce \pageref{fig:tank-ono-stranka}
        lze vidět strukturu webové stránky detailu ČS. Vlevo nahoře (a) je název
        a umístění ČS, zatímco vpravo dole je vidět tabulka PHM (b) včetně jejich
        cen v Kč (c) a eurech. Toto jsou data, která nás zajímají.
\end{enumerate}

\newpage

\begin{table}[]
    \centering
    \caption{Seznam ČS ONO}
    \label{tab:tank-ono-pobocky}
    \begin{tabular}{r l|r l}
        1.    & ČS Plzeň, Domažlická        & 24.   & ČS Sukorady u Mladé Boleslavi\\
        2.    & ČS Nýřany                   & 25.   & ČS Podolí u Písku\\
        3.    & ČS Plzeň, Studentská        & 26.   & ČS Planá nad Lužnicí\\
        4.    & ČS Sokolov                  & 27.   & ČS Spytihněv\\
        5.    & ČS Teplice                  & 28.   & ČS Mošnov\\
        6.    & ČS Cheb                     & 29.   & ČS Kojetice-Tůmovka\\
        7.    & ČS Milovice u Hořic         & 30.   & ČS Cvikov\\
        8.    & ČS Chlumec                  & 31.   & ČS Chomutov - Přečaply\\
        9.    & ČS Jihlava                  & 32.   & ČS Zádveřice u Zlína\\
        10.   & ČS Trutnov                  & 33.   & ČS Brno-Popovice\\
        11.   & ČS Jindřichův Hradec        & 34.   & ČS Roudnice nad Labem\\
        12.   & ČS Vysoké Mýto              & 35.   & ČS Vysokov u Náchoda\\
        13.   & ČS Havraň                   & 36.   & ČS Praha - Štěrboholy\\
        14.   & ČS Kolaje u Poděbrad        & 37.   & ČS Řasnice-Strážný\\
        15.   & ČS Ústí n. L., Božtěšická   & 38.   & ČS Břest u Kroměříže\\
        16.   & ČS Praha, Dolní Měcholupy   & 39.   & ČS Vojtanov\\
        17.   & ČS Kbelnice                 & 40.   & ČS Dobkovice u Děčína\\
        18.   & ČS Havlovice u Domažlic     & 41.   & ČS Březno u Loun D7\\
        19.   & ČS Mělník                   & 42.   & ČS Brno-Hviezdoslavova\\
        20.   & ČS Krupá u Rakovníka        & 43.   & ČS Studénka - D1 exit 336\\
        21.   & ČS Dolní Dvořiště - ONO I   & 44.   & ČS Ostrov nad Ohří\\
        22.   & ČS Dolní Dvořiště - ONO II  & 45.   & ČS Chlumčany u Přeštic\\
        23.   & ČS Církvice u Kutné Hory
    \end{tabular}
\end{table}

\subsubsection{Vyřazené čerpací stanice}

Během analýzy zmíněných webů ČS jsem zjistil, že právě pouze Globus
a ONO poskytují ceny PHM. Následující weby řetězců ČS jsem musel
z projektu vyřadit, neboť nesplňují požadavky k dolování dat.

\begin{itemize}
    \item \textbf{OMV:} Tato síť ČS poskytuje ceny PHM ve formátu,
        který vyžaduje použití OCR čtečky, jejíž implementace
        by byla velmi náročná, a časově by nebylo reálné čtečku
        do projektu zakomponovat.
    \item \textbf{Orlen:} Webové stránky této ČS využívají REST API,
        což je vysoké plus. Ale to samo o sobě nestačí. Pokud ani
        v API, ani na webové stránce nejsou k dispozici ceny PHM,
        jen slevy na tankovací kartu či aplikaci této sítě, nelze
        tato data v porovnání cen PHM použít.
    \item \textbf{Shell, EuroOil, MOL a Prim:} Tyto ČS informace
        o cenách PHM na svých webových stránkách vůbec neposkytují,
        čímž nelze data z těchto webových stránek použít.
    \item \textbf{Makro:} Tato ČS sice ceny na webu uvádí, a to
        v přijatelném formátu, ale web implementuje pokročilé
        antiscraping mechanismy, které detekují a blokují crawlery
        podobné tomuto projektu. Z toho důvodu byl tento crawler
        během jeho vývoje vyřazen. Crawler zkrátka nemohl tyto
        mechanismy obejít a k implementaci protichůdných opatření
        již nezbýval dostatek času.
\end{itemize}

\subsubsection{Threading a strukturizace crawleru}

V datové příloze a také na webové stránce
\texttt{\url{github.com/POL0423/petrolscan-crawler}} lze vidět zdrojový
kód crawleru. Součástí tohoto zdrojového kódu je soubor \texttt{Dockerfile},
který definuje obrázek pro Docker kontejner. Lze si jej představit jako jakousi
předlohu pro souborový systém uvnitř kontejneru. Crawler je rozdělen na několik
částí, která každá plní svou specifickou úlohu:

\begin{itemize}
    \item \textbf{Hlavní část:} Tuto část představuje soubor \texttt{src/main.ts}.
        Jedná se o hlavní spouštěcí kód, který volá další moduly.
    \item \textbf{Dílčí crawlery:} Tyto části provádějí samotnou sklizeň. Jedná se
        o samotnou implementaci logiky dílčích crawlerů. To jsou dva moduly: Globus
        a ONO. Oba tyto moduly mají společný nadřazený abstraktní modul, který
        definuje společné vlastnosti a metody těchto modulů.
    \item \textbf{Databázová spojka:} Tato část propojuje crawlery s databází
        a umožňuje jim do databáze zapisovat nasbíraná data.
    \item \textbf{Definice datových typů:} V této části definuji některé datové
        typy, které dílčí crawlery používají. Datové typy jsou definovány kvůli
        snadnější identifikaci a implementaci práce s informacemi.
    \item \textbf{Crontab:} Hlavní spouštěcí kód je definován jako index
        v souboru \texttt{package.json}. V souboru \texttt{scripts/run\_cron.sh}
        je definován příkaz ke spuštění tohoto hlavního spouštěcího kódu. Tento
        soubor je poté odkazován souborem \texttt{scripts/jobs.crontab},
        který tento skript periodicky spouští každý den ve 3:00 a zajišťuje
        zapisování výstupu konzole do logovacího souboru.
\end{itemize}

\subsection{Databáze}

Pokud je crawler mozkem aplikace, který vyhledává data a ukládá je,
pak databáze je určitě srdcem celé aplikace, protože umožňuje její fungování.
Bez databáze by aplikace fungovat nemohla. K dispozici je několik různých
softwarů. Především bych vyjmenoval následující možnosti:

\begin{itemize}
    \item \textbf{MySQL:} Zřejmě nejpoužívanější databázový software na světě.
        Jeho předností je jednoduché použití, všestrannost, a široká
        kompatibilita s různými softwary pro backend i frontend. MySQL
        je k dispozici ve více variantách:
        \begin{enumerate}
            \item \textbf{Oracle:} Originální verze MySQL, která je stále
                aktivně vyvíjená. Tuto verzi lze najít na většině komerčních
                serverů.
            \item \textbf{MariaDB:} Jedná se o fork originální verze MySQL
                (fork znamená, že se jedná o verzi odvozenou z původní).
                V dnešní době pro většinu menších projektů příjemná
                alternativa, která je zdarma a open-source.
        \end{enumerate}
    \item \textbf{PostgreSQL:} Rychlý a moderní nástupce MySQL. Tento
        databázový engine je nejvíce kompatibilní s moderními nástupci
        webové technologie (zejména Reactu, viz další podsekce), které
        v současnosti dominují internetu.
    \item \textbf{SQLite:} Jednoduchý a nenáročný databázový model určený
        pro lokální aplikace malé velikosti. Předností tohoto modelu
        je skutečnost, že běží přímo v dané aplikaci a nepotřebuje tedy
        zvlášť server pro správu dat. Nevýhodou je malá škálovatelnost.
\end{itemize}

Pro svou aplikaci jsem si vybral \textbf{MySQL}, jelikož poskytuje širokou
kompatibilitu, je jednoduchý k použití, všestranný a mám s ním již předchozí
zkušenost. Jedná se o průkopníka v moderních databázových systémech, který
vydláždil cestu jeho nástupcům. MySQL používá SQL pro komunikaci s aplikací,
což je strukturovaný dotazovací jazyk nápadně podobný angličtině.

\subsection{Frontend}

Frontend je tváří celé aplikace. Jedná se o tu část aplikace, kterou vidí
její uživatel. Tato část zpracovává data z databáze a zprostředkovává je
uživateli pro jeho další zpracování. Existuje mnoho webových technologií
založených na různých programovacích a značkovacích jazycích, které jsou
na internetu široce používané. Vyjmenoval bych zejména tyto nejpoužívanější:

\begin{itemize}
    \item \textbf{Statické HTML/CSS:} Jedná se čistě o statické webové
        stránky, tak jak jsou uložené na webovém serveru. Jednoduchým
        HTTP requestem si uživatel prostřednictvím webového prohlížeče
        zažádá stránku, webový server najde její odpovídající soubor,
        a obsah stránky zašle do počítače uživatele. Výhoda je extrémní
        jednoduchost, nevýhodou je, že nelze vytvářet složitější aplikace,
        které vyžadují určitou interaktivitu a případně nějaké složitější
        výpočty. Jedná se o první webovou technologii.
    \item \textbf{HTML/CSS s JS:} Malý upgrade z předchozí technologie,
        nyní obohacená o jednoduchou formu programování na webu. JavaScript
        nabízí možnosti, jak stránku doplnit o různé efekty a interaktivitu,
        a umožňuje jednoduché algoritmizační úkony. Výhodou je jednoduchost
        a jistá forma interaktivizace. Nevýhodou je fakt, že veškerý programový
        kód aplikace běží v počítači uživatele, a tak není možné pracovat
        s databází na serveru, aniž by došlo k úniku autentizačních údajů.
    \item \textbf{PHP:} Jedná se o doplňkový software v současnosti nejvíce
        využívaný společně s webovým serverem Apache a Nginx, a zároveň
        se jedná o programovací jazyk, ve kterém jsou psané webové aplikace.
        PHP může sloužit zároveň jako backend i frontend. Uživateli tento
        software posílá data nejčastěji ve formátu HTML/CSS/JS, ale může
        využívat i některé formáty serializovaných dat (např. XML nebo JSON),
        případně obrázkové, video či audio formáty. Výhodou je skutečnost,
        že tento software umožňuje část aplikace spouštět na serveru
        a uživatel tak uvidí konečný výstup, případně částečně zpracovaný
        výstup, se kterým lze posléze manipulovat na uživatelské straně.
        Nevýhodou je poměrně složitější a obtížnější serializace dat, která
        je způsobená zejména stářím standardu. V PHP existuje několik
        frameworků, ale jen jeden nejpoužívanější framework je univerzální
        pro širokou škálu různých projektů, které vyžadují vývoj vlastních
        funkcí: \textbf{Laravel}. Tento framework umožňuje tvorbu originální
        webové aplikace v PHP, kde jsou k dispozici vývojářské nástroje
        pro správu dat, jejich čtení, zápis a další zpracování. Nevýhodou
        tohoto frameworku je chybějící nástroje pro rychlý webdesign.
        Na druhou stranu však má vývojář volnou ruku ve volbě vlastního
        řešení.
    \item \textbf{Node.js:} Node.js lze úspěšně použít nejen pro backend,
        ale také i pro frontend. V kombinaci s dalšími frameworky, které
        umožňují jednodušší tvorbu webové aplikace, se jedná o moderní
        způsob poskytování webového obsahu. Toto jsou nejpoužívanější
        frameworky:
        \begin{itemize}
            \item \textbf{React:} Základní průkopník moderního webu. Od tohoto
                frameworku jsou odvozené všechny další dnes nejpoužívanější
                frameworky.
            \item \textbf{Next.js:} Moderní React framework vyvíjený společností
                Vercel. Výhodou je jednoduchost použití a nativní inkluze
                moderních CSS frameworků, jako např. Tailwind CSS nebo CSS
                Modules.
            \item \textbf{Vue.js:} Konkurenční React framework oproti Next.js.
                Jedná se o progresivní framework, který poskytuje jednoduché
                vývojářské nástroje pro frontend.
            \item \textbf{Nuxt.js:} Odvozený z Vue.js, doplňuje jej o full-stack
                nástroje, lze jej tedy využít i jako backend.
        \end{itemize}
\end{itemize}

Pro mou aplikaci jsem se rozhodl využít Node.js s využitím frameworku Next.js.
S Node.js již mám nějaké předchozí zkušenosti, JavaScript je poměrně jednoduchý
programovací jazyk, z něj odvozený TypeScript navíc přidává jednoduché
zabezpečení aplikace z hlediska datových typů. Next.js se mi jeví jako optimální
volba, vzhledem k jednoduchosti a dostupnosti různých předpřipravených nástrojů
pro rychlý webový design.

\endinput
