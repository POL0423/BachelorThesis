\chapter{Závěr}
\label{sec:conclusion}

Tvorba projektu byla poměrně náročná, nicméně velmi zajímavá.
Nejen, že~jsem využil něco z~toho, co již znám, ale~vyzkoušel jsem~si
i~něco nového, a~spoustu se~toho při~tvorbě této aplikace naučil.
Způsobů, jakým se~dají tvořit moderní webové stránky, je~nepřeberné
množství. Lze využít staré osvědčené metody, ale technologie a~trend
se~ubírají novým směrem a~posouvají~se kupředu. I~z~toho důvodu je~nutné
se~neustále vzdělávat a~držet krok s~dobou. Je to jedno z~mála, co~člověk
může udělat pro~to, aby~si~udržel přehled a~nezaostával za~ostatními.

Pokud bych měl zhodnotit splnění bodů, tak si~myslím, že~ačkoliv
se~při~rešerši a~vývoji vyskytly nečekané potíže, dokázal jsem~si s~nimi
poradit a~projekt dotáhnout zdárně do~úspěšného konce.

Musím zdůraznit jednu věc: Tento projekt je~pouze demonstrační.
Účelem bylo vyvinout webovou aplikaci, která pravidelně aktualizuje
ceny PHM, a~na~základě zadané geografické lokace je~porovnává, filtruje
a~nabízí uživateli. Pro~demonstraci byly nakonec vybrány 2~sítě čerpacích
stanic, ze~kterých jsou data získávána. Původním záměrem bylo sice použití
8~ČS (následně se~počet rozšířil na~9), ale při zkoumání dostupnosti dat
jsem zjistil, že~pouze 3~z~těchto čerpacích stanic nabízí data v~rozumném
formátu a~nakonec jsem zjistil během vývoje, že~jen 2~čerpací stanice
lze reálně použít. Problém by~nastal, kdyby dostupná byla jen 1~ČS.
Tím, že~dostupné jsou 2, je~porovnání cen možné. A~to~je základní předpoklad
k~funkční aplikaci porovnávače cen.

\begin{enumerate}
    \item \textbf{Motivace a~přehled podobných projektů.}
        
        Tato část zkoumá hnací impuls, který mě~vedl k~výběru tohoto
        tématu a~také podobné projekty, které jsou mému nápadu podobné.
        Popsal jsem tam webové stránky a~aplikace, které nabízí srovnání
        cen PHM na~ČS po~celé ČR nebo dokonce v~celé Evropě.

    \item \textbf{Technologie pro~návrh webových aplikací.}

        V~této části se~věnuji rozboru používaných technologií
        pro~návrh a~implementaci webových aplikací, kde~porovnávám
        jednotlivé knihovny a~frameworky, zhodnocuji jejich
        výhody a~nevýhody a~popisuji jejich typické využití.
        V~této části jsem využil umělé inteligence, konkrétně
        ChatGPT, k~rešerši. Tvrzení v~této kapitole jsou řádně
        citována, zdroj informací je~uveden jak z~ChatGPT, tak
        ze~zdrojů, které k~tvrzení umělá inteligence připojila.

    \item \textbf{Popis návrhu a~implementace.}

        Od~příprav, přes vývoj až~po~finální úpravy a~spuštění aplikace.
        Tato část popisuje postup tvorby jednotlivých dílčích součástí
        aplikace, jejich spojení, kontejnerizace a~následné finální
        spuštění na~VPS. Hlavním OS je~Rocky Linux, kde~je~nainstalován
        Nginx pro~provoz reverse proxy a~Docker, který obsahuje 3~kontejnery
        obsluhující jednotlivé části. Jeden kontejner je zaměřen
        na~databázi, další obsahuje crawler, který se~spouští periodicky
        každý den ve~3:00, a~poslední obsahuje webový frontend, který data
        z~databáze filtruje, řadí a~zobrazuje uživateli v~prohlížeči.
        Během vývoje aplikace jsem~si v~některých částech pomohl umělou
        inteligencí, konkrétně GitHub Copilotem, který mi~řešil obtížné
        chyby, jenž se~při~vývoji vyskytly, a~jejichž tradiční ladění
        by~znamenalo příliš velké prodloužení vývoje aplikace.

    \item \textbf{Zhodnocení dosažených výsledků.}

        Nebudu lhát, během vývoje došlo k~několika problémům, které
        finální podobu posunuly trochu jinak, než jsem původně očekával.
        Ve~všech případech si~však myslím, že~cíl bakalářské práce
        byl~splněn. Webová aplikace sice nevypadá tak, jak jsem si
        původně představoval, ale~základní funkcionalita byla dodržena.
        Crawler periodicky spouští procházení webu a aktualizuje databázi.
        Webový frontend nabízí uživateli možnost vyhledávání geografické
        lokace pomocí formuláře společně se~specifikací maximálního okruhu
        pro~porovnání, včetně možnosti změny řazení výsledků dle ceny
        (od~nejlevnějších nebo od~nejdražších) a~filtrování výsledků
        na~základě typu paliva či~jiného doplňkového produktu z~tankovacích
        stojanů či~jejich kvality. GUI webové aplikace je~sice minimální,
        ale~uživatelsky přívětivé a~přehledné.
\end{enumerate}

Ještě jednou pro~jistotu uvádím, že~jsem nad~rámec zadání bakalářské práce
po~dohodě s~vedoucím zařadil do~nabídky PHM k~porovnání také doplňkové
produkty dostupné z~tankovacích stojanů, které by~mohly řidiče zajímat.
Jedná~se zejména o~AdBlue (syntetickou močovinu) a~kapalinu do~ostřikovačů.
Tedy provozní kapaliny, které ze~své podstaty nejsou palivem, nicméně se~jedná
o~provozní kapaliny, které jsou nutné k~bezpečnému a~ekologickému provozu
automobilu.

Aplikaci by~samozřejmě šlo vylepšit a~případně přepracovat. Nabízí~se např.
využití umělé inteligence k~vyhledávání cen PHM a~dalších doplňkových
produktů a~služeb. Jiná možnost by~byla použití síly veřejnosti, která
by~mohla informace o~aktuálních cenách paliv a~doplňkových produktů a~služeb
doplňovat. Pokud jde o~prezentaci výsledků vyhledávání, tak více možností
pro~řazení, filtrování podle více typů a~kvalit produktů, apod. Geografické
určení uživatele by~mohlo případně probíhat i~geolokační službou prohlížeče.
Všechny tyto funkce jsou však již nad~rámec zadání této bakalářské práce
a~nejsou nezbytně nutné ke~splnění základních požadavků. Tyto možnosti však
lze nechat otevřené pro~případný výzkum v~rámci navazujícího studia
v~diplomové práci.

\endinput
