\chapter{Technologie pro návrh webových aplikací}
\label{ch:initial-research}

Tato kapitola se popisuje technologie pro návrh webových aplikací, 
porovnává je, hodnotí výhody a nevýhody a zabývá se vhodným použitím
jednotlivých technologií podle potřeb vývojáře.

Webové aplikace jsou složené z různých vrstev a součástí, které každá
využívá různé technologie, ať už se jedná o uživatelské rozhraní (frontend),
aplikační logiku (backend), úložiště zpracovávaných dat (databáze),
rozhraní, které je propojuje mezi sebou (API) nebo pomocné nástroje,
jako jsou například webové crawlery, které prohledávají web a stahují
webový obsah. Základem k tvorbě webové aplikace je porozumění jednotlivým
součástem a technologiím pro informovanou volbu a implementaci.
\cite{YHVfLHsNlUItkF6G} %Ref: GPT

\section{Frontend}
\label{sec:research-frontend}

V začátcích webu se webové stránky tvořily jen pomocí statického HTML a CSS,
které byly začátkem 90. let standardizovány. Postupně byla funkce doplněna
o JavaScript, který přidal možnost interaktivity. Vznikaly první JS knihovny,
které přinášely ještě více funkcí. Jednou z takových starších knihoven
je např. \emph{jQuery}, která zavedla jednoduchou manipulaci s DOM.
Komunitní průzkum StackOverflow z roku 2024 ukazuje, že jQuery v té době
stále používalo kolem 22~\% profesionálních vývojářů
\cite{YHVfLHsNlUItkF6G,w6F4OYb0neliWLGP}. %Ref: GPT/SOSurvey
Ke stylizované podobě webových stránek se ve velké míře využívaly CSS
frameworky, jakým je např. \emph{Bootstrap} (2011) nebo \emph{Foundation}
\cite{YHVfLHsNlUItkF6G,Nsz2b52zS5s0JglP}, %Ref: GPT/StateOfCSS
které obsahují hotové šablony a mřížkové systémy k tvorbě snadného mobilního
designu. Jejich velkou nevýhodou je však zvětšování objemu přenášených dat.

Od poloviny druhé dekády 21. století se stále více využívají komponentní
frameworky a knihovny, které umožňují tvorbu bohatých webových aplikací,
nejčastěji ve formě SPA. Mezi ty nejrozšířenější patří \emph{React},
který byl v roce 2013 vytvořen společností Facebook (dnes Meta), dále
mohou vývojáři využít ve svých aplikacích \emph{Angular}, který v roce
2016 poprvé představila společnost Google, a následuje \emph{Vue.js},
který je vývojářům k dispozici od roku 2014 díky společnosti Evan You.
\cite{YHVfLHsNlUItkF6G,1WL9hIh67tHjtVTy} %Ref: GPT/BrowserStack

React přišel s konceptem virtuálního DOM, čímž zvyšuje výkon a umožňuje
definici opakovaně použitelných komponent, čímž zjednodušuje údržbu
a škálovatelnost kódu aplikace. Angular je plnohodnotným frameworkem, který
poskytuje rozsáhlý ekosystém vestavěných modulů (např. HTTP, routing, aj.),
a díky tomu je považován za vhodný pro velké podnikové aplikace. Nevýhodou
je nutná znalost TypeScriptu, což začínajícím vývojářům ztěžuje jejich
aklimatizaci do tohoto prostředí. Vue.js sází na modularitu. Aplikaci
vytvořenou ve Vue.js lze nasadit postupně a je méně náročná na výchozí
konfiguraci. Vue.js má rovněž poměrně silnou komunitu.
\cite{YHVfLHsNlUItkF6G,1WL9hIh67tHjtVTy} %Ref: GPT/BrowserStack

Různé průzkumy odráží popularitu těchto nástrojů. Dle průzkumu StackOverflow
z roku 2024 React použilo kolem 41,6~\% profesionálních vývojářů, zatímco
na Angular si sáhlo 19,4~\%. U studentů nebo začínajících vývojářů se podíly
snižují a obrací \cite{YHVfLHsNlUItkF6G,w6F4OYb0neliWLGP}. %Ref: GPT/SOSurvey
Jiné zdroje obsahují podobné výsledky. React pohánělo v dubnu 2025
přes 34 milionů webů, Vue.js zvolili vývojáři 3,7 milionu webů a Angular
je motorem pouhých 96 tisíc webů \cite{YHVfLHsNlUItkF6G,1WL9hIh67tHjtVTy}.
%Ref: GPT/BrowserStack

Za zmínku stojí také kompilovaný framework \emph{Svelte}, který
má minimální overhead, a standardizovaná knihovna \emph{Web Components},
která implementuje vlastní HTML prvky.

\subsection{Srovnání výhod a nevýhod}
\label{sec:frontend-advatages-disadvantages}

\begin{itemize}
    \item \textbf{React} má velkou komunitní podporu a rozsáhlý ekosystém
        různých doplňkových frameworků a knihoven (např. Redux nebo Next.js)
        a umožňuje univerzální použití na webu i v mobilní aplikaci
        prostřednictvím frameworku \emph{React Native}.
        
        \begin{itemize}
            \item \textbf{Výhodou} jsou především rychlé aktualizace
                pomocí virtuálního DOM a přehledná komponentní
                architektura. \cite{YHVfLHsNlUItkF6G,1WL9hIh67tHjtVTy}
                %Ref: GPT/BrowserStack
            \item \textbf{Nevýhodou} může být nutnost učit se JSX syntaxi
                a časté změny v knihovně a ekosystému.
        \end{itemize}
    \item \textbf{Angular} klade důraz na průmyslové aplikace a nabízí
        k tomu kompletní rámec nástrojů. Od základu nabízí silnou kontrolu
        správných datových typů prostřednictvím TypeScriptu, nástroje
        typu \emph{Dependency Injection} a rozsáhlou dokumentaci.
        \cite{YHVfLHsNlUItkF6G,1WL9hIh67tHjtVTy} %Ref: GPT/BrowserStack

        \begin{itemize}
            \item \textbf{Výhodou} je vhodné použití pro velké týmy
                a složité aplikace, které jsou průmyslového charakteru.
                \cite{YHVfLHsNlUItkF6G,1WL9hIh67tHjtVTy}
                %Ref: GPT/BrowserStack
            \item \textbf{Nevýhodou} je naproti tomu složitost a množství
                konceptů (zejména směrování, moduly a služby), které
                zvyšují náročnost vývoje.
                \cite{YHVfLHsNlUItkF6G,1WL9hIh67tHjtVTy}
                %Ref: GPT/BrowserStack
        \end{itemize}
    \item \textbf{Vue} je lehčí na pochopení a hodí se i pro menší projekty.

        \begin{itemize}
            \item \textbf{Výhodou} je všestrannost, podporuje rychlý start
                i postupné rozšiřování aplikace.
                \cite{YHVfLHsNlUItkF6G,1WL9hIh67tHjtVTy}
                %Ref: GPT/BrowserStack
            \item \textbf{Nevýhodou} je menší tržní podíl oproti Reactu
                a Angularu, což se odráží ve velikosti ekosystému knihoven.
                \cite{YHVfLHsNlUItkF6G,1WL9hIh67tHjtVTy}
                %Ref: GPT/BrowserStack
        \end{itemize}
    \item \textbf{Další knihovny} (např. Svelte, Alpine, Lit, apod.)
        stále rostou v oblibě díky svému výkonu. Svelte toho dosahuje
        dopřednou kompilací kódu, zatímco Alpine pro změnu nabízí
        deklarativní jazyk umožňující jednoduché interakce. Jejich velkou
        nevýhodou je však malá adopce v komunitě vývojářů.
\end{itemize}

Frontendové knihovny a frameworky dnes využívají i přídavné technologie,
které zlepšují škálovatelnost, stabilitu a bezpečnost. Častěji se využívá
např. TypeScript, který zlepšuje bezpečnost kódu zavedením kontroly
datových typů při sestavování aplikace, CSS-in-JS a utility frameworky,
které zefektivňují stylování (např. Tailwind CSS), a PWA, které přístupy
považují za service workery. Dále se uplatňují koncepty SSR či SSG (Next.js),
které zlepšují SEO a zrychlují načítání. Veškeré tyto volby závisí
na konkrétních potřebách aplikace a týmu, který ji vytváří. Dnes vývojáři
zpravidla volí moderní JS/TS nástroje, zejména React, Angular nebo Vue,
pro nové projekty.

\section{Backend}
\label{sec:research-backend}

Backend tradičně zajišťovaly skriptovací jazyky a aplikační servery,
které existují již od začátku 90. let a stále se ve velké míře používají
i dnes. Již v 90. letech byl pro jednodušší weby dominantí PHP,
často v kombinaci označované jako \emph{LAMP stack} (Linux, Apache, MySQL
a PHP). Dodnes se v PHP využívají frameworky jako \emph{Laravel} nebo
\emph{Symfony}. Dříve se používaly také CGI skripty psané v Perlu nebo C,
případně Java servlety (JSP). Microsoft zavedl jazyk ASP, který později
nahradili jazykem ASP.NET, který nyní nese název .NET Core. Na přelomu
tisíciletí se vynořil agilní rámec \emph{Ruby on Rails}, který urychloval+
vývoj. Mnohé klasické aplikace byly tvořeny pomocí těchto technologií.

Moderní backend dnes zajišťují architektury, které jsou založené na REST
microservices, kontejnerizaci a cloudových službách. Velká část vývojářů
přechází na jazykové frameworky, které umožňují rychlý vývoj a dobrou
škálovatelnost.

\begin{itemize}
    \item \textbf{Node.js/JS} umožňuje spouštět JS kód na serveru
        \cite{YHVfLHsNlUItkF6G, kbr6yxw1ew4wJS2e}. %Ref: GPT/Node
        Jedná se o otevřené prostředí založené na V8 enginu z prohlížeče
        Chrome, které efektivně zpracovává I/O v jednovlánové smyčce
        událostí \cite{YHVfLHsNlUItkF6G,kbr6yxw1ew4wJS2e}. %Ref: GPT/Node
        Tak umožňuje bez nutnosti více vláken obsloužit až tisíce
        souběžných spojení. Pro Node bylo vytvořeno velké množství
        různých frameworků (např. \emph{Express.js} využívané pro REST API,
        \emph{NestJS} určené pro strukturované aplikace, Koa, Fastify aj.).
        Dle průzkumu StackOverflow z roku 2024 byl Node.js nejpoužívanější
        webovou technologií, kterou využívalo okolo 40,7~\% profesionálních
        vývojářů \cite{YHVfLHsNlUItkF6G,w6F4OYb0neliWLGP}. %Ref: GPT/SOSurvey
    \item \textbf{Python} je oblíbený nástroj pro rychlý vývoj, díky
        čitelnému kódu a silným knihovnám pro databáze a statistické
        využití. Nejznámějšími frameworky vyvíjenými v Pythonu jsou
        \emph{Django}, které je full-stack frameworkem s ORM a je vhodné
        pro ryhlé prototypování, a \emph{Flask}, který je mikroframeworkem
        a je flexibilní pro služby menší velikosti. Dle statistik průzkumu
        StackOverflow z roku 2024 používá Django okolo 11,4~\% vývojářů,
        zatímco Flask si vybralo zhruba 11,6~\% vývojářů
        \cite{YHVfLHsNlUItkF6G,w6F4OYb0neliWLGP}. %Ref: GPT/SOSurvey
        Python však na serveru bývá pomalejší oproti Go nebo Node, kvůli
        čemuž se často kombinuje s vyrovnáváním zátěže, případně
        asynchronními rozšířeními.
    \item \textbf{Java/Kotlin} jsou díky stabilitě a výkonu v podnikovém
        prostředí stále oblíbené. Hlavní platformou je \emph{Java EE/Spring}
        (Spring Boot), což jsou robustní frameworky obsahující množství
        různých modulů pro transace, zabezpečení a messaging. Spring Boot
        dále umožňuje rychle konfigurovat servery s REST API. V současnosti
        hlavně díky Androidu a serverovým knihovnám (díky Spring podpoře)
        rychle roste \emph{Kotlin}. Spring Boot podle průzkumu StackOverflow
        využívá okolo 14,2~\% vývojářů
        \cite{YHVfLHsNlUItkF6G,w6F4OYb0neliWLGP}. %Ref: GPT/SOSurvey
        Nevýhodou je především náročnější nasazení na JVM a větší spotřeba
        paměti a výpočetního výkonu.
    \item \textbf{C\#/.NET} jsou koncepty, které Microsoft poprvé
        představil v roce 2002. Dnešní cross-platform verze nese název
        \emph{ASP.NET Core} a představuje moderní rámec pro webové služby.
        Jeho hlavními výhodami jsou vysoká integrace s Microsoft ekosystémem
        a dobrý výkon. Dle průzkumu StackOverflow .NET Core používá
        zhruba 19,1~\% vývojářů
        \cite{YHVfLHsNlUItkF6G,w6F4OYb0neliWLGP}. %Ref: GPT/SOSurvey
        Nevýhodou může být zejména nutnost spravovat CLR, někdy také
        vyšší náročnost na spotřebu paměti.
    \item \textbf{Dalšími jazyky} jsou např. \emph{Go}, které se díky
        jednoduchosti a vysoce konkurenčnímu běhu používá pro microservices
        nebo \emph{Rust}, který do vývojářských nástrojů proniká díky
        bezpečnému a vysoce výkonnému kódu (používají ho např. frameworky
        Actix nebo Rocket). Méně časté, ale stále přítomné jsou
        \emph{Ruby/Ruby on Rails}, který byl kdysi na vrcholu, ale nyní
        v popularitě zaostává, nebo \emph{PHP}, které se současnými
        frameworky stále silně zůstává v CMS ekosystému.
\end{itemize}

Dle údajů používaných technologií z průzkumu StackOverflow z roku 2024
jsou mezi backendovými frameworky kromě Node.js populární také Spring
Boot, ASP.NET Core, Django a Flask. Node.js dle průzkumu využívá zhruba
40~\% vývojářů a React frontend používá 41~\%. Laravel běžící na PHP
má zastoupení okolo 8,6~\% a Ruby on Rails pouhých 5~\%. Tyto trendy
potvzují, že vývojáři si v současnosti často vybírají více moderní
a aktivně udržované nástroje, které mají velkou komunitu.
\cite{YHVfLHsNlUItkF6G,w6F4OYb0neliWLGP} %Ref: GPT/SOSurvey

Co se týče výhod a nevýhod, lze říct, že dynamické jazyky (JS/Node,
Python, Ruby či PHP) umožňují rychlý vývoj, ale díky interpretované
podobě mohou mít nižší výkon. Naproti tomu kompilované jazyky
(Java/Kotlin, C\#, Go či Rust) nabízejí od základu vyšší průchodnost,
efektivní spouštění kódu a kontrolu datových typů, ale někdy díky
tomu zpomalují vývoj. Frameworky, jako jsou Spring nebo ASP.NET,
jsou robustní a díky své vysoké spolehlivosti vhodné pro korporátní
využití \cite{YHVfLHsNlUItkF6G,w6F4OYb0neliWLGP}, %Ref: GPT/SOSurvey
zatímco mikroframeworky (Flask či Express) jsou snadnější k pochopení,
ale vyžadují více tvorby vlastních vestavěných knihoven v aplikaci.
Volba technologie je silně závislá na velikosti projektu, jeho
požadavcích na výkon či škálovatelnost a znalostech vývojářského
týmu.

\section{Databáze}
\label{sec:research-db}

Tradiční formou úložiště dat je relační databáze definující strukturovaná
data v tabulkách, které mají pevně definované schéma. Příklady RDBMS mohou
být \emph{MySQL}, \emph{PostgreSQL}, \emph{Oracle DB}, \emph{Microsoft
SQL Server} nebo \emph{SQLite}. Tyto systémy zajišťují dobré ACID vlastnosti
a mají dobrou podporu složitých SQL dotazů. Výhodou je silná integrita dat,
která zahrnuje klíče a normalizaci, a inkluze standardních nástrojů
pro zálohování. Nevýhody relačních databází spočívají ve škálovatelnosti.
Data jsou škálována vertikálně v závislosti na výkonu hardwaru a podle
rigidního schématu, kde změna datového modelu vyžaduje migraci dat.
\cite{YHVfLHsNlUItkF6G,Fny73hg0lVaoqYAl} % Ref: GPT/MongoDB

Modernější modely (někdy označované jako \emph{NoSQL} nebo
\emph{multimodely}) se využívají pro velké objemy nestrukturovaných nebo
rychle se měnících dat. Tyto systémy často umožňují i horizontální
škálování. NoSQL databáze se dělí na:

\begin{itemize}
    \item \textbf{dokumentové,} které ukládají data do JSON objektů
        (např. MongoDB nebo CouchDB);
    \item \textbf{typu klíč-hodnota} používané pro jednoduché páry
        (např. Redis);
    \item \textbf{sloupcové} vhodné pro analytické využití
        (např. Cassanda nebo HBase)
    \item a \textbf{grafové} sloužící pro propojená data
        (např. Neo4j či Amazon Neptune).
\end{itemize}

Tyto systémy mají flexibilní schéma umožňující měnit strukturu dat
za běhu a vyšší dostupnost, jelikož jsou navržené pro cloud a horizontální
replikaci \cite{YHVfLHsNlUItkF6G,Fny73hg0lVaoqYAl}. %Ref: GPT/MongoDB
Silné ACID vlastnosti však často obětují za rychlost a škálovatelnost.
NoSQL databáze vynikají v případech, kdy je nutné zohlednit vysokou
propustnost nebo ukládat data, která mají volnější strukturu
(např. neomezené sloupce či vnořené objekty). Kupříkladu MongoDB obsahuje
vestavěnou podporu pro distribuci dat (tzv. sharding) a replikaci.
Naproti tomu Redis nabízí extrémně rychlou cache párů typu klíč-hodnota.

Žebříček DB-Engines z června roku 2025 popisuje popularitu databází,
kde nejvyšší příčky obsadily databáze Oracle, MySQL, MS SQL Server
a PostgreSQL, tedy tradiční RDBMS. Na 6. místě se nachází MongoDB,
nejpopulárnější NoSQL databáze, která má robustní adopci, dále pak
Redis, založené na typu klíč-hodnota, a ElasticSearch, které nabízí
fulltextové vyhledávání. Grafové databáze (např. Neo4j) se na špičce
žebříčku vyskytují, přestože jsou specifické určitým doménám
(sociální sítě či doporučovací systémy). Vysoký nárůst nově zaznamenaly
cloudové databáze, jako např. Snowflake nabízející datové sklady či služby
typu Amazon DynamoDB, které nabízí databáze typu klíč-hodnota
pro bezserverové architektury. \cite{YHVfLHsNlUItkF6G,gT0jW3Rz4pdfcjnO}
%Ref: GPT/DBEngines

Na další stránce (\pageref{tab:db-overview}) můžete vidět
tabulku \ref{tab:db-overview}, která srovnává typy databází.

\begin{table}
    \caption{Srovnání typů databází}
    \label{tab:db-overview}
    \centering
    \begin{tabular}{|p{.2\textwidth}|p{.15\textwidth}|p{.55\textwidth}|}
        \hline
        \thead{\textbf{Typ databáze}}
            &   \thead{\textbf{Příklady}}
            &   \thead{\textbf{Vlastnosti}}\\
        \hline  \hline
        Relační (SQL)
            &   \parbox[t]{\textwidth}{MySQL,\\ PostgreSQL,\\ Oracle}
            &   Ukládají data v tabulkách s pevnou stukturou
                (označované jako schéma). Zaručují ACID transakce, které ručí
                spolehlivostí a konzistencí. Jsou vhodné pro normalizovaná
                data a složité relační dotazy. Jejich nevýhodou je méně
                flexibilní schéma a škálovatelnost omezená na vertikální.
                \cite{YHVfLHsNlUItkF6G,Fny73hg0lVaoqYAl}\\ %Ref: GPT/MongoDB
        \hline
        Dokumentová (NoSQL)
            &   \parbox[t]{\textwidth}{MongoDB,\\ CouchDB,\\ Firestore}
            &   Data se ukládají jako JSON nebo binární
                dokumenty. Umožňují flexibilní schéma, dají se snadno horizontálně škálovat a jsou rychlé ve čtení i zápisu dat. Hodí se pro nestandardizovaná nebo rychle se měnící data.
                \cite{YHVfLHsNlUItkF6G,Fny73hg0lVaoqYAl}\\ %Ref: GPT/MongoDB
        \hline
        Klíč-hodnota (NoSQL)
            &   \parbox[t]{\textwidth}{Redis,\\ DynamoDB}
            &   Extrémně jednoduchá úložiště, kde každý záznam je pár
                typu klíč-hodnota. Podporují velmi rychlé operace
                a jsou snadno škálovatelné. Často jsou používané jako
                cache nebo pro ukládání jednoduchých stavů.\\
        \hline
        Sloupcové (NoSQL)
            &   \parbox[t]{\textwidth}{Cassandra,\\ HBase}
            &   Jsou optimalizované pro distribuci a analýzý velkých dat.
                Data jsou ukládána po řádcích rozdělené ve sloupcových
                rodinách. Databáze lze škálovat na stovky uzlů.\\
        \hline
        Grafové (NoSQL)
            &   \parbox[t]{\textwidth}{Neo4j,\\ Amazon\\ Neptune}
            &   Tyto databáze data modelují jako uzly a hrany s atributy.
                Jsou ideální pro úlohy se složitými vztahy (doporučovací
                systémy či sociální sítě). Nejdůležitější grafovou
                databází je Neo4j
                \cite{YHVfLHsNlUItkF6G,gT0jW3Rz4pdfcjnO}. %Ref: GPT/DBEngines
                Grafové databáze jsou všeobecně méně obvyklé, jak
                značí jejich nižší podíl na trhu.\\
        \hline
    \end{tabular}
\end{table}

\newpage

\section{API}
\label{sec:research-api}

Klasickým přístupem pro webová rozhraní (API) byl SOAP, který byl protokolem
založeným na XML s WSDL kontraktem. SOAP je robustní, jazykově a platformově
nezávislý a obsahuje zabudované bezpečnostní rozšíření \emph{WS-Security},
díky čemuž se často používal v ekonomických odvětvích, např. bankovnictví
nebo korporátní integrace. Díky využití formátu XML je však velmi náročný
na šířku pásma, neboť XML zprávy obsahují spoustu metaúdajů.
\cite{YHVfLHsNlUItkF6G,Sj7FFY7SXnJ6m41T} %Ref: GPT/Altex

Od přelomu tisíciletí se novým standardem pro webová API stal REST, který
definoval soubor architektonických zásad, jako např. klient-server, stateless,
jednotné rozhraní, caching apod. REST API data v praxi vystavují jako zdroje
na unikátních URL a používají běžně užívané standardní HTTP metody (GET, POST,
PUT a DELETE). Díky své jednoduchosti a kompatibilitě se současným webem
je REST velmi populární. Odpovědi jsou obvykle ve formátu JSON, který je méně
datově náročný než XML, a umožňují cache na úrovni HTTP. Výhody RESTu obvykle
spočívají ve volném spárování klienta a serveru, kdy je lze vyvíjet nezávisle
na sobě, a v možnostech využití infrastruktury HTTP (zejména caching,
autentizace a load balancing). Mezi nevýhodami jsou nejednoznačná struktura,
neboť neexistuje univerzální šablona určená k modelování zdrojů a konkrétní
implementace záleží na návrhu, a problém s nadbytečností či nedostatkem dat
(klient někdy dostane buď příliš mnoho nebo příliš málo informace,
což může vést k dodatečným požadavkům nebo objemnému množství dat).
\cite{YHVfLHsNlUItkF6G,Sj7FFY7SXnJ6m41T} %Ref: GPT/Altex

V roce 2015 Facebook představil GraphQL, dotazovací jazyk a runtime pro API
umožňující klientovi specifikovat přesně data, která požaduje. Místo různých
statických endpointů využívá jednoho univerzálního HTTP endpointu, kam klient
posílá dotaz, ve kterém definuje strukturu požadované odpovědi. GraphQL
používá typové schéma a umožňuje např. i realtime subskripce. Primární výhodou
je optimalizace množství dat: klient dostane jen to, co potřebuje a v jednom
volání může získat složitě propojená data (např. položku s jejími komentáři)
\cite{YHVfLHsNlUItkF6G,Sj7FFY7SXnJ6m41T}. %Ref: GPT/Altex
K tomu má GraphQL navíc vestavěný mechanismus verzování, kdy se mění schéma,
ale klienti stále mohou pracovat s jedním endpointem.
Nevýhody spočívají v komplexnosti implementace
\cite{YHVfLHsNlUItkF6G,Sj7FFY7SXnJ6m41T}. %Ref: GPT/Altex
Velké a složité dotazy mohou server přetížit a na rozdíl od čistého RESTu
GraphQL nevyužívá out-of-the-box HTTP caching, díky čemuž musí aplikace cache
budovat vlastní \cite{YHVfLHsNlUItkF6G,Sj7FFY7SXnJ6m41T}. %Ref: GPT/Altex
Též vyžaduje uvést podrobné schéma v SDL a naučit se novou syntaxi,
což zpravidla zvyšuje náklady na vývoj. GraphQL je navzdory tomu populární
pro mobilní a komplexní frontendy, kde ho úspěšně nasazují firmy jako Meta,
GitHub či Shopify.

\newpage

K řešení API tedy lze využít různé vzory. SOAP se v dnešní době využívá
hlavně ve starších systémech, kde jsou požadavky na formální kontrakt
a vysokou úroveň zabezpečení, a REST či JSON zůstává dominantní pro většinu
veřejných i interních webových služeb díky využití jednoduchého HTTP modelu
a široké podpoře. GraphQL je relativně nový formát, který je vhodný tam,
kde klienti potřebují maximální flexibilitu dotazů a konsolidaci dat
z více různých zdrojů, čímž ale také přináší složitější architekturu. Volba
tedy závisí na konkrétním případě: kupříkladu \emph{„REST vyhovuje
pro jednoduché CRUD API s mnoha uživateli“}, zatímco \emph{„GraphQL se hodí
pro mobilní aplikaci, která má různá data přizpůsobená na míru“}
\cite{YHVfLHsNlUItkF6G,Sj7FFY7SXnJ6m41T}. %Ref: GPT/Altex

\section{Webové crawlery}
\label{sec:research-crawlers}

Pod pojmem \emph{webový crawler} (též označovaný jako spider nebo scraper)
si lze představit program či framework, který prochází webové stránky
a stahuje jejich obsah, často za účelem indexace nebo extrakce dat.
Při návrhu crawleru se obvykle rozlišuje, zda stránka vyžaduje pouze
statickž HTTP přenos, nebo ke svému fungování potřebuje i JavaScript.

Tradiční crawlovací skripty (v Pythnu, Javě apod.) využívají různé knihovny,
jako např. \emph{BeautifulSoup}, která funguje jako jednoduchý HTML parser,
\emph{Requests}, která umožňuje stahování obsahu, nebo \emph{java.net}. Tyto
knihovny fungují pro statické webové stránky. Postupně se pro crawlery
na této bázi vyvinuly rámce jako \emph{Apache Nutch}, který je psaný v Javě,
je škálovatelný, a používají ho vyhledávače, nebo specializované knihovny
pro některé programovací jazyky. Crawler na této bázi typicky respektuje
soubor \texttt{robots.txt}, který určuje, které cesty smí crawler procházet
a které nikoliv \cite{YHVfLHsNlUItkF6G,adi8S69Mmo0Mi7FC}, %Ref: GPT/GoogleDev
a řídí rychlost požadavků, aby jimi nezahltil webový provoz.

S rostoucím množstvím stránek s dynamickým obsahem díky použití JavaScriptu
se začaly objevovat frameworky a knihovny používající headless prohlížeče
a nástroje pro automatizaci, čímž položily základy pro moderní technologie.
Mezi tyto nástroje patří:

\begin{itemize}
    \item \textbf{Selenium:} Jedná se o knihovnu pro ovládání skutečného
        prohlížeče, kterým může být např. Chrome, Firefox apod. Tato knihovna
        umožňuje programovat klikáí, vyplňování formuláře a čekat na spuštění
        JS kódu. Výhodou je, že stránku interpretuje podobně jako uživatel,
        takže zvládne i složité interakce. Nevýhodou je výrazně pomalejší
        běh, neboť se spouští reálný prohlížeč s celou renderovací smyčkou,
        a vyšší spotřeba paměti a výpočetního výkonu. Selenium je k dispozici
        ve více programovacích jazycích (např. Python, Java, C\#, JavaScript
        aj.) a používá se spíše pro menší objemy dat či tam, kde je potřeba
        interaktivita (např. vyplnění přihlašovacích formulářů). Dle odborné
        analýzy je Selenium \emph{„relativně pomalé a náročné na zdroje“}
        a nevhodné pro masivní scraping bez paralelizace.
        \cite{YHVfLHsNlUItkF6G,XDjScR8U0lQdUngn} %Ref: GPT/WebScapingAPI
    \item \textbf{Scrapy:} V Pythonu napsaný asynchronní rámec, který
        je zaměřený přímo na vysokovýkonné stahování webu. Scrapy obsahuje
        vlastní plánovač, správu fronty požadavků, parsování pomocí XPath
        nebo CSS selektorů a pipeline pro zpracování dat. K velmi rychlému
        stahování statických stránek využívá automatické škálování
        a paralelní vykonávání. Nevýhodou je absence podpory JavaScriptu.
        Pokud stránka obsahuje dynamicky generovaný obsah, který se načítá až
        po načtení stránky, je nutné obsah předzpracovat. K tomu je možné
        využít např. Selenium nebo Splash. Scrapy je tedy obvykle ideální
        pro crawling statických stránek nebo stránek generovaných ve statické
        formě (např. sběr produktů, odkazů či textu).
        \cite{YHVfLHsNlUItkF6G,XDjScR8U0lQdUngn} %Ref: GPT/WebScapingAPI
    \item \textbf{Puppeteer:} Knihovna určená pro Noje.js, která automatizuje
        headless Chrome či Chromium prohlížeč. Oproti knihovně Selenium
        nabízí rychlejší a vyšší úroveň ovládání, neboť je speciálně
        optimalizovaná pro Chrome. Puppeteer kromě scrapingu umožňuje také
        generování PDF, screenshotů apod. Výhodou je velmi přesný render
        moderních webů zahrnující také React či Vue aplikace. Nevýhodou
        se podobá knihovně Selenium, neboť je náročná na systémové zdroje.
        Pupeteer je \emph{„podstatně rychlejší než Selenium“}, které
        je komplexnější a multiplatformní, čímž si získává vysokou
        výkonnostní příčku. Běží-li projekt výhradně na prohlížeči Chrome,
        Puppeteer bývá zpravidla doporučenou volbou, díky vysoké rychlosti
        a jednoduché API. \cite{YHVfLHsNlUItkF6G,KVcwMBKpAmczm8Mo}
        %Ref: GPT/Oxylabs
\end{itemize}

Výběr nástrojů závisí na potřebách. Je nutné zhodnotit výhody a nevýhody
jednotlivých knihoven a nástrojů a rozhodnout se na základě nabízených
funkcí. Kromě předešlých moderních nástrojů může být další možností
\emph{Playwright}, který je konkurentem Puppeteeru a podporuje i jiné
prohlížeče, z těch tradičních např. \emph{BeautifulSoup/Jsoup}, který
se hodí pro jednoduché parsování HTML, případně specializované služby,
neboli servery s API optimalizovanou pro scraping.

Na začátek se doporučuje použít jednoduché nástroje (Scrapy pro klasický
crawling) a podle potřeb rozšířit sadu nástrojů o ty typu Selenium nebo
Puppeteer pro stránky vyžadující interakci. V každém případě je však
důležité dodržovat etiketu: respektovat \texttt{robots.txt}
\cite{YHVfLHsNlUItkF6G,adi8S69Mmo0Mi7FC} %Ref: GPT/GoogleDev
a šetřit síťovou kapacitu cílů přidáním prodlevy mezi požadavky.

Crawlery lze použít k různým účelům: k indexaci stránek pro vyhledávače,
sběru dat (market iltelligence či \textbf{agregace cen}) nebo analýze obsahu.
V moderních výzkumech lze vidět, že spojením více API (např. GraphQL)
a webových crawlerů je možné vytvářet sofistikované služby, skrze které
mohou \emph{„klienti získávat data z různých databází a zdrojů přes
jediné API“} \cite{YHVfLHsNlUItkF6G,Sj7FFY7SXnJ6m41T}. %Ref: GPT/Altex
Konečný výběr technologií závisí především na povaze cílového webu (jedná se
o statický či dynamický web, jak velký je objem dat, případně jaké jsou
požadavky na interaktivitu) a také na požadavcích samotného projektu.

\endinput
