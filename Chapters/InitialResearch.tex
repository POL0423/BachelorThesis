\chapter{Technologie pro návrh webových aplikací}

Webové aplikace jsou složené z různých vrstev a součástí, které každá
využívá různé technologie, ať už se jedná o uživatelské rozhraní (frontend),
aplikační logiku (backend), úložiště zpracovávaných dat (databáze),
rozhraní, které je propojuje mezi sebou (API) nebo pomocné nástroje,
jako jsou například webové crawlery, které prohledávají web a stahují
webový obsah. Základem k tvorbě webové aplikace je porozumění jednotlivým
součástem a technologiím pro informovanou volbu a implementaci.
% Ref: ChatGPT konverzace

\section{Frontend}

V začátcích webu se webové stránky tvořily jen pomocí statického HTML a CSS,
které byly začátkem 90. let standardizovány. Postupně byla funkce doplněna
o JavaScript, který přidal možnost interaktivity. Vznikaly první JS knihovny,
které přinášely ještě více funkcí. Jednou z takových starších knihoven
je např. \emph{jQuery}, která zavedla jednoduchou manipulaci s DOM.
Komunitní průzkum StackOverflow z roku 2024 ukazuje, že jQuery v té době
stále používalo kolem 22~\% profesionálních vývojářů. %Ref: GPT/SOSurvey
Ke stylizované podobě webových stránek se ve velké míře využívaly CSS
frameworky, jakým je např. \emph{Bootstrap} (2011) nebo \emph{Foundation},
které obsahují hotové šablony a mřížkové systémy k tvorbě snadného mobilního
designu. Jejich velkou nevýhodou je však zvětšování objemu přenášených dat.

Od poloviny druhé dekády 21. století se stále více využívají komponentní
frameworky a knihovny, které umožňují tvorbu bohatých webových aplikací,
nejčastěji ve formě SPA. Mezi ty nejrozšířenější patří \emph{React},
který byl v roce 2013 vytvořen společností Facebook (dnes Meta), dále
mohou vývojáři využít ve svých aplikacích \emph{Angular}, který v roce
2016 poprvé představila společnost Google, a následuje \emph{Vue.js},
který je vývojářům k dispozici od roku 2014 díky společnosti Evan You.
%Ref: GPT/BrowserStack(R/A/V)

React přišel s konceptem virtuálního DOM, čímž zvyšuje výkon a umožňuje
definici opakovaně použitelných komponent, čímž zjednodušuje údržbu
a škálovatelnost kódu aplikace. %Ref: GPT/BrowserStack(R/A/V)
Angular je plnohodnotným frameworkem, který poskytuje rozsáhlý ekosystém
vestavěných modulů (např. HTTP, routing, aj.), a díky tomu je považován
za vhodný pro velké podnikové aplikace. Nevýhodou je nutná znalost
TypeScriptu, což začínajícím vývojářům ztěžuje jejich aklimatizaci
do tohoto prostředí. %Ref: GPT/BrowserStack(R/A/V)
Vue.js sází na modularitu. Aplikaci vytvořenou ve Vue.js lze nasadit
postupně a je méně náročná na výchozí konfiguraci. Vue.js má rovněž
poměrně silnou komunitu. %Ref: GPT/BrowserStack(R/A/V)

Různé průzkumy odráží popularitu těchto nástrojů. Dle průzkumu StackOverflow
z roku 2024 React použilo kole 41,6~\% profesionálních vývojářů, zatímco
na Angular si sáhlo 19,4~\%. U studentů nebo začínajících vývojářů se podíly
snižují a obrací. %Ref: GPT/SOSurvey
Jiné zdroje obsahují podobné výsledky. React pohánělo v dubnu 2025
přes 34 milionů webů, Vue.js zvolili vývojáři 3,7 milionu webů a Angular
je motorem pouhých 96 tisc webů. %Ref: GPT/BrowserStack(R/A/V)

Za zmínku stojí také kompilovaný framework \emph{Svelte}, který
má minimální overhead, a standardizovaná knihovna \emph{Web Components},
která implementuje vlastní HTML prvky.

\subsection{Srovnání výhod a nevýhod}

\begin{itemize}
    \item \textbf{React} má velkou komunitní podporu a rozsáhlý ekosystém
        různých doplňkových frameworků a knihoven (např. Redux nebo Next.js)
        a umožňuje univerzální použití na webu i v mobilní aplikaci
        prostřednictvím frameworku \emph{React Native}.
        
        \begin{itemize}
            \item \textbf{Výhodou} jsou především rychlé aktualizace
                pomocí virtuálního DOM a přehledná komponentní
                architektura. %Ref: GPT/BrowserStack(R/A/V)
            \item \textbf{Nevýhodou} může být nutnost učit se JSX syntaxi
                a časté změny v knihovně a ekosystému.
        \end{itemize}
    \item \textbf{Angular} klade důraz na průmyslové aplikace a nabízí
        k tomu kompletní rámec nástrojů. Od základu nabízí silnou kontrolu
        správných datových typů prostřednictvím TypeScriptu, nástroje
        typu \emph{Dependency Injection} a rozsáhlou dokumentaci.
        %Ref: GPT/BrowserStack(R/A/V)

        \begin{itemize}
            \item \textbf{Výhodou} je vhodné použití pro velké týmy
                a složité aplikace, které jsou průmyslového charakteru.
            \item \textbf{Nevýhodou} je naproti tomu složitost a množství
                konceptů (zejména směrování, moduly a služby), které
                zvyšují náročnost vývoje. %Ref: GPT/BrowserStack(R/A/V)
        \end{itemize}
    \item \textbf{Vue} je lehčí na pochopení a hodí se i pro menší projekty.

        \begin{itemize}
            \item \textbf{Výhodou} je všestrannost, podporuje rychlý start
                i postupné rozšiřování aplikace.
            \item \textbf{Nevýhodou} je menší tržní podíl oproti Reactu
                a Angularu, což se odráží ve velikosti ekosystému knihoven.
                %Ref: GPT/BrowserStack(R/A/V)
        \end{itemize}
    \item \textbf{Další knihovny} (např. Svelte, Alpine, Lit, apod.)
        stále rostou v oblibě díky svému výkonu. Svelte toho dosahuje
        dopřednou kompilací kódu, zatímco Alpine pro změnu nabízí
        deklarativní jazyk umožňující jednoduché interakce. Jejich velkou
        nevýhodou je však malá adopce v komunitě vývojářů.
\end{itemize}

Frontendové knihovny a frameworky dnes využívají i přídavné technologie,
které zlepšují škálovatelnost, stabilitu a bezpečnost. Častěji se využívá
např. TypeScript, který zlepšuje bezpečnost kódu zavedením kontroly
datových typů při sestavování aplikace, CSS-in-JS a utility frameworky,
které zefektivňují stylování (např. Tailwind CSS), a PWA, které přístupy
považují za service workery. Dále se uplatňují koncepty SSR či SSG (Next.js),
které zlepšují SEO a zrychlují načítání. Veškeré tyto volby závisí
na konkrétních potřebách aplikace a týmu, který ji vytváří. Dnes vývojáři
zpravidla volí moderní JS/TS nástroje, zejména React, Angular nebo Vue,
pro nové projekty. %Ref: GPT

\section{Backend}

Backend tradičně zajišťovaly skriptovací jazyky a aplikační servery,
které existují již od začátku 90. let a stále se ve velké míře používají
i dnes. Již v 90. letech byl pro jednodušší weby dominantí PHP,
často v kombinaci označované jako \emph{LAMP stack} (Linux, Apache, MySQL
a PHP). Dodnes se v PHP využívají frameworky jako \emph{Laravel} nebo
\emph{Symfony}. Dříve se používaly také CGI skripty psané v Perlu nebo C,
případně Java servlety (JSP). Microsoft zavedl jazyk ASP, který později
nahradili jazykem ASP.NET, který nyní nese název .NET Core. Na přelomu
tisíciletí se vynořil agilní rámec \emph{Ruby on Rails}, který urychloval+
vývoj. Mnohé klasické aplikace byly tvořeny pomocí těchto technologií.

Moderní backend dnes zajišťují architektury, které jsou založené na REST
microservices, kontejnerizaci a cloudových službách. Velká část vývojářů
přechází na jazykové frameworky, které umožňují rychlý vývoj a dobrou
škálovatelnost.

\begin{itemize}
    \item \textbf{Node.js/JS} umožňuje spouštět JS kód na serveru.
        %Ref: GPT/NodeIntro
        Jedná se o otevřené prostředí založené na V8 enginu z prohlížeče
        Chrome, které efektivně zpracovává I/O v jednovlánové smyčce
        událostí. %Ref: GPT/NodeIntro
        Tak umožňuje bez nutnosti více vláken obsloužit až tisíce
        souběžných spojení. Pro Node bylo vytvořeno velké množství
        různých frameworků (např. \emph{Express.js} využívané pro REST API,
        \emph{NestJS} určené pro strukturované aplikace, Koa, Fastify aj.).
        Dle průzkumu StackOverflow z roku 2024 byl Node.js nejpoužívanější
        webovou technologií, kterou využívalo okolo 40,7~\% profesionálních
        vývojářl. %Ref: GPT/SOSurvey
    \item \textbf{Python} je oblíbený nástroj pro rychlý vývoj, díky
        čitelnému kódu a silným knihovnám pro databáze a statistické
        využití. NEjznámějšími frameworky vyvíjené v Pythonu jsou
        \emph{Django}, které je full-stack frameworkem s ORM a je vhodné
        pro ryhlé prototypování, a \emph{Flask}, který je mikroframeworkem
        a je flexibilní pro služby menší velikosti. Dle statistik průzkumu
        StackOverflow z roku 2024 používá Django okolo 11,4~\% vývojářů,
        zatímco Flask si vybralo zhruba 11,6~\% vývojářů. %Ref: GPT/SOSurvey
        Python však na serveru bývá pomalejší oproti Go nebo Node, kvůli
        čemuž se často kombinuje s vyrovnáváním zátěže, případně
        asynchronními rozšířeními.
    \item \textbf{Java/Kotlin} jsou díky stabilitě a výkonu v podnikovém
        prostředí stále oblíbené. Hlavní platformou je \emph{Java EE/Spring}
        (Spring Boot), což jsou robustní frameworky obsahující množství
        různých modulů pro transace, zabezpečení a messaging. Spring Boot
        dále umožňuje rychle konfigurovat servery s REST API. V současnosti
        hlavně díky Androidu a serverovým knihovnám (díky Spring podpoře)
        rychle roste \emph{Kotlin}. Spring Boot podle průzkumu StackOverflow
        využívá okolo 14,2~\% vývojářů. %Ref: GPT/SOSurvey
        Nevýhodou je především náročnější nasazení na JVM a větší spotřeba
        paměti a výpočetního výkonu.
    \item \textbf{C\#/.NET} jsou koncepty, které Microsoft poprvé
        představil v roce 2002. Dnešní cross-platform verze nese název
        \emph{ASP.NET Core} a představuje moderní rámec pro webové služby.
        Jeho hlavními výhodami jsou vysoká integrace s Microsoft ekosystémem
        a dobrý výkon. Dle průzkumu StackOverflow .NET Core používá
        zhruba 19,1~\% vývojářů. %Ref: GPT/SOSurvey
        Nevýhodou může být zejména nutnost spravovat CLR, někdy také
        vyšší náročnost na spotřebu paměti.
    \item \textbf{Dalšími jazyky} jsou např. \emph{Go}, které se díky
        jednoduchosti a vysoce konkurenčnímu běhu používá pro microservices
        nebo \emph{Rust}, který do vývojářských nástrojů proniká díky
        bezpečnému a vysoce výkonnému kódu (používají ho např. frameworky
        Actix nebo Rocket). Méně časté, ale stále přítomné jsou
        \emph{Ruby/Ruby on Rails}, který byl kdysi na vrcholu, ale nyní
        v popularitě zaostává, nebo \emph{PHP}, které se současnými
        frameworky stále silně zůstává v CMS ekosystému.
\end{itemize}

Dle údajů používaných technologií z průzkumu StackOverflow z roku 2024
jsou mezi backendovými frameworky kromě Node.js populární také Spring
Boot, ASP.NET Core, Django a Flask. Node.js dle průzkumu využívá zhruba
40~\% vývojářů a React frontend používá 41~\%. Laravel běžící na PHP
má zastoupení okolo 8,6~\% a Ruby on Rails pouhých 5~\%. Tyto trendy
potvzují, že vývojáři si v současnosti často vybírají více moderní
a aktivně udržované nástroje, které mají velkou komunitu.

Co se týče výhod a nevýhod, lze říct, že dynamické jazyky (JS/Node,
Python, Ruby, PHP) umožňují rychlý vývoj, ale díky interpretované
podobě mohou mít nižší výkon. Naproti tomu kompilované jazyky
(Java/Kotlin, C\#, Go, Rust) nabízejí od základu vyšší průchodnost,
efektivní spouštění kódu a kontrolu datových typů, ale někdy díky
tomu zpomalují vývoj. Frameworky, jako jsou Spring nebo ASP.NET,
jsou robustní a díky své vysoké spolehlivosti vhodné pro korporátní
využití, %Ref: GPT/SOSurvey
zatímco mikroframworky (Flask, Express) jsou snadnější k pochopení,
ale vyžadují více tvorby vlastních vestavěných knihoven v aplikaci.
Volba technologie je silně závislá na velikosti projektu, jeho
požadavcích na výkon či škálovatelnost a znalostech vývojářského
týmu.

\section{Databáze}

% TODO: Doplnění informací od databázích

\section{API}

% TODO: Doplnění informací od API (REST/SOAP/GraphQL)

\section{Webové crawlery}

% TODO: Doplnění informací o webových crawlerech

\endinput
