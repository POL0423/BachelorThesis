\chapter{Úvod}
\label{ch:Introduction}

V této práci jsem se zaměřil na výzkum a tvorbu aplikace k porovnávání cen
pohonných hmot na různých čerpacích stanicích. Hlavním motivem pro tento
nápad byla složitost porovnávání cen pohonných hmot na internetu a také chuť
rozšířit své znalosti o moderní webové technologie, které se dnes využívají.
Cílem této práce je navrhnout webovou aplikaci, pomocí které by bylo možné
jednoduše s použitím vyhledávání vztažné lokace v zadaném okruhu porovnat
ceny PHM na různých čerpacích stanicích. Uživatel jednoduše zadá lokaci
a okruh porovnávání, a aplikace mu nabídne v okolí dostupné čerpací stanice,
které si uživatel může seřadit dle libosti (ve výchozím nastavení je řazení
od nejlevnější po nejdražší).

Základním stavebním kamenem pro takovou aplikaci je samozřejmě zdroj 
informací. Informace tak lze čerpat z webových stránek různých čerpacích
stanic. Způsob, jakým jsou informace z webových stránek dolovány, je určen
především rozsahem takových informací a jejich formátem. Data je nutné
určitým způsobem zpracovat a posléze v určitém námi stanoveném formátu
prezentovat. Pro ukládání kopie dat lze využít strukturovaných relačních
databází, například SQL databáze. V oblasti servírování dat je možné využít
preprocesory (například PHP), případně různé frameworky (React, apod.),
které data naformátují v námi požadovaném formátu.

Po dohodě se svým vedoucím práce bylo při rešerši a implementaci aplikace
využito nástrojů umělé inteligence, jako např. ChatGPT nebo GitHub Copilot.
Umělá inteligence je v práci řádně citována.

\endinput
