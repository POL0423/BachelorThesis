\chapter{Úvod}
\label{sec:Introduction}
V této práci jsem se zaměřil na výzkum a tvorbu aplikace k porovnávání cen pohonných hmot na různých čerpacích stanicích. Hlavním motivem pro tento nápad byla frustrace při složitosti porovnávání cen pohonných hmot. Cílem této práce je navrhnout webovou a mobilní aplikaci, pomocí které by bylo možné jednoduše s použitím mapy a lokalizačních metod (GPS, apod.) v zadaném okruhu porovnat ceny PHM na různých čerpacích stanicích. Uživatel jednoduše zadá lokaci a radius, a aplikace mu nabídne v okolí dostupné čerpací stanice, které si uživatel může seřadit dle libosti (ve výchozím nastavení je řazení od nejlevnější po nejdražší).

Základním stavebním kamenem pro takovou aplikaci je samozřejmě zdroj informací. Informace tak lze čerpat z webových stránek různých čerpacích stanic. Způsob, jakým jsou informace z webových stránek dolovány, je určen především rozsahem takových informací, a jejich formátem. Data je nutné určitým způsobem zpracovat a posléze v určitém námi stanoveném formátu prezentovat. Pro ukládání kopie dat lze využít strukturovaných relačních databází, například SQL databáze. V oblasti servírování dat je možné využít preprocesory (například PHP), případně různé frameworky (React, apod.), které data naformátují v námi požadovaném formátu.

V návrhu aplikace, která má potenciální komerční využití, je třeba postupovat následovně. Základem je rešerše dostupné technologie, metod získávání dat pro danou aplikaci, způsobů, jakými lze aplikaci provozovat, průzkum trhu na podobné řešení a následně realizace samotného projektu. Smyslem tohoto projektu je vytvořit poměrně jednoduchou všeobecnou aplikaci, jejímž účelem bude poskytnout lidem snadný způsob porovnání cen PHM z domova na pár kliknutí, případně během cesty vyhledat ceny PHM v okolí pomocí svého chytrého mobilního telefonu. Tato práce si klade za cíl položit základní kámen pro všeobecnou službu porovnávače cen PHM a vytvořit funkční prototyp takové aplikace.
\endinput
