% Nejprve uvedeme tridu dokumentu s volbami
\documentclass[czech,bachelor]{diploma}
% Dalsi doplnujici baliky maker
\usepackage[autostyle=true,czech=quotes]{csquotes} % korektni sazba uvozovek, podpora pro balik biblatex
\usepackage[backend=biber, style=iso-numeric, alldates=iso]{biblatex} % bibliografie
\usepackage{dcolumn} % sloupce tabulky s ciselnymi hodnotami
\usepackage{subfig} % makra pro "podobrazky" a "podtabulky"
\usepackage[sql,php,html,js,ts,css]{diplomalst}

% Zadame pozadovane vstupy pro generovani titulnich stran.
\ThesisAuthor{Marek Poláček}

\ThesisSupervisor{prof. Ing. Miroslav Vozňák, Ph.D.}

\CzechThesisTitle{Webová aplikace pro porovnání cen PHM}

\EnglishThesisTitle{Web Application for Fuel Prices Comparison}

\SubmissionYear{2025}

\ThesisAssignmentFileName{ThesisSpecification_POL0423_vsboee24030DA7.pdf}

% Pokud nechceme nikomu dekovat makro zapoznamkujeme.
\Acknowledgement{Rád bych na tomto místě poděkoval všem, kteří mi s prací pomohli, protože bez nich by tato práce nevznikla.}

\CzechAbstract{
    TODO: Doplnit zhodnocení práce
}

\CzechKeywords{aplikace; pohonné hmoty; srovnávač cen; \LaTeX; bakalářská práce}

\EnglishAbstract{
    TODO: Fill in assignment conclusion
}

\EnglishKeywords{application; fuel; price comparison; \LaTeX; bachelor thesis}

\AddAcronym{PHM}{Pohonné hmoty}
\AddAcronym{ČS}{Čerpací stanice}
\AddAcronym{GPS}{Global Positioning System}
\AddAcronym{HTML}{Hyper Text Markup Language}
\AddAcronym{CSS}{Cascade Style Sheet}
\AddAcronym{SQL}{Structured Query Language}
\AddAcronym{PHP}{Hyper Text Preprocessor}
\AddAcronym{VŠB-TUO}{Vysoká škola báňská – Technická univerzita Ostrava}
\AddAcronym{RHEL}{Red Hat Enterprise Linux}
\AddAcronym{DNS}{Domain Name System}
\AddAcronym{VM}{Virtual Machine}
\AddAcronym{IP}{Internet Protocol}
\AddAcronym{TCP}{Transmission Control Protocol}
\AddAcronym{OS}{Operační systém}
\AddAcronym{GUI}{Graphical User Interface}

\addbibresource{resources.bib}      % Jediné skutečné knižní prameny

% Novy druh tabulkoveho sloupce, ve kterem jsou cisla zarovnana podle desetinne carky
\newcolumntype{d}[1]{D{,}{,}{#1}}


% Zacatek dokumentu
\begin{document}

% Nechame vysazet titulni strany.
\MakeTitlePages

% Jsou v praci obrazky? Pokud ano vysazime jejich seznam a odstrankujeme.
% Pokud ne smazeme nasledujici dve makra.
\listoffigures
\clearpage

% Jsou v praci tabulky? Pokud ano vysazime jejich seznam a odstrankujeme.
% Pokud ne smazeme nasledujici dve makra.
\listoftables
\clearpage

% A nasleduje text zaverecne prace.
\chapter{Úvod}
\label{sec:Introduction}
V této práci jsem se zaměřil na výzkum a tvorbu aplikace k porovnávání cen pohonných hmot na různých čerpacích stanicích. Hlavním motivem pro tento nápad byla frustrace při složitosti porovnávání cen pohonných hmot. Cílem této práce je navrhnout webovou a mobilní aplikaci, pomocí které by bylo možné jednoduše s použitím mapy a lokalizačních metod (GPS, apod.) v zadaném okruhu porovnat ceny PHM na různých čerpacích stanicích. Uživatel jednoduše zadá lokaci a radius, a aplikace mu nabídne v okolí dostupné čerpací stanice, které si uživatel může seřadit dle libosti (ve výchozím nastavení je řazení od nejlevnější po nejdražší).

Základním stavebním kamenem pro takovou aplikaci je samozřejmě zdroj informací. Informace tak lze čerpat z webových stránek různých čerpacích stanic. Způsob, jakým jsou informace z webových stránek dolovány, je určen především rozsahem takových informací, a jejich formátem. Data je nutné určitým způsobem zpracovat a posléze v určitém námi stanoveném formátu prezentovat. Pro ukládání kopie dat lze využít strukturovaných relačních databází, například SQL databáze. V oblasti servírování dat je možné využít preprocesory (například PHP), případně různé frameworky (React, apod.), které data naformátují v námi požadovaném formátu.

V návrhu aplikace, která má potenciální komerční využití, je třeba postupovat následovně. Základem je rešerše dostupné technologie, metod získávání dat pro danou aplikaci, způsobů, jakými lze aplikaci provozovat, průzkum trhu na podobné řešení a následně realizace samotného projektu. Smyslem tohoto projektu je vytvořit poměrně jednoduchou všeobecnou aplikaci, jejímž účelem bude poskytnout lidem snadný způsob porovnání cen PHM z domova na pár kliknutí, případně během cesty vyhledat ceny PHM v okolí pomocí svého chytrého mobilního telefonu. Tato práce si klade za cíl položit základní kámen pro všeobecnou službu porovnávače cen PHM a vytvořit funkční prototyp takové aplikace.
\endinput

\chapter{Počáteční průzkum}
\section{Popis problému}
% Popis problému, který se snažím vyřešit

\section{Procházení webu a dolování dat}
% Problematika týkající se dolování dat z webů čerpacích stanic

\section{Mapy a ostatní API}
% Problematika týkající se zpracování a zobrazení dat mimo jiné také na mapě
% Výběr API pro danou problematiku

\endinput

\chapter{Motivace a přehled podobných projektů}
\label{ch:motivation-and-similar-applications}

V této kapitole se zabývám hlavní motivací a přehledem podobných projektů,
které obsahují již hotové řešení aplikace pro porovnávání cen pohonných hmot.

Hlavní motivace je složitost zjišťování cen pohonných hmot na internetu.
Řidiči si často musí udržovat přehled o cenách PHM ve svém bezprostředním
okolí. Velmi často si takový přehled udržují jednoduše aktivním využíváním
automobilu, např. při cestách do práce a zpět. Pokud se však rozhodne
vycestovat mimo své bydliště, udržet si přehled o cenách najednou přestává
být triviální. Řešením je internet, na kterém lze ceny PHM v některých
případech najít. Užitečným pomocníkem je poté srovnávací aplikace, která
rešerši pro zjištění a porovnání cen PHM udělá za vás.

Prostým položením dotazu do Google vyhledávače jsem byl schopen najít
informace k již existujícím podobným aplikacím. Jedná se zejména o články
porovnávající jednotlivé aplikace mezi sebou.

\section{iPumpuj a Pumpdroid}
\label{sec:ipumpuj-pumpdroid}

Tato aplikace poskytuje informace o cenách PHM na čerpacích stanicích.
Kromě toho poskytuje i detailní informace o dané ČS, včetně otevírací
doby a informací o případných pokutách od ČOI. Verze pro Android
se nazývá \emph{Pumpdroid}, zatímco uživatelé iOS aplikaci najdou
pod názvem \emph{iPumpuj}.
\cite{Vrablova2022, Sarikova2021}

\section{mBenzin.cz}
\label{sec:mbenzin}

Nejedná se o mobilní aplikaci, ale o web, na kterém lze vyhledávat
nejbližší a nejlevnější ČS s možností notifikací na motoristické
novinky. Web ukazuje také průměrné ceny PHM v Česku i v Evropě.
Magazín také poskytuje spojnicové grafy ukazující změnu cen
v průběhu jednotlivých dnů a měsíců.
\cite{Vrablova2022}

\section{Waze}
\label{sec:waze}

Tato aplikace je primárně určena k navigaci, ale uživatelům poskytuje
také přehled cen PHM. Ve vyhledávacím poli stačí vybrat symbol ČS.
Aplikace se tím transformuje na srovnávač cen PHM a automaticky zobrazí
ČS v okolí s preferovaným typem paliva společně s aktuální cenou.
\cite{Vrablova2022}

\section{Mapy.cz}
\label{sec:mapy}

Známý český vyhledávač Seznam.cz je autorem aplikace Mapy.cz, která
primárně slouží pro navigaci a orientaci. Tato aplikace je však také
schopna srovnat ceny PHM, podobně jako Waze.
\cite{Vrablova2023}

\section{Evropské aplikace}
\label{sec:european-apps}

Podobné aplikace jsou dostupné také pro ostatní evropské země. Jednou
takovou může být aplikace \emph{Mehr-tanken}, která je dostupná
pro uživatele iOS i Androidu. Tato aplikace získává data od uživatelů
a také od autorizované jednotky hlídající transparentnost trhu.
Aplikace podporuje i notifikace oznamující, kdy některá z oblíbených
ČS nebo ČS v okolí významně sníží ceny PHM. Aplikace také nabízí
srovnávač nabíjecích stanic pro elektromobily a vodíkových ČS.
\cite{r6fadX3YRnFIir68}

Aplikaci k porovnávání PHM vyvinul a zdarma publikoval také automobilový
svaz ADAC pod názvem \emph{ADAC Spritpreise}. Aplikace je vhodná pro rychlý
přehled cen PHM.
\cite{r6fadX3YRnFIir68}

Další evropskou aplikací je také např. \emph{PACE Drive}, která
porovnává řadu typů PHM a aplikace je bez reklam dostupná zdarma
pro uživatele iOS a Androidu. Porovnávat lze ceny PHM v Německu,
Španělsku, Francii, Itálii a Portugalsku.
\cite{r6fadX3YRnFIir68}

\section{Aplikace tankovacích karet}
\label{sec:tank-card-apps}

Držitelé tankovacích karet mohou také využít aplikace vydavatelů těchto
karet, které mají v sobě zabudovaný srovnávač cen PHM na ČS podporující
tyto tankovací karty. Jedním takovým vydavatelem je například CCS, který
nabízí aplikaci s názvem \emph{Tank Navigator}. Aplikaci lze zdarma
používat na iOS i Androidu, a nabízí vyhledávání ČS v okolí dle vzdálenosti,
zobrazení ČS na mapě, navigaci k vybrané ČS, filtraci ČS dle značky řetězce
a dostupných služeb, a další.
\cite{Khcm5FZT2rH5pABQ}

\endinput

\chapter{Technické detaily}
Pro provoz webové aplikace je zapotřebí backend server.
Pro tento účel jsem zvolil ICT centrum VŠB-TUO, kde jsem si nechal
vytvořit virtuální server.

\section{Operační systém a konfigurace DNS}
Jako operační systém serveru jsem zvolil Rocky Linux,
který mi byl doporučen administrátory školní sítě. Tato linuxová
distribuce je založena na RHEL. Součástí konfigurace VM je také
nastavení sítě. Neměnná IP adresa je získána z DHCP serveru,
avšak se jedná pouze o IPv4. IPv6 je potřeba nastavit ručně.
Pro server je předkonfigurovaný také DNS záznam pro IPv4 a IPv6,
který vypadá následovně:

\begin{verbatim}
Name                       Type   TTL   Section    IPAddress
----                       ----   ---   -------    ---------
pol0423-stu.vsb.cz         AAAA   7200  Answer     2001:718:1001:207::64
pol0423-stu.vsb.cz         A      7200  Answer     158.196.109.64
\end{verbatim}

DNS záznam AAAA je vazba na IPv6 adresu, zatímco DNS záznam A
je vazba na IPv4.

\section{Instalace OS a počáteční konfigurace serveru}
Server se nachází na VMware vSphere serveru, dostupný
na adrese vcs.vsb.cz. Prvním krokem byla instalace OS Rocky Linux,
která proběhla přes virtuální konzoli serveru. Instalační program
byl ve formě GUI, přes které jsem nastavil uživatelský účet správce
\texttt{marpolda}, nastavil jsem mu heslo a zapnul mu práva správce.
Účet správce \texttt{root} je vypnutý, lze ho tedy použít jen pomocí
příkazu \texttt{sudo}. Po instalaci OS byla dalším krokem instalace
nástrojů virtuálního stroje. Po neúspěšném pokusu v instalaci balíčku
VMware Tools jsem se rozhodl využít místo toho otevřený balíček
\texttt{open-vm-tools}. Pro práci s textovými soubory jsem si
také nainstaloval balíček \texttt{nano}, který poskytuje jednoduchý
textový editor přímo v příkazovém řádku.

\begin{figure}
    \centering
    \includegraphics[width=0.75\linewidth]{Figures/vmware_console.png}
    \caption{Konzole virtuálního serveru ve VMware Workstation Pro}
    \label{fig:vmware-workstation-pro}
\end{figure}

Dalším krokem je nastavení IPv6 sítě. Jelikož IPv6 adresa není
DHCP serverem přidělena, je potřeba adresu nastavit ručně. Pro to
využijeme výše zmíněný DNS záznam pro IPv6 vazbu. Tento proces
je také zapotřebí provést pomocí virtuální konzole.

\begin{verbatim}
# ip -6 addr add 2001:718:1001:207::64/64 dev ens33
# ip -6 route add default via 2001:718:1001:207::1 dev ens33
\end{verbatim}

Pro přístup na server pomocí SSH jsem si také importoval ručně
přes konzoli SSH klíče obou mých počítačů, které využívají kvantově
rezistentní algoritmus \texttt{ed25519}. Z důvodu bezpečnosti jsem
také provedl vypnutí přihlašování pomocí uživatelského hesla.
Tento krok jsem provedl vytvořením nového souboru
\texttt{/etc/ssh/sshd\_config.d/01-nopasswordlogin.conf}
s následujícím obsahem:

\begin{verbatim}
#################################################
# Disable password logins
#################################################

PasswordAuthentication no
\end{verbatim}

Soubory v adresáři \texttt{/etc/ssh/sshd\_config.d} jsou automaticky
importovány v souboru\\
\texttt{/etc/ssh/sshd\_config}, který obsahuje konfiguraci SSH Daemon
serveru.

Následně stačilo restartovat službu SSH Daemon:

\begin{verbatim}
# systemctl restart sshd.service
\end{verbatim}

Přístup na server pomocí SSH jsem následně otestoval:
\begin{verbatim}
PS C:\Users\marpo> ssh marpolda@pol0423-stu.vsb.cz
The authenticity of host 'pol0423-stu.vsb.cz (158.196.109.64)' can't be established.
ED25519 key fingerprint is SHA256:oHy0UKZisrWxLKQtp5Xpezo53FNXZudKJ6/WVHeScI4.
This host key is known by the following other names/addresses:
    ~/.ssh/known_hosts:18: 158.196.109.64
Are you sure you want to continue connecting (yes/no/[fingerprint])? yes
Warning: Permanently added 'pol0423-stu.vsb.cz' (ED25519) to the list of known hosts.
Enter passphrase for key 'C:\Users\marpo/.ssh/id_ed25519':
Last login: Sat Mar  1 14:01:55 2025 from 158.196.52.150
[marpolda@pol0423-stu ~]$
\end{verbatim}

Z mého druhého počítače jsem se také úspěšně přihlásil:
\begin{verbatim}
[marpolda@archlinuxx-laptop ~]$ ssh pol0423-stu.vsb.cz
Enter passphrase for key '/home/marpolda/.ssh/id_ed25519':
Last login: Sun Mar  2 19:21:04 2025 from 2001:718:1001:698:99e2:f1ff:4e67:7618
[marpolda@pol0423-stu ~]$
\end{verbatim}

\section{Kontejnerizace a instalace služeb}
Nyní, když máme nastavený základní přístup, můžeme přistoupit
k instalaci softwaru pro kontejnerizaci a samotných kontejnerů
potřebných služeb. Pro kontejnerizaci jsem si vybral Docker,
který jsem nainstaloval z příslušného balíčku následovně:

\begin{verbatim}
# dnf install docker
\end{verbatim}

% TODO: Po dokončení konfigurace IPv6 nainstalovat kontejnery
%       pro jednotlivé služby (web crawler, databáze, apod.)

\endinput

\chapter{Funkce vyvíjené aplikace}

\endinput

\chapter{Závěr}
\label{sec:conclusion}

Tvorba projektu byla poměrně náročná, nicméně velmi zajímavá.
Nejen, že~jsem využil něco z~toho, co již znám, ale~vyzkoušel jsem~si
i~něco nového, a~spoustu se~toho při~tvorbě této aplikace naučil.
Způsobů, jakým se~dají tvořit moderní webové stránky, je~nepřeberné
množství. Lze využít staré osvědčené metody, ale technologie a~trend
se~ubírají novým směrem a~posouvají~se kupředu. I~z~toho důvodu je~nutné
se~neustále vzdělávat a~držet krok s~dobou. Je to jedno z~mála, co~člověk
může udělat pro~to, aby~si~udržel přehled a~nezaostával za~ostatními.

Pokud bych měl zhodnotit splnění bodů, tak si~myslím, že~ačkoliv
se~při~rešerši a~vývoji vyskytly nečekané potíže, dokázal jsem~si s~nimi
poradit a~projekt dotáhnout zdárně do~úspěšného konce.

Musím zdůraznit jednu věc: Tento projekt je~pouze demonstrační.
Účelem bylo vyvinout webovou aplikaci, která pravidelně aktualizuje
ceny PHM, a~na~základě zadané geografické lokace je~porovnává, filtruje
a~nabízí uživateli. Pro~demonstraci byly nakonec vybrány 2~sítě čerpacích
stanic, ze~kterých jsou data získávána. Původním záměrem bylo sice použití
8~ČS (následně se~počet rozšířil na~9), ale při zkoumání dostupnosti dat
jsem zjistil, že~pouze 3~z~těchto čerpacích stanic nabízí data v~rozumném
formátu a~nakonec jsem zjistil během vývoje, že~jen 2~čerpací stanice
lze reálně použít. Problém by~nastal, kdyby dostupná byla jen 1~ČS.
Tím, že~dostupné jsou 2, je~porovnání cen možné. A~to~je základní předpoklad
k~funkční aplikaci porovnávače cen.

\begin{enumerate}
    \item \textbf{Motivace a~přehled podobných projektů.}
        
        Tato část zkoumá hnací impuls, který mě~vedl k~výběru tohoto
        tématu a~také podobné projekty, které jsou mému nápadu podobné.
        Popsal jsem tam webové stránky a~aplikace, které nabízí srovnání
        cen PHM na~ČS po~celé ČR nebo dokonce v~celé Evropě.

    \item \textbf{Technologie pro~návrh webových aplikací.}

        V~této části se~věnuji rozboru používaných technologií
        pro~návrh a~implementaci webových aplikací, kde~porovnávám
        jednotlivé knihovny a~frameworky, zhodnocuji jejich
        výhody a~nevýhody a~popisuji jejich typické využití.
        V~této části jsem využil umělé inteligence, konkrétně
        ChatGPT, k~rešerši. Tvrzení v~této kapitole jsou řádně
        citována, zdroj informací je~uveden jak z~ChatGPT, tak
        ze~zdrojů, které k~tvrzení umělá inteligence připojila.

    \item \textbf{Popis návrhu a~implementace.}

        Od~příprav, přes vývoj až~po~finální úpravy a~spuštění aplikace.
        Tato část popisuje postup tvorby jednotlivých dílčích součástí
        aplikace, jejich spojení, kontejnerizace a~následné finální
        spuštění na~VPS. Hlavním OS je~Rocky Linux, kde~je~nainstalován
        Nginx pro~provoz reverse proxy a~Docker, který obsahuje 3~kontejnery
        obsluhující jednotlivé části. Jeden kontejner je zaměřen
        na~databázi, další obsahuje crawler, který se~spouští periodicky
        každý den ve~3:00, a~poslední obsahuje webový frontend, který data
        z~databáze filtruje, řadí a~zobrazuje uživateli v~prohlížeči.
        Během vývoje aplikace jsem~si v~některých částech pomohl umělou
        inteligencí, konkrétně GitHub Copilotem, který mi~řešil obtížné
        chyby, jenž se~při~vývoji vyskytly, a~jejichž tradiční ladění
        by~znamenalo příliš velké prodloužení vývoje aplikace.

    \item \textbf{Zhodnocení dosažených výsledků.}

        Nebudu lhát, během vývoje došlo k~několika problémům, které
        finální podobu posunuly trochu jinak, než jsem původně očekával.
        Ve~všech případech si~však myslím, že~cíl bakalářské práce
        byl~splněn. Webová aplikace sice nevypadá tak, jak jsem si
        původně představoval, ale~základní funkcionalita byla dodržena.
        Crawler periodicky spouští procházení webu a aktualizuje databázi.
        Webový frontend nabízí uživateli možnost vyhledávání geografické
        lokace pomocí formuláře společně se~specifikací maximálního okruhu
        pro~porovnání, včetně možnosti změny řazení výsledků dle ceny
        (od~nejlevnějších nebo od~nejdražších) a~filtrování výsledků
        na~základě typu paliva či~jiného doplňkového produktu z~tankovacích
        stojanů či~jejich kvality. GUI webové aplikace je~sice minimální,
        ale~uživatelsky přívětivé a~přehledné.
\end{enumerate}

Ještě jednou pro~jistotu uvádím, že~jsem nad~rámec zadání bakalářské práce
po~dohodě s~vedoucím zařadil do~nabídky PHM k~porovnání také doplňkové
produkty dostupné z~tankovacích stojanů, které by~mohly řidiče zajímat.
Jedná~se zejména o~AdBlue (syntetickou močovinu) a~kapalinu do~ostřikovačů.
Tedy provozní kapaliny, které ze~své podstaty nejsou palivem, nicméně se~jedná
o~provozní kapaliny, které jsou nutné k~bezpečnému a~ekologickému provozu
automobilu.

Aplikaci by~samozřejmě šlo vylepšit a~případně přepracovat. Nabízí~se např.
využití umělé inteligence k~vyhledávání cen PHM a~dalších doplňkových
produktů a~služeb. Jiná možnost by~byla použití síly veřejnosti, která
by~mohla informace o~aktuálních cenách paliv a~doplňkových produktů a~služeb
doplňovat. Pokud jde o~prezentaci výsledků vyhledávání, tak více možností
pro~řazení, filtrování podle více typů a~kvalit produktů, apod. Geografické
určení uživatele by~mohlo případně probíhat i~geolokační službou prohlížeče.
Všechny tyto funkce jsou však již nad~rámec zadání této bakalářské práce
a~nejsou nezbytně nutné ke~splnění základních požadavků. Tyto možnosti však
lze nechat otevřené pro~případný výzkum v~rámci navazujícího studia
v~diplomové práci.

\endinput


% Seznam literatury
\printbibliography[title={Literatura}, heading=bibintoc]

% Prilohy
\appendix
%\chapter{Screenshoty}

\begin{sidewaysfigure}
    \centering
    \includegraphics[width=\textwidth]{Figures/vmware_console.png}
    \caption{Konzole virtuálního serveru ve VMware Workstation Pro}
    \label{fig:vmware-workstation-pro}
\end{sidewaysfigure}

\begin{sidewaysfigure}
    \centering
    \includegraphics[width=\textwidth]{Figures/globus_vyber.jpg}
    \caption{Základní informace o prodejně Globus, včetně cen PHM
        přidružené ČS}
    \label{fig:globus-cs}
\end{sidewaysfigure}

%\chapter{Instalace serveru a vývoj aplikace}

\begin{sidewaysfigure}
    \centering
    \includegraphics[width=\textwidth]{Figures/vmware_console.jpg}
    \caption{Konzole virtuálního serveru ve VMware Workstation Pro}
    \label{fig:vmware-workstation-pro}
\end{sidewaysfigure}

\newpage        %% BEGIN:LST

\lstinputlisting[label=src:docker-compose.yml,caption={Soubor \texttt{docker-compose.yml} k sestavení Docker kontejnerů}]{SourceCodes/docker-compose.yml}

\lstinputlisting[label=lst:npx-create-next,caption={Tvorba Next.js projektu},literate={√}{{\checkmark}}1]{SourceCodes/npx-create-next.log}

\lstinputlisting[label=src:pol0423-stu.vsb.cz.conf,caption={Soubor \texttt{/etc/nginx/conf.d/pol0423-stu.vsb.cz.conf}}]{SourceCodes/pol0423-stu.vsb.cz.conf}

\newpage

\lstinputlisting[label=src:snippets/ssl-params.conf,caption={Soubor \texttt{/etc/nginx/snippets/ssl-params.conf}}]{SourceCodes/snippets/ssl-params.conf}

\lstinputlisting[label=src:snippets/self-signed.conf,caption={Soubor \texttt{/etc/nginx/snippets/self-signed.conf}}]{SourceCodes/snippets/self-signed.conf}

\lstinputlisting[label=lst:net-modify,caption={Postup modifikace IPv6}]{SourceCodes/net-modify.log}

\newpage        %% END:LST

\begin{sidewaysfigure}
    \centering
    \includegraphics[width=\textwidth]{Figures/web-home-desktop.jpg}
    \caption{Hlavní stránka aplikace zobrazená v prohlížeči na PC}
    \label{fig:web-home-desktop}
\end{sidewaysfigure}

\begin{figure}
    \centering
    \includegraphics[width=0.5\linewidth]{Figures/web-home-mobile.jpg}
    \caption{Hlavní stránka aplikace v mobilním zobrazení}
    \label{fig:web-home-mobile}
\end{figure}

\begin{sidewaysfigure}
    \centering
    \includegraphics[width=\textwidth]{Figures/web-search-desktop.jpg}
    \caption{Stránka aplikace s výsledky vyhledávání na PC}
    \label{fig:web-search-desktop}
\end{sidewaysfigure}

\begin{figure}
    \centering
    \includegraphics[width=0.5\linewidth]{Figures/web-search-mobile.jpg}
    \caption{Stránka aplikace s výsledky vyhledávání v mobilním zobrazení}
    \label{fig:web-search-mobile}
\end{figure}

\begin{sidewaysfigure}
    \centering
    \includegraphics[width=\linewidth]{Figures/web-server.jpg}
    \caption{Stránka aplikace na serveru}
    \label{fig:web-server}
\end{sidewaysfigure}

\endinput


% Priloha vlozena primo do hlavniho LaTeX souboru. Ne vsechny prilohy je nutne mit ve zvlastnich souborech.
%\chapter{Dlouhý zdrojový kód}
%\lstinputlisting[label=src:CppExternal,caption={Dlouhý zdrojový kód v jazyce C++ načtený s externího souboru}]{SourceCodes/ArraySortingAlgorithms.cpp}

\end{document}
