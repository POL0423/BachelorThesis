\UseRawInputEncoding
% Nejprve uvedeme tridu dokumentu s volbami
\documentclass[czech,bachelor]{diploma}
% Dalsi doplnujici baliky maker
\usepackage[autostyle=true,czech=quotes]{csquotes} % korektni sazba uvozovek, podpora pro balik biblatex
\usepackage[backend=biber, style=iso-numeric, alldates=iso]{biblatex} % bibliografie
\usepackage{dcolumn} % sloupce tabulky s ciselnymi hodnotami
\usepackage{subfig} % makra pro "podobrazky" a "podtabulky"
\usepackage[sql,php,html,js,css]{diplomalst}
\usepackage{amssymb} % for \checkmark
\usepackage{amsmath}
\usepackage{makecell}

\renewcommand\theadfont{\normalsize}

% Zadame pozadovane vstupy pro generovani titulnich stran.
\ThesisAuthor{Marek Poláček}

\ThesisSupervisor{prof. Ing. Miroslav Vozňák, Ph.D.}

\CzechThesisTitle{Webová aplikace pro~porovnání cen PHM}

\EnglishThesisTitle{Web Application for~Fuel Prices Comparison}

\SubmissionYear{2025}

\ThesisAssignmentFileName{ThesisSpecification_POL0423_vsboee24030DA7.pdf}

% Pokud nechceme nikomu dekovat makro zapoznamkujeme.
\Acknowledgement{Rád bych na tomto místě poděkoval všem kamarádům a vedoucímu
práce, kteří mi s prací pomohli, protože bez nich by tato práce nevznikla.}

\CzechAbstract{
    V~této bakalářské práci se~věnuji návrhu a~implementaci webové aplikace
    pro~porovnávání cen pohonných hmot na~internetu. Práci začínám popisem
    motivace k~tvorbě tohoto projektu spolu s~přehledem podobných řešení.
    Následně pokračuji rozborem technologií používaných pro~návrh
    a~implementaci webových aplikací s~podrobným porovnáním jednotlivých
    knihoven a~frameworků, zhodnocením jejich výhod a~nevýhod a~určením jejich
    optimálních aplikací. Poté řeším technický návrh struktury této
    konkrétní webové aplikace, přípravu prostředí a~samotnou implementaci
    a~vývoj aplikace. Práci posléze zakončuji zhodnocením projektu.
}

\CzechKeywords{webová aplikace; pohonné hmoty; srovnávač cen;
    \LaTeX; bakalářská práce}

\EnglishAbstract{
    In~this Bachelor Thesis, I~cover the~design and~implementation
    of~the~web application for~fuel comparisons on~the~Internet. I~open
    this thesis by~exploring the~motivations behind the~creation of~such
    project, together with the~overview of~similar apps and~projects.
    Next I~cover the~research of~the~technologies used for designing
    and~implementation of~the~web applications with detailed comparison
    between each library and~frameworks, assessing their pros and~cons,
    and~determining their optimal usage. Thesis is~then used to~describe 
    technical structure design of~this particular web application,
    preparation of~the~environment, and~the~actual implementation. Finally,
    I~close~up this thesis by~the~project conclusion.
}

\EnglishKeywords{web application; fuel; price comparison;
    \LaTeX; bachelor thesis}

\AddAcronym{ACID}{Atomicity, Consistency, Isolation, Durability}
\AddAcronym{API}{Application Interface}
\AddAcronym{ASCII}{American Standard Code for Information Interchange}
\AddAcronym{CLR}{Common Language Runtime}
\AddAcronym{CMS}{Content Management System}
\AddAcronym{CRUD}{Create, Read, Update, Delete}
\AddAcronym{CSS}{Cascade Style Sheet}
\AddAcronym{ČOI}{Česká obchodní inspekce}
\AddAcronym{ČR}{Česká republika}
\AddAcronym{ČS}{Čerpací stanice}
\AddAcronym{DHCP}{Dynamic Host Configuration Protocol}
\AddAcronym{DNS}{Domain Name Service}
\AddAcronym{DOM}{Document Object Model}
\AddAcronym{GPS}{Global Positioning System}
\AddAcronym{GUI}{Graphical User Interface}
\AddAcronym{HTML}{Hyper Text Markup Language}
\AddAcronym{IP}{Internet Protocol}
\AddAcronym{JS}{JavaScript}
\AddAcronym{JSON}{JavaScript Object Notation}
\AddAcronym{JVM}{Java Virtual Machine}
\AddAcronym{Kč}{Koruna česká}
\AddAcronym{OCR}{Optical Character Recognition}
\AddAcronym{ORM}{Object Relational Mapping}
\AddAcronym{OS}{Operační systém}
\AddAcronym{OSM}{OpenStreetMap}
\AddAcronym{PC}{Personal Computer}
\AddAcronym{PHM}{Pohonné hmoty}
\AddAcronym{PHP}{Hyper Text Preprocessor}
\AddAcronym{PWA}{Progressive Web App}
\AddAcronym{RDBMS}{Relational Database Management System}
\AddAcronym{REST}{Representational State Transfer}
\AddAcronym{RHEL}{Red Hat Enterprise Linux}
\AddAcronym{SDL}{Specification and Description Language}
\AddAcronym{SEO}{Search Engine Optimization}
\AddAcronym{SOAP}{Simple Object Access Protocol}
\AddAcronym{SPA}{Single Page Application}
\AddAcronym{SQL}{Structured Query Language}
\AddAcronym{SSG}{Special Services Group}
\AddAcronym{SSR}{Solid State Relay}
\AddAcronym{SW}{Software}
\AddAcronym{TS}{TypeScript}
\AddAcronym{VM}{Virtual Machine}
\AddAcronym{VŠB-TUO}{Vysoká škola báňská – Technická univerzita Ostrava}
\AddAcronym{WSDL}{Web Services Description Language}
\AddAcronym{XML}{Extensible Markup Language}

\addbibresource{resources.bib}      % Jediné skutečné knižní prameny

% Novy druh tabulkoveho sloupce, ve kterem jsou cisla zarovnana podle desetinne carky
\newcolumntype{d}[1]{D{,}{,}{#1}}

% Zacatek dokumentu
\begin{document}

% Nechame vysazet titulni strany.
\MakeTitlePages

% Jsou v praci obrazky? Pokud ano vysazime jejich seznam a odstrankujeme.
% Pokud ne smazeme nasledujici dve makra.
\listoffigures
\clearpage

% Jsou v praci tabulky? Pokud ano vysazime jejich seznam a odstrankujeme.
% Pokud ne smazeme nasledujici dve makra.
\listoftables
\clearpage

% A nasleduje text zaverecne prace.
\chapter{Úvod}
\label{sec:Introduction}
V této práci jsem se zaměřil na výzkum a tvorbu aplikace k porovnávání cen pohonných hmot na různých čerpacích stanicích. Hlavním motivem pro tento nápad byla frustrace při složitosti porovnávání cen pohonných hmot. Cílem této práce je navrhnout webovou a mobilní aplikaci, pomocí které by bylo možné jednoduše s použitím mapy a lokalizačních metod (GPS, apod.) v zadaném okruhu porovnat ceny PHM na různých čerpacích stanicích. Uživatel jednoduše zadá lokaci a radius, a aplikace mu nabídne v okolí dostupné čerpací stanice, které si uživatel může seřadit dle libosti (ve výchozím nastavení je řazení od nejlevnější po nejdražší).

Základním stavebním kamenem pro takovou aplikaci je samozřejmě zdroj informací. Informace tak lze čerpat z webových stránek různých čerpacích stanic. Způsob, jakým jsou informace z webových stránek dolovány, je určen především rozsahem takových informací, a jejich formátem. Data je nutné určitým způsobem zpracovat a posléze v určitém námi stanoveném formátu prezentovat. Pro ukládání kopie dat lze využít strukturovaných relačních databází, například SQL databáze. V oblasti servírování dat je možné využít preprocesory (například PHP), případně různé frameworky (React, apod.), které data naformátují v námi požadovaném formátu.

V návrhu aplikace, která má potenciální komerční využití, je třeba postupovat následovně. Základem je rešerše dostupné technologie, metod získávání dat pro danou aplikaci, způsobů, jakými lze aplikaci provozovat, průzkum trhu na podobné řešení a následně realizace samotného projektu. Smyslem tohoto projektu je vytvořit poměrně jednoduchou všeobecnou aplikaci, jejímž účelem bude poskytnout lidem snadný způsob porovnání cen PHM z domova na pár kliknutí, případně během cesty vyhledat ceny PHM v okolí pomocí svého chytrého mobilního telefonu. Tato práce si klade za cíl položit základní kámen pro všeobecnou službu porovnávače cen PHM a vytvořit funkční prototyp takové aplikace.
\endinput

\chapter{Motivace a přehled podobných projektů}
\label{ch:motivation-and-similar-applications}

V této kapitole se zabývám hlavní motivací a přehledem podobných projektů,
které obsahují již hotové řešení aplikace pro porovnávání cen pohonných hmot.

Hlavní motivace je složitost zjišťování cen pohonných hmot na internetu.
Řidiči si často musí udržovat přehled o cenách PHM ve svém bezprostředním
okolí. Velmi často si takový přehled udržují jednoduše aktivním využíváním
automobilu, např. při cestách do práce a zpět. Pokud se však rozhodne
vycestovat mimo své bydliště, udržet si přehled o cenách najednou přestává
být triviální. Řešením je internet, na kterém lze ceny PHM v některých
případech najít. Užitečným pomocníkem je poté srovnávací aplikace, která
rešerši pro zjištění a porovnání cen PHM udělá za vás.

Prostým položením dotazu do Google vyhledávače jsem byl schopen najít
informace k již existujícím podobným aplikacím. Jedná se zejména o články
porovnávající jednotlivé aplikace mezi sebou.

\section{iPumpuj a Pumpdroid}
\label{sec:ipumpuj-pumpdroid}

Tato aplikace poskytuje informace o cenách PHM na čerpacích stanicích.
Kromě toho poskytuje i detailní informace o dané ČS, včetně otevírací
doby a informací o případných pokutách od ČOI. Verze pro Android
se nazývá \emph{Pumpdroid}, zatímco uživatelé iOS aplikaci najdou
pod názvem \emph{iPumpuj}.
\cite{Vrablova2022, Sarikova2021}

\section{mBenzin.cz}
\label{sec:mbenzin}

Nejedná se o mobilní aplikaci, ale o web, na kterém lze vyhledávat
nejbližší a nejlevnější ČS s možností notifikací na motoristické
novinky. Web ukazuje také průměrné ceny PHM v Česku i v Evropě.
Magazín také poskytuje spojnicové grafy ukazující změnu cen
v průběhu jednotlivých dnů a měsíců.
\cite{Vrablova2022}

\section{Waze}
\label{sec:waze}

Tato aplikace je primárně určena k navigaci, ale uživatelům poskytuje
také přehled cen PHM. Ve vyhledávacím poli stačí vybrat symbol ČS.
Aplikace se tím transformuje na srovnávač cen PHM a automaticky zobrazí
ČS v okolí s preferovaným typem paliva společně s aktuální cenou.
\cite{Vrablova2022}

\section{Mapy.cz}
\label{sec:mapy}

Známý český vyhledávač Seznam.cz je autorem aplikace Mapy.cz, která
primárně slouží pro navigaci a orientaci. Tato aplikace je však také
schopna srovnat ceny PHM, podobně jako Waze.
\cite{Vrablova2023}

\section{Evropské aplikace}
\label{sec:european-apps}

Podobné aplikace jsou dostupné také pro ostatní evropské země. Jednou
takovou může být aplikace \emph{Mehr-tanken}, která je dostupná
pro uživatele iOS i Androidu. Tato aplikace získává data od uživatelů
a také od autorizované jednotky hlídající transparentnost trhu.
Aplikace podporuje i notifikace oznamující, kdy některá z oblíbených
ČS nebo ČS v okolí významně sníží ceny PHM. Aplikace také nabízí
srovnávač nabíjecích stanic pro elektromobily a vodíkových ČS.
\cite{r6fadX3YRnFIir68}

Aplikaci k porovnávání PHM vyvinul a zdarma publikoval také automobilový
svaz ADAC pod názvem \emph{ADAC Spritpreise}. Aplikace je vhodná pro rychlý
přehled cen PHM.
\cite{r6fadX3YRnFIir68}

Další evropskou aplikací je také např. \emph{PACE Drive}, která
porovnává řadu typů PHM a aplikace je bez reklam dostupná zdarma
pro uživatele iOS a Androidu. Porovnávat lze ceny PHM v Německu,
Španělsku, Francii, Itálii a Portugalsku.
\cite{r6fadX3YRnFIir68}

\section{Aplikace tankovacích karet}
\label{sec:tank-card-apps}

Držitelé tankovacích karet mohou také využít aplikace vydavatelů těchto
karet, které mají v sobě zabudovaný srovnávač cen PHM na ČS podporující
tyto tankovací karty. Jedním takovým vydavatelem je například CCS, který
nabízí aplikaci s názvem \emph{Tank Navigator}. Aplikaci lze zdarma
používat na iOS i Androidu, a nabízí vyhledávání ČS v okolí dle vzdálenosti,
zobrazení ČS na mapě, navigaci k vybrané ČS, filtraci ČS dle značky řetězce
a dostupných služeb, a další.
\cite{Khcm5FZT2rH5pABQ}

\endinput

\chapter{Počáteční průzkum}
\section{Popis problému}
% Popis problému, který se snažím vyřešit

\section{Procházení webu a dolování dat}
% Problematika týkající se dolování dat z webů čerpacích stanic

\section{Mapy a ostatní API}
% Problematika týkající se zpracování a zobrazení dat mimo jiné také na mapě
% Výběr API pro danou problematiku

\endinput

\chapter{Příprava}
Pro provoz webové aplikace je zapotřebí backend server.
Pro tento účel jsem zvolil ICT centrum VŠB-TUO, kde jsem si nechal
vytvořit virtuální server.

\section{Operační systém a konfigurace DNS}
Jako operační systém serveru jsem zvolil Rocky Linux,
který mi byl doporučen administrátory školní sítě. Tato linuxová
distribuce je založena na RHEL. Součástí konfigurace VM je také
nastavení sítě. Neměnná IP adresa je získána z DHCP serveru,
avšak se jedná pouze o IPv4. IPv6 je potřeba nastavit ručně.
Pro server je předkonfigurovaný také DNS záznam pro IPv4 a IPv6,
který vypadá následovně:

\begin{verbatim}
Name                       Type   TTL   Section    IPAddress
----                       ----   ---   -------    ---------
pol0423-stu.vsb.cz         AAAA   7200  Answer     2001:718:1001:207::64
pol0423-stu.vsb.cz         A      7200  Answer     158.196.109.64
\end{verbatim}

DNS záznam AAAA je vazba na IPv6 adresu, zatímco DNS záznam A
je vazba na IPv4.

\section{Instalace OS a počáteční konfigurace serveru}
Server se nachází na VMware vSphere serveru, dostupný
na adrese vcs.vsb.cz. Prvním krokem byla instalace OS Rocky Linux,
která proběhla přes virtuální konzoli serveru. Instalační program
byl ve formě GUI, přes které jsem nastavil uživatelský účet správce
\texttt{marpolda}, nastavil jsem mu heslo a zapnul mu práva správce.
Účet správce \texttt{root} je vypnutý, lze ho tedy použít jen pomocí
příkazu \texttt{sudo}. Po instalaci OS byla dalším krokem instalace
nástrojů virtuálního stroje. Po neúspěšném pokusu v instalaci balíčku
VMware Tools jsem se rozhodl využít místo toho otevřený balíček
\texttt{open-vm-tools}. Pro práci s textovými soubory jsem si
také nainstaloval balíček \texttt{nano}, který poskytuje jednoduchý
textový editor přímo v příkazovém řádku.

Dalším krokem je nastavení IPv6 sítě. Jelikož IPv6 adresa není
DHCP serverem přidělena, je potřeba adresu nastavit ručně. Pro to
využijeme výše zmíněný DNS záznam pro IPv6 vazbu. Tento proces
je také zapotřebí provést pomocí virtuální konzole.

\begin{verbatim}
# ip a
1: lo: <LOOPBACK,UP,LOWER_UP> mtu 65536 qdisc noqueue state UNKNOWN
group default qlen 1000
    link/loopback 00:00:00:00:00:00 brd 00:00:00:00:00:00
    inet 127.0.0.1/8 scope host lo
       valid_lft forever preferred_lft forever
    inet6 ::1/128 scope host
       valid_lft forever preferred_lft forever
2: ens33: <BROADCAST,MULTICAST,UP,LOWER_UP> mtu 1500 qdisc fq_codel
state UP group default qlen 1000
    link/ether 00:50:56:ad:97:78 brd ff:ff:ff:ff:ff:ff
    altname enp2s1
    inet 158.196.109.64/24 brd 158.196.109.255 scope global dynamic
noprefixroute ens33
       valid_lft 55961sec preferred_lft 55961sec
    inet6 fe80::250:56ff:fead:9778/64 scope link noprefixroute
       valid_lft forever preferred_lft forever
# ip -6 addr add 2001:718:1001:207::64/64 dev ens33
# ip -6 route add default via 2001:718:1001:207::1 dev ens33
# ip a
1: lo: <LOOPBACK,UP,LOWER_UP> mtu 65536 qdisc noqueue state UNKNOWN
group default qlen 1000
    link/loopback 00:00:00:00:00:00 brd 00:00:00:00:00:00
    inet 127.0.0.1/8 scope host lo
       valid_lft forever preferred_lft forever
    inet6 ::1/128 scope host
       valid_lft forever preferred_lft forever
2: ens33: <BROADCAST,MULTICAST,UP,LOWER_UP> mtu 1500 qdisc fq_codel
state UP group default qlen 1000
    link/ether 00:50:56:ad:97:78 brd ff:ff:ff:ff:ff:ff
    altname enp2s1
    inet 158.196.109.64/24 brd 158.196.109.255 scope global dynamic
noprefixroute ens33
       valid_lft 55961sec preferred_lft 55961sec
    inet6 2001:718:1001:207::64/64 scope global
       valid_lft forever preferred_lft forever
    inet6 fe80::250:56ff:fead:9778/64 scope link noprefixroute
       valid_lft forever preferred_lft forever
\end{verbatim}

Otestujeme konektivitu IPv4 a IPv6:

\begin{verbatim}
PS C:\Users\marpo> ping -4 pol0423-stu.vsb.cz

Pinging pol0423-stu.vsb.cz [158.196.109.64] with 32 bytes of data:
Reply from 158.196.109.64: bytes=32 time<1ms TTL=61
Reply from 158.196.109.64: bytes=32 time<1ms TTL=61
Reply from 158.196.109.64: bytes=32 time<1ms TTL=61
Reply from 158.196.109.64: bytes=32 time<1ms TTL=61

Ping statistics for 158.196.109.64:
    Packets: Sent = 4, Received = 4, Lost = 0 (0% loss),
Approximate round trip times in milli-seconds:
    Minimum = 0ms, Maximum = 0ms, Average = 0ms
PS C:\Users\marpo> ping -6 pol0423-stu.vsb.cz

Pinging pol0423-stu.vsb.cz [2001:718:1001:207::64] with 32 bytes of data:
Reply from 2001:718:1001:207::64: time=2ms
Reply from 2001:718:1001:207::64: time=1ms
Reply from 2001:718:1001:207::64: time<1ms
Reply from 2001:718:1001:207::64: time=1ms

Ping statistics for 2001:718:1001:207::64:
    Packets: Sent = 4, Received = 4, Lost = 0 (0% loss),
Approximate round trip times in milli-seconds:
    Minimum = 0ms, Maximum = 2ms, Average = 1ms
\end{verbatim}

Pro přístup na server pomocí SSH jsem si také importoval ručně
přes konzoli SSH klíče obou mých počítačů, které využívají kvantově
rezistentní algoritmus \texttt{ed25519}. Z důvodu bezpečnosti jsem
také provedl vypnutí přihlašování pomocí uživatelského hesla.
Tento krok jsem provedl vytvořením nového souboru
\texttt{/etc/ssh/sshd\_config.d/01-nopasswordlogin.conf}
s následujícím obsahem:

\begin{verbatim}
#################################################
# Disable password logins
#################################################

PasswordAuthentication no
\end{verbatim}

Soubory v adresáři \texttt{/etc/ssh/sshd\_config.d} jsou automaticky
importovány v souboru\\
\texttt{/etc/ssh/sshd\_config}, který obsahuje konfiguraci SSH Daemon
serveru.

Následně stačilo restartovat službu SSH Daemon:

\begin{verbatim}
# systemctl restart sshd.service
\end{verbatim}

Přístup na server pomocí SSH jsem následně otestoval:
\begin{verbatim}
PS C:\Users\marpo> ssh marpolda@pol0423-stu.vsb.cz
The authenticity of host 'pol0423-stu.vsb.cz (158.196.109.64)' can't be established.
ED25519 key fingerprint is SHA256:oHy0UKZisrWxLKQtp5Xpezo53FNXZudKJ6/WVHeScI4.
This host key is known by the following other names/addresses:
    ~/.ssh/known_hosts:18: 158.196.109.64
Are you sure you want to continue connecting (yes/no/[fingerprint])? yes
Warning: Permanently added 'pol0423-stu.vsb.cz' (ED25519) to the list of known hosts.
Enter passphrase for key 'C:\Users\marpo/.ssh/id_ed25519':
Last login: Sat Mar  1 14:01:55 2025 from 158.196.52.150
[marpolda@pol0423-stu ~]$
\end{verbatim}

Z mého druhého počítače jsem se také úspěšně přihlásil:
\begin{verbatim}
[marpolda@archlinuxx-laptop ~]$ ssh pol0423-stu.vsb.cz
Enter passphrase for key '/home/marpolda/.ssh/id_ed25519':
Last login: Sun Mar  2 19:21:04 2025 from 2001:718:1001:698:99e2:f1ff:4e67:7618
[marpolda@pol0423-stu ~]$
\end{verbatim}

Pro další práci, zejména s frontendem aplikace, se nám
bude hodit self-signed SSL certifikát. V reálném použití
bychom použili místo toho certifikační autoritu (např.
\emph{Let's Encrypt}), ale pro naše účely bude stačit
self-signed certifikát. Prohlížeči se to sice líbit
nebude, ale to není předmětem této práce.

Pro náš certifikát si nejdříve připravíme vhodné prostředí.
Certifikát vytvoříme pomocí uživatelského účtu \texttt{root}.
Příkazem \texttt{sudo bash} spustíme Bash v režimu účtu
\texttt{root}, a přesuneme se do adresáře \texttt{/etc/ssl}.
V něm vytvoříme nový adresář s názvem \texttt{self-signed},
a toto bude náš pracovní adresář.

\begin{verbatim}
[root@pol0423-stu ssl]# mkdir self-signed
[root@pol0423-stu ssl]# cd self-signed/
[root@pol0423-stu self-signed]# pwd
/etc/ssl/self-signed
\end{verbatim}

V tomto adresáři vytvoříme nový pár certifikátu a jeho
soukromého klíče. Oba soubory budou patřit účtu \texttt{root}.
K certifikátu vepíšeme údaje pro identifikaci certifikátu
(země původu, provincie, lokalita, název organizace, název
organizační jednotky, doména serveru a email správce).

\begin{verbatim}
[root@pol0423-stu self-signed]# openssl req -newkey rsa:4096
-x509  -sha512  -days 365 -nodes -out certificate.pem -keyout
privatekey.pem
..+.+...........+.+..+..........+......+.....+......+......+
.......+++++++++++++++++++++++++++++++++++++++++++++*..+.+
....................+...+.......+............+.....+.......
+.....+....+..+++++++++++++++++++++++++++++++++++++++++++++*
...+...+...+...........................+.........+......+.......
+................................+....+.....+.......+...+..+...+
......+.+..+..........+...........................+...+.........
......+............+...+...+++++
...+.+.....+......+...+....+.....+...+.......+........+...+....+
......+..+..................+.+.....+.+.....+.........+...+.......
+......+........+.......+..+.+.........+.....+.+......+.........+
.....+......+.+..+.......+..............+.+..............+....+...+
.....+.......+..+.............+.....+..........+...+..+.......+...
+++++++++++++++++++++++++++++++++++++++++++++*......+.+...+..+.+..
+...+.........+.........+.......+..+.+..+.............+.........+
.....+...+...+......+...+.+......+.........+.....+.........+......
+.+++++++++++++++++++++++++++++++++++++++++++++*.....+......+....+
........+...+....+...+.....+.+.....+...................+..+......+
.........+....+..............+++++
-----
You are about to be asked to enter information that will be incorporated
into your certificate request.
What you are about to enter is what is called a Distinguished Name or a DN.
There are quite a few fields but you can leave some blank
For some fields there will be a default value,
If you enter '.', the field will be left blank.
-----
Country Name (2 letter code) [XX]:CZ
State or Province Name (full name) []:Moravskoslezský kraj
Locality Name (eg, city) [Default City]:Ostrava
Organization Name (eg, company) [Default Company Ltd]:Vysoká škola
báňská - Technická univerzita Ostrava
Organizational Unit Name (eg, section) []:Fakulta elektrotechniky
a informatiky
Common Name (eg, your name or your server's hostname) []:pol0423-stu.vsb.cz
Email Address []:marek.polacek.st@vsb.cz
[root@pol0423-stu self-signed]# ll
total 8
-rw-r--r--. 1 root root 2464 Mar  3 17:05 certificate.pem
-rw-------. 1 root root 3272 Mar  3 17:04 privatekey.pem
\end{verbatim}

Obsah certifikátu si ověříme:
\begin{verbatim}
[root@pol0423-stu self-signed]# openssl x509 -noout -in
certificate.pem -text
Certificate:
    Data:
        Version: 3 (0x2)
        Serial Number:
            43:26:47:1f:c5:16:c6:8a:ef:7b:0f:52:17:3c:
28:a8:55:75:9b:c1
        Signature Algorithm: sha512WithRSAEncryption
        Issuer: C=CZ, ST=Moravskoslezský kraj, L=Ostrava,
O=Vysoká Å¡kola báÅská - Technická univerzita Ostrava,
OU=Fakulta elektrotechniky a informatiky, CN=pol0423-stu.vsb.cz,
emailAddress=marek.polacek.st@vsb.cz
        Validity
            Not Before: Mar  3 16:05:17 2025 GMT
            Not After : Mar  3 16:05:17 2026 GMT
        Subject: C=CZ, ST=Moravskoslezský kraj, L=Ostrava,
O=Vysoká Å¡kola báÅská - Technická univerzita Ostrava,
OU=Fakulta elektrotechniky a informatiky, CN=pol0423-stu.vsb.cz,
emailAddress=marek.polacek.st@vsb.cz
        Subject Public Key Info:
            Public Key Algorithm: rsaEncryption
                Public-Key: (4096 bit)
                Modulus:
                    00:c9:62:38:15:9d:07:f5:3e:3d:b2:38:53:28:72:
                    a6:8a:c4:34:2e:ba:71:0b:f3:9a:e3:2c:08:43:7d:
                    34:84:a0:14:38:e5:89:2b:01:05:b1:66:38:f5:08:
                    4b:2e:ac:87:42:bd:09:a6:97:1c:2b:b6:28:ab:4e:
                    9b:3c:f2:47:e1:37:3e:14:a8:ee:dc:91:4b:9e:23:
                    31:66:c6:49:35:b8:0c:c3:31:ee:e0:b1:96:3a:ed:
                    d1:b9:24:31:a3:79:8e:23:83:59:9d:54:50:3e:24:
                    52:b1:45:1f:24:b5:e3:40:30:b5:6c:92:fb:db:fa:
                    c8:9d:4d:e3:7a:c6:b1:76:9d:3f:d2:44:61:50:e0:
                    4f:2a:f5:b9:ee:5b:3c:56:02:24:39:91:d8:03:bf:
                    07:e2:b6:bd:bd:25:04:f5:8e:f4:c4:f4:dc:ef:36:
                    9f:9a:60:2f:7a:86:fb:03:3a:40:85:13:17:6b:1b:
                    2a:16:f2:8a:47:c5:78:2c:0a:eb:85:06:1f:7c:8b:
                    e5:c6:6a:fa:8a:3a:8e:47:a3:bb:25:74:e2:f5:13:
                    47:a1:6a:b9:9c:90:70:cc:71:b4:e9:1b:28:6e:f3:
                    2a:0a:31:f4:2d:b6:a6:35:5a:7b:65:28:1a:bf:0b:
                    4d:ca:4a:3b:4b:4f:71:e5:fc:a3:08:f0:41:52:94:
                    6e:cd:73:89:09:60:bc:29:2d:5e:a0:72:3a:b6:22:
                    c0:98:5e:df:8e:83:c8:a0:13:23:7d:a5:55:ec:bd:
                    91:2a:cf:68:52:1e:ec:ee:1a:db:ae:a4:da:19:e7:
                    a1:dd:91:5e:66:70:a8:f2:35:35:6c:24:c0:03:10:
                    48:a6:50:39:59:8e:fd:ee:51:40:ba:a3:b5:fb:44:
                    3a:0c:c1:be:87:8d:9b:82:ec:5a:c0:6d:cf:27:73:
                    32:9a:bb:cc:45:86:b6:46:6b:e2:21:e9:dc:4c:17:
                    06:3b:98:07:49:cc:71:b2:b6:c8:57:d6:dd:d9:7a:
                    95:b3:77:30:5c:d4:dc:a5:b7:07:e1:6e:7f:cd:23:
                    2a:61:a7:9d:f8:b7:cc:e5:3f:7e:75:e6:68:b3:18:
                    fe:29:f3:99:2e:11:f2:13:c0:d9:0b:0e:d8:eb:2c:
                    f4:bf:ee:98:0c:0a:56:c5:cd:12:6d:06:3d:17:99:
                    a6:a2:4b:16:6d:5f:51:90:b3:87:04:10:57:f8:54:
                    b6:dd:8d:9c:ed:e9:26:88:28:59:18:d6:94:b4:88:
                    8c:4d:3a:0f:a8:48:fb:44:1e:66:6e:b7:45:19:74:
                    69:d4:cb:ec:9e:67:79:e2:cf:3e:d2:d4:2c:15:e1:
                    b1:64:d2:76:a4:e1:e8:8f:df:90:2b:71:09:96:98:
                    46:78:d1
                Exponent: 65537 (0x10001)
        X509v3 extensions:
            X509v3 Subject Key Identifier:
                F2:41:E9:50:AC:72:59:5B:91:4A:C3:89:B7:30:5C:9E:
4E:2B:56:2A
            X509v3 Authority Key Identifier:
                F2:41:E9:50:AC:72:59:5B:91:4A:C3:89:B7:30:5C:9E:
4E:2B:56:2A
            X509v3 Basic Constraints: critical
                CA:TRUE
    Signature Algorithm: sha512WithRSAEncryption
    Signature Value:
        a9:ca:16:c1:fb:3d:3d:7c:d2:d0:d2:70:07:74:c2:3e:fb:6f:
        59:20:0e:b4:e2:9b:b6:41:02:94:ef:6d:f6:1d:f1:b6:83:8a:
        7e:aa:17:37:1e:30:f6:9d:26:f3:c7:6d:b2:c4:59:27:6a:b1:
        2f:e2:c5:c3:1d:fd:f2:81:fa:2e:27:51:5b:e3:be:ee:3a:40:
        fc:05:50:63:01:ab:b3:80:c1:a1:70:22:e3:8d:6f:2f:e4:e1:
        2d:89:15:c0:68:8b:9c:1c:5f:d6:6b:37:fb:06:53:61:fb:2d:
        94:af:36:e4:05:cc:8a:a3:92:d7:00:c0:26:19:6b:92:a8:ea:
        2a:15:44:87:dd:e5:73:3d:ed:dd:94:b0:47:a1:cd:b4:12:d1:
        5f:89:d7:73:48:5a:37:e3:e4:96:49:af:87:1c:b1:2a:bd:3f:
        6e:a5:33:fd:bc:84:55:81:c3:9d:54:c9:1b:d8:d0:c7:b9:60:
        27:a3:4d:e8:86:a4:2e:b2:07:8a:4d:d4:c2:f7:fb:19:16:c7:
        16:09:90:d5:e0:87:82:30:98:54:05:e3:4b:67:3d:e3:47:dd:
        1a:75:eb:95:df:80:0c:1b:c7:88:6e:1b:1f:10:49:bf:7f:06:
        2d:f1:1d:8a:59:54:8b:c0:9e:a8:a5:26:eb:4d:dd:b6:a9:af:
        63:b8:0f:19:cf:f1:54:62:29:18:23:9c:40:6d:57:bf:dc:52:
        4d:5b:fc:1b:90:75:17:e8:5b:a6:e4:3d:48:7a:59:29:d1:54:
        c4:18:91:db:c5:2d:dc:e4:4f:e5:ef:05:e4:ab:19:0e:6b:d6:
        7e:f5:24:91:3a:5c:9a:28:be:e8:20:dc:17:3b:06:57:99:48:
        0f:4e:0d:d6:0b:dc:7b:17:19:50:de:ca:81:fe:95:c1:f6:42:
        bc:b9:f4:bb:54:0e:01:e0:5f:38:7e:ec:e8:eb:33:24:de:cd:
        b6:71:cd:70:2e:d7:e6:43:69:b0:20:a6:dc:fd:43:dd:b4:97:
        c0:6f:0f:02:e6:ea:95:c8:f3:6e:8b:7d:d7:46:a3:cf:56:38:
        bd:c9:89:85:8c:df:73:6a:2f:6f:08:d2:e7:ac:86:c9:a5:69:
        41:79:2c:3c:1b:33:15:ff:d8:75:b5:15:5e:d4:4c:d3:a4:85:
        28:e2:b1:b3:03:e4:34:9d:37:22:68:8b:f3:f4:09:4c:01:4b:
        47:e2:5e:63:4e:19:b5:c3:d5:9f:b1:6e:7a:46:56:31:9e:77:
        ba:cc:b5:d1:df:a3:e6:82:06:7c:14:5a:2d:36:ee:ef:62:3e:
        f6:bc:8c:58:6c:a4:47:ca:06:8e:74:3e:1a:66:8a:67:b4:cd:
        3b:46:0d:f2:d5:f1:59:cf
\end{verbatim}

\section{Kontejnerizace a instalace služeb}
Nyní, když máme nastavený základní přístup, můžeme přistoupit
k instalaci softwaru pro kontejnerizaci a samotných kontejnerů
potřebných služeb. Pro kontejnerizaci jsem si vybral Docker,
který jsem nainstaloval z příslušného balíčku následovně:

\begin{verbatim}
# dnf install docker
\end{verbatim}

Dalším krokem je příprava samotných součástí aplikace. K tomu
jsem si vytvořil vývojové prostředí Gitu, ve kterém bude
probíhat vývoj aplikace a příprava pro její nasazení na server.
Toto prostředí se také nachází na GitHubu, kde je vidět
aktuální podoba této aplikace. Aplikaci jsem se rozhodl pojmenovat
\emph{PetrolScan}. Všechny součásti aplikace obsahují soubor
\texttt{Dockerfile}, který vytváří Docker obrázek dané součásti.
Všechny tyto součásti jsou pospojovány souborem \texttt{docker-compose.yml},
který součásti propojuje a vytváří tak ucelenou službu.

Vyvíjené součásti aplikace jsou:

\begin{itemize}
    \item Webová aplikace:
        \texttt{\url{https://github.com/POL0423/petrolscan-web-app}}
    \item Crawler:
        \texttt{\url{https://github.com/POL0423/petrolscan-crawler}}
    \item Docker Compose:
        \texttt{\url{https://github.com/POL0423/petrolscan-docker-compose}}
\end{itemize}

\subsection{Výběr softwaru pro crawler}

Jako první jsem se rozhodl najít vhodné self-hosted webové crawlery
a scrapery. Na dotaz k vyhledávání mi ChatGPT našlo následující možnosti:

\begin{itemize}
    \item Crawlee
    \item Scrapy
    \item EasySpider
\end{itemize}

Rozhodl jsem se pro Crawlee.

\subsubsection{Zkouška Crawlee}

Tento crawler funguje v Node.js, a nabízí jednoduchou instalaci pomocí
jednoduchého příkazu. Crawler simuluje procházení webu člověkem pomocí
bezhlavičkových webových prohlížečů, které jsou schopné také interpretovat
Javascriptový kód.

Zkušební test ukázkového crawleru proběhl v pořádku.
První pokus o spuštění neuspěl, neboť chyběly Playwright bezhlavičkové
prohlížeče. Po instalaci těchto prohlížečů se již crawler spustil.

\begin{verbatim}
$ npx crawlee create test-crawler
Need to install the following packages:
crawlee@3.13.0
Ok to proceed? (y) y
npm WARN deprecated lodash.isequal@4.5.0: This package is deprecated.
Use require('node:util').isDeepStrictEqual instead.
? Please select the template for your new Crawlee project Getting started
example [TypeScript]
npm WARN deprecated lodash.isequal@4.5.0: This package is deprecated. Use
require('node:util').isDeepStrictEqual instead.

added 309 packages, and audited 310 packages in 12s

81 packages are looking for funding
  run `npm fund` for details

found 0 vulnerabilities
Project test-crawler was created. To run it, run "cd test-crawler" and
"npm start".
$ cd test-crawler/
$ npm start

> test-crawler@0.0.1 start
> npm run start:dev


> test-crawler@0.0.1 start:dev
> tsx src/main.ts

INFO  PlaywrightCrawler: Starting the crawler.
INFO  PlaywrightCrawler: Final request statistics: {"requestsFinished":0,
"requestsFailed":0,"retryHistogram":[],"requestAvgFailedDurationMillis":
null,"requestAvgFinishedDuration
Millis":null,"requestsFinishedPerMinute":0,"requestsFailedPerMinute":0,
"requestTotalDurationMillis":0,"requestsTotal":0,"crawlerRuntimeMillis":685}
INFO  PlaywrightCrawler: Finished! Total 0 requests: 0 succeeded, 0 failed.
{"terminal":true}
node:internal/process/esm_loader:34
      internalBinding('errors').triggerUncaughtException(
                                ^

Failed to launch browser. Please check the following:
- Try installing the required dependencies by running `npx playwright
install --with-deps` (https://playwright.dev/docs/browsers).
[...]
$ npx playwright install
Downloading Chromium 134.0.6998.35 (playwright build v1161) from ...
141.8 MiB [====================] 100% 0.0s
Chromium 134.0.6998.35 (playwright build v1161) downloaded to ...
Downloading Chromium Headless Shell 134.0.6998.35 (playwright build v1161)
from ...
87.8 MiB [====================] 100% 0.0s
Chromium Headless Shell 134.0.6998.35 (playwright build v1161) downloaded
to ...
Downloading Firefox 135.0 (playwright build v1475) from ...
91.5 MiB [====================] 100% 0.0s
Firefox 135.0 (playwright build v1475) downloaded to ...
Downloading Webkit 18.4 (playwright build v2140) from ...
52.8 MiB [====================] 100% 0.0s
Webkit 18.4 (playwright build v2140) downloaded to ...
Downloading FFMPEG playwright build v1011 from ...
1.3 MiB [====================] 100% 0.0s
FFMPEG playwright build v1011 downloaded to ...
Downloading Winldd playwright build v1007 from ...
0.1 MiB [====================] 100% 0.0s
Winldd playwright build v1007 downloaded to ...
$ npm start

> test-crawler@0.0.1 start
> npm run start:dev


> test-crawler@0.0.1 start:dev
> tsx src/main.ts

INFO  PlaywrightCrawler: Starting the crawler.
INFO  PlaywrightCrawler: Title of https://crawlee.dev/ is 'Crawlee · Build
reliable crawlers. Fast.'
INFO  PlaywrightCrawler: Title of https://crawlee.dev/docs/examples is
'Examples | Crawlee · Build reliable crawlers. Fast.'
INFO  PlaywrightCrawler: Title of https://crawlee.dev/docs/quick-start is
'Quick Start | Crawlee · Build reliable crawlers. Fast.'
INFO  PlaywrightCrawler: Title of https://crawlee.dev/api/core is
'@crawlee/core | API | Crawlee · Build reliable crawlers. Fast.'
INFO  PlaywrightCrawler: Title of https://crawlee.dev/api/core/changelog
is 'Changelog | API | Crawlee · Build reliable crawlers. Fast.'
INFO  PlaywrightCrawler: Title of https://crawlee.dev/blog is 'Crawlee
Blog - learn how to build better scrapers | Crawlee · Build reliable
crawlers. Fast.'
INFO  PlaywrightCrawler: Title of https://crawlee.dev/python is 'Crawlee
for Python · Fast, reliable Python web crawlers.'
INFO  PlaywrightCrawler: Title of https://crawlee.dev/docs/next/quick-
start is 'Quick Start | Crawlee · Build reliable crawlers. Fast.'
INFO  PlaywrightCrawler: Title of https://crawlee.dev/docs/3.12/quick-
start is 'Quick Start | Crawlee · Build reliable crawlers. Fast.'
INFO  PlaywrightCrawler: Title of https://crawlee.dev/docs/3.10/quick-
start is 'Quick Start | Crawlee · Build reliable crawlers. Fast.'
INFO  PlaywrightCrawler: Title of https://crawlee.dev/docs/3.11/quick-
start is 'Quick Start | Crawlee · Build reliable crawlers. Fast.'
INFO  PlaywrightCrawler: Title of https://crawlee.dev/docs/3.9/quick-
start is 'Quick Start | Crawlee · Build reliable crawlers. Fast.'
INFO  PlaywrightCrawler: Title of https://crawlee.dev/docs/3.8/quick-
start is 'Quick Start | Crawlee · Build reliable crawlers. Fast.'
INFO  PlaywrightCrawler: Title of https://crawlee.dev/docs/3.7/quick-
start is 'Quick Start | Crawlee · Build reliable crawlers. Fast.'
INFO  PlaywrightCrawler: Title of https://crawlee.dev/docs/3.3/quick-
start is 'Crawlee · Build reliable crawlers. Fast.'
INFO  PlaywrightCrawler: Title of https://crawlee.dev/docs/3.4/quick-
start is 'Quick Start | Crawlee · Build reliable crawlers. Fast.'
INFO  PlaywrightCrawler: Title of https://crawlee.dev/docs/3.6/quick-
start is 'Quick Start | Crawlee · Build reliable crawlers. Fast.'
INFO  PlaywrightCrawler: Title of https://crawlee.dev/docs/3.5/quick-
start is 'Quick Start | Crawlee · Build reliable crawlers. Fast.'
INFO  PlaywrightCrawler: Title of https://crawlee.dev/docs/3.2/quick-
start is 'Quick Start | Crawlee · Build reliable crawlers. Fast.'
INFO  PlaywrightCrawler: Title of https://crawlee.dev/docs/3.0/quick-
start is 'Quick Start | Crawlee · Build reliable crawlers. Fast.'
INFO  PlaywrightCrawler: Title of https://crawlee.dev/docs/3.1/quick-
start is 'Quick Start | Crawlee · Build reliable crawlers. Fast.'
INFO  PlaywrightCrawler: Title of https://crawlee.dev/docs/introduction
is 'Introduction | Crawlee · Build reliable crawlers. Fast.'
INFO  PlaywrightCrawler: Title of https://crawlee.dev/docs/guides/
javascript-rendering is 'JavaScript rendering | Crawlee · Build reliable
crawlers. Fast.'
INFO  PlaywrightCrawler: Crawler reached the maxRequestsPerCrawl limit
of 20 requests and will shut down soon. Requests that are in progress
will be allowed to finish.
INFO  PlaywrightCrawler: Earlier, the crawler reached the
maxRequestsPerCrawl limit of 20 requests and all requests that were in
progress at that time have now finished. In total, the crawler processed
23 requests and will shut down.
INFO  PlaywrightCrawler: Final request statistics: {"requestsFinished":23,
"requestsFailed":0,"retryHistogram":[23],"requestAvgFailedDurationMillis":
null,"requestAvgFinishedDurationMillis":3071,"requestsFinishedPerMinute":
45,"requestsFailedPerMinute":0,"requestTotalDurationMillis":70640,
"requestsTotal":23,"crawlerRuntimeMillis":30383}
INFO  PlaywrightCrawler: Finished! Total 23 requests: 23 succeeded,
0 failed. {"terminal":true}
\end{verbatim}

Velmi příjemným zjištěním bylo, že builder již automaticky vygeneruje
soubor \texttt{Dockerfile}, pro crawler tak lze rovnou vytvořit
Docker image, který se jen posléze nahraje na Docker Hub a lze s ním
dále pracovat.

\subsubsection{Tvorba crawleru}

Nejprve je nutné prozkoumat webové stránky různých čerpacích stanic.
Uvažujme tak tyto čerpací stanice:

\begin{itemize}
    \item \textbf{Globus}
    \item Orlen
    \item Shell
    \item EuroOil
    \item \textbf{ONO}
    \item MOL
    \item OMV
    \item Prim
\end{itemize}

Pro všechny tyto ČS tak je nutné prozkoumat a analyzovat jejich webové
stránky, abychom mohli z daných webových stránek vytáhnout příslušná data.
Začíná mravenčí práce, a sice prozkoumávání a analýza skladby jednotlivých
webových stránek ČS, zápis struktury každého webu a sestavení jednotlivých
šablon pro náš crawler tak, aby byl crawler schopen stránky procházet
a získávat daný obsah. Crawler tato data bude tahat a zaznamenávat
do databáze strukturovaně podle typu jednotlivých PHM, jejich značce
a umístění ČS.

\subsubsection{Globus}

Tato ČS patří velkoobchodnímu řetězci (hypermarketu) stejného jména
a nachází se výhradně vždy v blízkosti prodejny. Globus lze v době
psaní této bakalářské práce najít jen v 16 různých městech a městských
částí v ČR:

\begin{itemize}
    \item Brno
    \item České Budějovice
    \item Chomutov
    \item Havířov
    \item Karlovy Vary ‒ Jenišov
    \item Liberec
    \item Olomouc
    \item Opava
    \item Ostrava
    \item Pardubice
    \item Plzeň ‒ Chotíkov
    \item Praha
    \begin{itemize}
        \item Čakovice
        \item Černý Most
        \item Štěrboholy
        \item Zličín
    \end{itemize}
    \item Trmice
\end{itemize}

Postup pro získání informací o cenách PHM z ČS je následující.

\begin{enumerate}
    \item Na webové stránce www.globus.cz klikneme na tlačítko s ikonou
        špendlíku. Pokud se nás web zeptá na výběr prodejny, zvolíme
        dle požadavků (například Ostravu). Pokud je zvolená prodejna
        nesprávná, lze výběr změnit kliknutím na tlačítko „Změnit“
        v prvním sloupci pop-up okénka.
    \item Požadované informace z této ČS jsou ve třetím sloupci (viz
        screenshot na obrázku \ref{fig:globus-cs} na stránce
        \pageref{fig:globus-cs})
\end{enumerate}

Tento postup je potřeba automatizovat. K tomu nám právě poslouží již
zmíněný crawler. Každá ČS má jinou strukturu webu, a proto je potřeba
vytvořit pro každý web jeho vlastní crawler. K tomu lze využít threading
v Node.js. Například pro ČS Globus vypadá technický automatizovaný postup
následovně:

\begin{enumerate}
    \item Crawler načte webovou stránku www.globus.cz
    \item Je-li zobrazeno upozornění na cookies, odklikne se souhlas.
        Toto tlačítko je popsáno následujícím CSS selektorem:

        \texttt{
            div\#CybotCookiebotDialog >
            div.CybotCookiebotDialogContentWrapper >
            div\#CybotCookiebotDialogFooter > 
            div.CybotCookiebotScrollArea >
            div\#CybotCookiebotDialogBodyButtons >
            div\#CybotCookiebotDialogBodyButtonsWrapper >
            button\#CybotCookiebotDialogBodyLevelButtonLevelOptinAllowAll
        }

    \item Klikneme na tlačítko \texttt{Vybrat pobočku}. To je popsáno
        následujícím CSS selektorem:

        \texttt{
            div\#\_\_nuxt > div\#layout-default > header\#header >
            div.container > div.items-center > div.flex-auto >
            div.flex > button.btn-lg
        }

    \item Jednotlivé pobočky zkopírujeme do seznamu, ze kterého poté
        čerpáme. Prvky v tomto seznamu jsou popsány následujícími CSS
        selektory:

        \texttt{
            /* Seznam */\\
            \#input\_9 > ul
        }

        \texttt{
            /* Položka [i] = 1 .. n; // n = 16 */\\
            \#input\_9 > ul > li:nth-child(i)
        }

    \item Klikneme postupně na všechny prodejny v seznamu. Prvek
        formuláře je popsán následujícím CSS selektorem:

        \texttt{
            div\#teleport-target > div.items-end > div.items-end >
            div.flex-col > div.overscroll-contain > div.max-h-full >
            div.formkit-outer > fieldset\#input_9 > ul.formkit-options >
            li.grid > label.formkit-wrapper > span.formkit-inner >
            input.formkit-input
        }

    \item Počkáme, než se načtou informace (asi 1 s).
    \item Zkopírujeme informace o stanici a palivech (název, cena).
        Tyto informace jsou popsány následujícími CSS selektory:

        \texttt{
            /* Název stanice */\\
            \#teleport-target > div > div > div >
            div.shrink-1.grow-0.max-h-full.min-h-0.overflow-y-auto.overflow-auto.overscroll-contain >
            div > section > div.flex.items-center.gap-x-4 > h2 >
            span.text-sm.lg\textbackslash :text-base
        }

        \texttt{
            /* Název paliva */\\
            div#teleport-target > div.items-end > div.items-end >
            div.flex-col > div.overscroll-contain > div.max-h-full >
            div.lg\textbackslash :pl-6 > section.space-y-2 > table.w-full >
            tbody > tr > th.text-left
        }

        \texttt{
            /* Cena paliva */\\
            div\#teleport-target > div.items-end > div.items-end >
            div.flex-col > div.overscroll-contain > div.max-h-full >
            div.lg\textbackslash :pl-6 > section.space-y-2 > table.w-full >
            tbody > tr > td.text-right
        }

    \item Klikneme na tlačítko \texttt{Zmenit} pro výběr další pobočky.
        Toto tlačítko je popsáno následujícím CSS selektorem:

        \texttt{
            \#teleport-target > div > div > div >
            div.shrink-1.grow-0.max-h-full.min-h-0.overflow-y-auto.overflow-auto.overscroll-contain >
            div > section > div.flex.items-center.gap-x-4 > button
        }
        
    \item Opakujeme od bodu 5, dokud neprojdeme všechny prodejny.
\end{enumerate}

\subsubsection{ONO}

% TODO: Umístění a struktura webu

\subsubsection{Vyřazené čerpací stanice}

Během analýzy zmíněných webů ČS jsem zjistil, že právě pouze Globus
a ONO poskytují ceny PHM. Následující weby řetězců ČS jsem musel
z projektu vyřadit, neboť nesplňují požadavky k dolování dat.

\begin{itemize}
    \item \textbf{OMV:} Tato síť ČS poskytuje ceny PHM ve formátu,
        který vyžaduje použití OCR čtečky, jejíž implementace
        by byla velmi náročná, a časově by nebylo reálné čtečku
        do projektu zakomponovat.
    \item \textbf{Orlen:} Webové stránky této ČS využívají REST API,
        což je vysoké plus. Ale to samo o sobě nestačí. Pokud ani
        v API, ani na webové stránce nejsou k dispozici ceny PHM,
        jen slevy na tankovací kartu či aplikaci této sítě, nelze
        tato data v porovnání cen PHM použít.
    \item \textbf{Shell, EuroOil, MOL a Prim:} Tyto ČS informace
        o cenách PHM na svých webových stránkách vůbec neposkytují,
        čímž nelze data z těchto webových stránek použít.
\end{itemize}

\subsubsection{Threading a strukturizace crawleru}

% TODO: Postup tvorby crawleru

\subsection{Databáze}

% TODO: Databázové systémy

\subsection{Frontend}

% TODO: Tvorba frontendu

\endinput

\chapter{Vývoj aplikace}

Samotná implementace celé aplikace je rozdělena na několik částí.

\begin{itemize}
    \item \textbf{Crawler:} Sbírá data a zapisuje je do relační databáze.
        Využit je software \emph{Crawlee} s knihovnou \emph{Playwright Chromium}.
        K geolokaci jednotlivých stanic je použit API mapového poskytovatele OSM.
    \item \textbf{Frontend:} Čte data z databáze, filtruje na základě
        uživatelského vstupu geolokace a zvoleného okruhu a filtrovaná
        data následně porovnává a řadí ve výchozím nastavení od nejlevnějších
        po nejdražší. Využit je framework \emph{Next.js} s použitím API
        mapového poskytovatele OSM.
    \item \textbf{Databáze:} Uchovává nasbíraná data z crawleru a poskytuje je
        frontendu k uživatelskému zpracování. Využit je software \emph{Oracle MySQL}.
    \item \textbf{Reverse proxy:} Zprostředkovává přístup k frontendu z webového
        prohlížeče uživatele a zajišťuje kryptografické šifrování komunikace
        mezi prohlížečem a serverem. Využit je software \emph{Nginx}.
\end{itemize}

\section{Crawler}

Implementace crawleru je řešena pomocí \emph{Node.js}, který poskytuje funkční
rozhraní pro spouštění aplikace, s využitím knihoven \emph{Crawlee},
\emph{Playwright} a \emph{MySQL}, které poskytují rozhraní nutné k procházení
webů a zapisování do databáze. Nejdůležitější součástí crawleru jsou dílčí
moduly, které implementují procházení webů. Technický postup automatizace
tohoto procházení a čtení dat je popsán v následujících podsekcích.

\subsection{Procházení Globusu}

\begin{enumerate}
    \item Po načtení webové stránky se crawler snaží identifikovat tlačítko
        souhlasu s využitím cookies. Tlačítko souhlasu je popsáno CSS selektorem\\
        \texttt{button\#CybotCookiebotDialogBodyLevelButtonLevelOptinAllowAll}.
        Pokud není toto tlačítko nalezeno, procedura se přeskočí.
    \item Nyní je třeba zjistit seznam jednotlivých poboček kliknutím na tlačítko
        \emph{Vybrat pobočku}. Toto tlačítko se nachází v hlavičce stránky,
        popsáno je CSS selektorem \texttt{div\#\_\_nuxt header\#header button.btn-lg}.
    \item [...]
\end{enumerate}

\subsection{Procházení ČS ONO}

[...]

\section{Frontend}

[...]

\section{Ostatní součásti a jejich propojení}

[...]

\endinput

\chapter{Závěr}
\label{sec:conclusion}

Tvorba projektu byla poměrně náročná, nicméně velmi zajímavá.
Nejen, že~jsem využil něco z~toho, co již znám, ale~vyzkoušel jsem~si
i~něco nového, a~spoustu se~toho při~tvorbě této aplikace naučil.
Způsobů, jakým se~dají tvořit moderní webové stránky, je~nepřeberné
množství. Lze využít staré osvědčené metody, ale technologie a~trend
se~ubírají novým směrem a~posouvají~se kupředu. I~z~toho důvodu je~nutné
se~neustále vzdělávat a~držet krok s~dobou. Je to jedno z~mála, co~člověk
může udělat pro~to, aby~si~udržel přehled a~nezaostával za~ostatními.

Pokud bych měl zhodnotit splnění bodů, tak si~myslím, že~ačkoliv
se~při~rešerši a~vývoji vyskytly nečekané potíže, dokázal jsem~si s~nimi
poradit a~projekt dotáhnout zdárně do~úspěšného konce.

Musím zdůraznit jednu věc: Tento projekt je~pouze demonstrační.
Účelem bylo vyvinout webovou aplikaci, která pravidelně aktualizuje
ceny PHM, a~na~základě zadané geografické lokace je~porovnává, filtruje
a~nabízí uživateli. Pro~demonstraci byly nakonec vybrány 2~sítě čerpacích
stanic, ze~kterých jsou data získávána. Původním záměrem bylo sice použití
8~ČS (následně se~počet rozšířil na~9), ale při zkoumání dostupnosti dat
jsem zjistil, že~pouze 3~z~těchto čerpacích stanic nabízí data v~rozumném
formátu a~nakonec jsem zjistil během vývoje, že~jen 2~čerpací stanice
lze reálně použít. Problém by~nastal, kdyby dostupná byla jen 1~ČS.
Tím, že~dostupné jsou 2, je~porovnání cen možné. A~to~je základní předpoklad
k~funkční aplikaci porovnávače cen.

\begin{enumerate}
    \item \textbf{Motivace a~přehled podobných projektů.}
        
        Tato část zkoumá hnací impuls, který mě~vedl k~výběru tohoto
        tématu a~také podobné projekty, které jsou mému nápadu podobné.
        Popsal jsem tam webové stránky a~aplikace, které nabízí srovnání
        cen PHM na~ČS po~celé ČR nebo dokonce v~celé Evropě.

    \item \textbf{Technologie pro~návrh webových aplikací.}

        V~této části se~věnuji rozboru používaných technologií
        pro~návrh a~implementaci webových aplikací, kde~porovnávám
        jednotlivé knihovny a~frameworky, zhodnocuji jejich
        výhody a~nevýhody a~popisuji jejich typické využití.
        V~této části jsem využil umělé inteligence, konkrétně
        ChatGPT, k~rešerši. Tvrzení v~této kapitole jsou řádně
        citována, zdroj informací je~uveden jak z~ChatGPT, tak
        ze~zdrojů, které k~tvrzení umělá inteligence připojila.

    \item \textbf{Popis návrhu a~implementace.}

        Od~příprav, přes vývoj až~po~finální úpravy a~spuštění aplikace.
        Tato část popisuje postup tvorby jednotlivých dílčích součástí
        aplikace, jejich spojení, kontejnerizace a~následné finální
        spuštění na~VPS. Hlavním OS je~Rocky Linux, kde~je~nainstalován
        Nginx pro~provoz reverse proxy a~Docker, který obsahuje 3~kontejnery
        obsluhující jednotlivé části. Jeden kontejner je zaměřen
        na~databázi, další obsahuje crawler, který se~spouští periodicky
        každý den ve~3:00, a~poslední obsahuje webový frontend, který data
        z~databáze filtruje, řadí a~zobrazuje uživateli v~prohlížeči.
        Během vývoje aplikace jsem~si v~některých částech pomohl umělou
        inteligencí, konkrétně GitHub Copilotem, který mi~řešil obtížné
        chyby, jenž se~při~vývoji vyskytly, a~jejichž tradiční ladění
        by~znamenalo příliš velké prodloužení vývoje aplikace.

    \item \textbf{Zhodnocení dosažených výsledků.}

        Nebudu lhát, během vývoje došlo k~několika problémům, které
        finální podobu posunuly trochu jinak, než jsem původně očekával.
        Ve~všech případech si~však myslím, že~cíl bakalářské práce
        byl~splněn. Webová aplikace sice nevypadá tak, jak jsem si
        původně představoval, ale~základní funkcionalita byla dodržena.
        Crawler periodicky spouští procházení webu a aktualizuje databázi.
        Webový frontend nabízí uživateli možnost vyhledávání geografické
        lokace pomocí formuláře společně se~specifikací maximálního okruhu
        pro~porovnání, včetně možnosti změny řazení výsledků dle ceny
        (od~nejlevnějších nebo od~nejdražších) a~filtrování výsledků
        na~základě typu paliva či~jiného doplňkového produktu z~tankovacích
        stojanů či~jejich kvality. GUI webové aplikace je~sice minimální,
        ale~uživatelsky přívětivé a~přehledné.
\end{enumerate}

Ještě jednou pro~jistotu uvádím, že~jsem nad~rámec zadání bakalářské práce
po~dohodě s~vedoucím zařadil do~nabídky PHM k~porovnání také doplňkové
produkty dostupné z~tankovacích stojanů, které by~mohly řidiče zajímat.
Jedná~se zejména o~AdBlue (syntetickou močovinu) a~kapalinu do~ostřikovačů.
Tedy provozní kapaliny, které ze~své podstaty nejsou palivem, nicméně se~jedná
o~provozní kapaliny, které jsou nutné k~bezpečnému a~ekologickému provozu
automobilu.

Aplikaci by~samozřejmě šlo vylepšit a~případně přepracovat. Nabízí~se např.
využití umělé inteligence k~vyhledávání cen PHM a~dalších doplňkových
produktů a~služeb. Jiná možnost by~byla použití síly veřejnosti, která
by~mohla informace o~aktuálních cenách paliv a~doplňkových produktů a~služeb
doplňovat. Pokud jde o~prezentaci výsledků vyhledávání, tak více možností
pro~řazení, filtrování podle více typů a~kvalit produktů, apod. Geografické
určení uživatele by~mohlo případně probíhat i~geolokační službou prohlížeče.
Všechny tyto funkce jsou však již nad~rámec zadání této bakalářské práce
a~nejsou nezbytně nutné ke~splnění základních požadavků. Tyto možnosti však
lze nechat otevřené pro~případný výzkum v~rámci navazujícího studia
v~diplomové práci.

\endinput


% Seznam literatury
\printbibliography[title={Literatura}, heading=bibintoc]

% Prilohy
\appendix
\chapter{Screenshoty}

\begin{sidewaysfigure}
    \centering
    \includegraphics[width=\textwidth]{Figures/vmware_console.png}
    \caption{Konzole virtuálního serveru ve VMware Workstation Pro}
    \label{fig:vmware-workstation-pro}
\end{sidewaysfigure}

\begin{sidewaysfigure}
    \centering
    \includegraphics[width=\textwidth]{Figures/globus_vyber.jpg}
    \caption{Základní informace o prodejně Globus, včetně cen PHM
        přidružené ČS}
    \label{fig:globus-cs}
\end{sidewaysfigure}

\chapter{Instalace serveru a vývoj aplikace}

\begin{sidewaysfigure}
    \centering
    \includegraphics[width=\textwidth]{Figures/vmware_console.jpg}
    \caption{Konzole virtuálního serveru ve VMware Workstation Pro}
    \label{fig:vmware-workstation-pro}
\end{sidewaysfigure}

\newpage        %% BEGIN:LST

\lstinputlisting[label=src:docker-compose.yml,caption={Soubor \texttt{docker-compose.yml} k sestavení Docker kontejnerů}]{SourceCodes/docker-compose.yml}

\lstinputlisting[label=lst:npx-create-next,caption={Tvorba Next.js projektu},literate={√}{{\checkmark}}1]{SourceCodes/npx-create-next.log}

\lstinputlisting[label=src:pol0423-stu.vsb.cz.conf,caption={Soubor \texttt{/etc/nginx/conf.d/pol0423-stu.vsb.cz.conf}}]{SourceCodes/pol0423-stu.vsb.cz.conf}

\newpage

\lstinputlisting[label=src:snippets/ssl-params.conf,caption={Soubor \texttt{/etc/nginx/snippets/ssl-params.conf}}]{SourceCodes/snippets/ssl-params.conf}

\lstinputlisting[label=src:snippets/self-signed.conf,caption={Soubor \texttt{/etc/nginx/snippets/self-signed.conf}}]{SourceCodes/snippets/self-signed.conf}

\lstinputlisting[label=lst:net-modify,caption={Postup modifikace IPv6}]{SourceCodes/net-modify.log}

\newpage        %% END:LST

\begin{sidewaysfigure}
    \centering
    \includegraphics[width=\textwidth]{Figures/web-home-desktop.jpg}
    \caption{Hlavní stránka aplikace zobrazená v prohlížeči na PC}
    \label{fig:web-home-desktop}
\end{sidewaysfigure}

\begin{figure}
    \centering
    \includegraphics[width=0.5\linewidth]{Figures/web-home-mobile.jpg}
    \caption{Hlavní stránka aplikace v mobilním zobrazení}
    \label{fig:web-home-mobile}
\end{figure}

\begin{sidewaysfigure}
    \centering
    \includegraphics[width=\textwidth]{Figures/web-search-desktop.jpg}
    \caption{Stránka aplikace s výsledky vyhledávání na PC}
    \label{fig:web-search-desktop}
\end{sidewaysfigure}

\begin{figure}
    \centering
    \includegraphics[width=0.5\linewidth]{Figures/web-search-mobile.jpg}
    \caption{Stránka aplikace s výsledky vyhledávání v mobilním zobrazení}
    \label{fig:web-search-mobile}
\end{figure}

\begin{sidewaysfigure}
    \centering
    \includegraphics[width=\linewidth]{Figures/web-server.jpg}
    \caption{Stránka aplikace na serveru}
    \label{fig:web-server}
\end{sidewaysfigure}

\endinput

\chapter{Datové struktury}

\begin{longtable}{|l|c|c|c|l|} 
    \caption{Struktura tabulky petrolscan\_data} \label{tab:petrolscan_data-structure} \\
    \hline \multicolumn{1}{|c|}{\textbf{Sloupec}} & \multicolumn{1}{|c|}{\textbf{Typ}} & \multicolumn{1}{|c|}{\textbf{Prázdný}} & \multicolumn{1}{|c|}{\textbf{Výchozí}} & \multicolumn{1}{|c|}{\textbf{Komentáře}} \\ \hline \hline
\endfirsthead
    \caption{Struktura tabulky petrolscan\_data (pokračování)} \\ 
    \hline \multicolumn{1}{|c|}{\textbf{Sloupec}} & \multicolumn{1}{|c|}{\textbf{Typ}} & \multicolumn{1}{|c|}{\textbf{Prázdný}} & \multicolumn{1}{|c|}{\textbf{Výchozí}} & \multicolumn{1}{|c|}{\textbf{Komentáře}} \\ \hline \hline \endhead \endfoot 
    \textbf{\textit{id}} & int(11) & Ne &  \\ \hline 
    timestamp & timestamp & Ne & current\_timestamp() \\ \hline 
    station\_name & varchar(255) & Ne &  \\ \hline 
    station\_loc\_name & varchar(255) & Ne &  \\ \hline 
    station\_loc\_lat & double & Ne &  \\ \hline 
    station\_loc\_lon & double & Ne &  \\ \hline 
    fuel\_type & varchar(255) & Ne &  \\ \hline 
    fuel\_quality & varchar(255) & Ano & NULL \\ \hline 
    fuel\_name & varchar(255) & Ne &  \\ \hline 
    fuel\_price & float & Ne &  \\ \hline 
\end{longtable}

\endinput


\end{document}
